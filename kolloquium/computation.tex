
\section{Berechnung von Walras-Preisen}
\subsection{Informationszugang zum Markt}
\begin{frame}{Informationszugang zum Markt}
	Man unterscheidet die drei Sichtweisen:
	\begin{itemize}[label=\color{darkblue}$\bullet$]
		\pause\item \emph{Die Mikroskopische Sicht:} Ein \emph{Wert-Orakel} bestimmt anhand eines Käufers $i\in\first{m}$ und eines Bündels $x\in\ffirst{\fs}$ den Wert $v_i(x)$.
		\pause\item \emph{Die Agenten-Sicht:} Ein \emph{Nachfrage-Orakel} bestimmt anhand eines Käufers $i\in\first{m}$ und eines Preisvektors $\fp\in\R^n$ ein Bündel $x\in\NB(v_i, \fp)$.
		\pause\item \emph{Die Makroskopische Sicht:} Ein \emph{Aggregierte-Nachfrage-Orakel} bestimmt anhand eines Preisvektors $\fp\in\R^n$ eine aggregierte Nachfrage $\dem(\fp)$, für den $x^{(i)}\in\NB(v_i, \fp)$ für alle $i\in\first{m}$ existieren mit $\dem(\fp) = \sum_{i\in\first{m}} x^{(i)}$.
	\end{itemize}

\vspace{1em}

\pause
	\parbox{\textwidth}{\emph{Ziel:} Berechnung von Walras-Preisen mit dem Aggregierte-Nachfrage-Orakel.}
	
\vspace{1em}

\pause
	\parbox{\textwidth}{
		\emph{Vorgehen:} Berechnung mittels Ellipsoidmethode. Voraussetzungen:
		\begin{itemize}[label=\color{darkblue}$\bullet$]
			\item Beschränkung der Menge der Walras-Preise
			\item Konstruktion eines Trennorakels
			\item Abschätzungen für Laufzeitanalyse
		\end{itemize}
	}
\end{frame}

\subsection{Berechnung mittels Ellipsoidmethode}
\begin{frame}{Beschränkung der Walras-Preise}
\begin{lemma}
	Für $M\coloneqq \max_{i\in\first{m},x\in\ffirst{\fs}} \abs{v_i(x)}$ sind alle Walras-Preise in $[-2M, 2M]^n$ enthalten.
\end{lemma}
\end{frame}

\begin{frame}[t]{Konstruktion eines Trennorakels}
\pause\begin{definition}[Subgradient]
	Seien $f:\R^n \rightarrow \R$ eine konvexe Funktion.
	Man nennt $g\in\R^n$ einen \emph{Subgradient von $f$ an $x_0$}, falls $f(x)\geq f(x_0) + g \bcdot (x - x_0)$ für alle $x\in\R^n$ gilt.
\end{definition}
\def\svgwidth{\columnwidth}
	\import{kolloquium/}{subgradient.pdf_tex}
\end{frame}

\begin{frame}[t]{Konstruktion eines Trennorakels}
\addtocounter{theorem}{-1}
\vspace{0.2em}
\begin{definition}[Subgradient]
	Seien $f:\R^n \rightarrow \R$ eine konvexe Funktion.
	Man nennt $g\in\R^n$ einen \emph{Subgradient von $f$ an $x_0$}, falls $f(x)\geq f(x_0) + g \bcdot (x - x_0)$ für alle $x\in\R^n$ gilt.
\end{definition}
\vspace{-0.4em}
\begin{definition}[Marktpotenzialfunktion]
Die \emph{Marktpotenzialfunktion} ist definiert als
\[ f: \R^n \rightarrow \R, \quad \fp \mapsto \sum_{i\in\first{m}} \left( 
\max_{x\in\ffirst{\fs}} v_i(x) - \fp \bcdot x
\right) + \fp\bcdot \fs \]
\end{definition}
\vspace{-0.4em}
\pause \begin{lemma}
Für alle $\fp\in\R^n$ ist $\fs - \dem(\fp)$ ein Subgradient von $f$ an $\fp$.
\end{lemma}
\vspace{-0.4em}
\pause \begin{definition}[Trennorakel]
	Ein Trennorakel für eine konvexe Menge $K$ gibt gegeben eines Punktes $\fp\in\R^n$ entweder die Meldung \glqq $\fp\in K$\grqq\ oder eine Halbebene aus, die $\fp$ von $K$ trennt, also einen Vektor $g\in\R^n$ mit $g\bcdot \fp \leq g \bcdot \fp^*$ für alle $\fp^*\in K$.
\end{definition}
\end{frame}

\begin{frame}{Abschätzungen zur Analyse der Laufzeit}
	\pause
	\vspace{-1em}
	\begin{align*}
		\tag{D}
		&\min_{\fp,\fu} \sum_{i\in\first{m}} u_i + \fp \bcdot \fs \\
		\text{udN.} \quad &  u_i \geq v_i(x) - \fp \bcdot x & \text{für alle $i\in\first{m}, x\in\ffirst{\fs}$}
	\end{align*}
	\begin{lemma}[Darstellung als LP]
		Das folgende Programm ist äquivalent zu~\eqref{DP}:
		\vspace{5pt}
		\hrule
		\begin{align*}
		\tag{TD}\label{TDP}
		&\min_{\fp, u} u + \fp \bcdot \fs \\
		\text{udN.}\quad& u \geq \sum_{i\in\first{m}} \left( v_i(x^{(i)}) - \fp \bcdot x^{(i)} \right) & \text{für alle $\fx = (x^{(i)})_{i\in\first{m}}\in\ffirst{\fs}^m$}
		\end{align*}
	\end{lemma}
	\pause \begin{lemma}
		Die Walras-Preise bilden ein Polytop, dessen Ecken $\fp$ von der Form $p_j=a_j/b$ mit $a_j,b\in\Z$ und $\abs{b}\leq (n+1)!\, (mS)^n$ für $S\coloneqq\max_{j\in\first{n}} s_j$.
	\end{lemma}
\end{frame}

\begin{frame}{Berechnung von Walras-Preisen}
	\begin{theorem}[Ellipsoid-Methode]
		Sei ein Trennorakel einer konvexen Menge $K\subseteq B_r(\zero)$ gegeben.
		
		Mit $t$ Abfragen des Orakels lässt sich entweder ein Punkt $\fp\in K$ oder eine Ellipse $E$ mit $\vol(E) \leq \exp(-t/(2n+1)) \cdot \vol(B_r(\zero))$ und $K\subseteq E$ ermitteln.
	\end{theorem}
	\pause \begin{theorem}[Berechnung von Walras-Preisen]
		Gilt $\interior(K)\neq \emptyset$ für die Menge $K$ der Walras-Preise, kann man mit $\bigO(n^3\log(nmS) + n^2\log(M))$ Abfragen des Aggregierte-Nachfrage-Orakel einen Walras-Preisvektor bestimmen.
	\end{theorem}
\end{frame}