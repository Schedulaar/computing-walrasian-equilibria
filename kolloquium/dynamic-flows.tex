\section{Walras-Gleichgewichte}
\subsection{Grundlegende Definitionen}

\begin{frame}{Grundlegende Definitionen}
	\begin{notation}
		Für $k\in\Z_{\geq0}$ bezeichne $\first{k}\coloneqq \{ 1, 2, \dots, k \}$, $\ffirst{k}\coloneqq \{0, 1, \dots, k \}$.
		
		\pause
		Für $\fs=(s_1, \dots, s_n)\in\Z_{\geq0}^n$ bezeichne $\ffirst{\fs}\coloneqq \prod_{j\in\first{n}} \ffirst{s_j}$.
	\end{notation}
	\begin{definition}[Markt]
		Ein Markt besteht aus:
		\begin{itemize}[label=\color{darkblue}$\bullet$]
			\item einer Menge $\first{n}$ von Gütern sowie einem Angebot $s_j\in\Z_{>0}$ von jedem Gut $j\in\first{n}$,
			\item einer Menge $\first{m}$ von $m\geq 2$ Käufern mit je einer Bewertungsfunktion $v_i:\ffirst{\fs} \rightarrow \Z$ mit $v(\zero) = 0$ für $i\in\first{m}$.
		\end{itemize}
	\end{definition}
	\begin{definition}[Nachfragebereich]
		Bei Preisen $\fp\in\R^n$ ist $u_i(x; \fp)\coloneqq v_i(x) - \fp \bcdot x$ der \emph{Nutzen eines Bündels} $x\in\ffirst{\fs}$ für einen Käufer $i\in\first{m}$.
		
		Der \emph{Nachfragebereich} eines Käufers $i$ bei Preisen $\fp$ bezeichnet $\NB(v_i, \fp)\coloneqq \arg\max_{x\in\fs} u_i(x; \fp)$.
	\end{definition}
\end{frame}

