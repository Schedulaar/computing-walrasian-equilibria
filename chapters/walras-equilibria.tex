\section{Walras-Gleichgewicht}

\newcommand{\first}[1]{\left[ #1 \right]}
\newcommand{\ffirst}[1]{\left\llbracket #1 \right\rrbracket}
\newcommand{\Z}{\mathbb{Z}}
\newcommand{\zero}{\mathbf{0}}
\newcommand{\fp}{\mathbf{p}}
\newcommand{\fx}{\mathbf{x}}
\newcommand{\fy}{\mathbf{y}}
\newcommand{\fs}{\mathbf{s}}
\newcommand{\fu}{\mathbf{u}}


\makeatletter
\newcommand*\bcdot{\mathpalette\bcdot@{.5}}
\newcommand*\bcdot@[2]{\mathbin{\vcenter{\hbox{\scalebox{#2}{$\m@th#1\bullet$}}}}}
\makeatother

Um Walras-Gleichgewichte formal einführen zu können, werden zunächst einige grundlegende Begriffe und Notationen erklärt.

\begin{notation}
	Für $k\in\Z_{\geq 0}$ sei $\first{k}\coloneqq \{ 1, 2, \dots, k \}$ die Menge der ersten $k$ natürlichen Zahlen; um die $0$ hinzuzunehmen, schreibt man $\ffirst{k}\coloneqq \{0,1,\dots, k \}$.
	Für einen Vektor $\fs = (s_1, \dots, s_n)\in\Z_{\geq0}^n$ definiert man $\ffirst{\fs}\coloneqq \Pi_{j\in\first{n}} \ffirst{s_j}$.
\end{notation}

Das Modell des Marktes besteht hier aus einer Menge $\first{m}$ von $m\geq 2$ \emph{Käufern}, einer Menge $\first{n}$ von \emph{Gütern} sowie aus einem \emph{Angebot} $s_j\in\Z_{>0}$ für jedes Gut $j\in\first{n}$.

Jedem Käufer $i\in\first{m}$ ist eine Bewertungsfunktion $v_i:\Z^n_{\geq 0} \rightarrow \Z$ zugeordnet, die einem $\emph{Bündel}$, also einer Multimenge an Gütern aus $\first{m}$, einen ganzzahligen Wert zuschreibt, wobei $v_i(\zero) = 0$ gilt.

Sind ein \emph{Preisvektor} $\fp\in\R^n$ und ein Bündel $x\in\Z^{m}_{\geq0}$ gegeben, bezeichnet $u_i(x;\fp) \coloneqq v_i(x) - \fp \bcdot x$ den Nutzen von Bündel $x$ bei Preisen $\fp$ für Käufer $i$.
Hierbei ist $\fp \cdot x$ das Skalarprodukt von $\fp$ und $x$.

\begin{definition}[Allokation]
	Eine \emph{Allokation} $\fx\coloneqq (x^{(i)})_{i\in\first{m}}$ weist jedem Käufer $i\in\first{m}$ ein Bündel $x^{(i)} \in\Z^n_{\geq0}$ zu.
	Ist ein Angebotsvektor $\fs\in\Z^m_{\geq0}$ gegeben, nennt man eine Allokation \emph{gültig}, falls sie genau das Angebot verteilt, das heißt, falls $\sum_{i\in\first{m}} x^{(i)} = \fs$ gilt.
	Alle gültigen Allokationen werden in der Menge $\FA$ gesammelt.
	
	Das \emph{soziale Wohl} einer gültigen Allokation $\fx$ ist definiert als $\SW(\fx) \coloneqq \sum_{i\in\first{m}} v_i(x^{(i)})$.
	Eine gültige Allokation mit maximalem sozialen Wohl wird \emph{optimale Allokation} genannt.
\end{definition}

\begin{definition}[Nachfragebereich]
	Man nennt die Menge $\NB(v,\fp)$ der Bündel, die den Nutzen unter einer Bewertungsfunktion $v:\Z^n_{\geq 0} \rightarrow \Z$ bei Preisen $\fp$ maximiert, den Nachfragebereich von $v$ bei Preisen $\fp$.
	Es ist also $ \NB(v, \fp) \coloneqq \arg\max_{x\in\ffirst{\fs}} v(x) - \fp \bcdot x$.
	Für den Nachfragebereich eines Käufers $i\in\first{m}$ wird abkürzend $\NB(i,p):=\NB(v_i, p)$ geschrieben.
\end{definition}

\begin{definition}[Walras-Gleichgewicht]
	Ein Paar $(\fx, \fp)$ bestehend aus einer gültigen Allokation $\fx$ und einem Preisvektor $\fp$ heißt \emph{(Walras-)Gleichgewicht}, falls jedem Käufer ein Bündel aus seinem Nachfragebereich zugewiesen wird, falls also $x^{(i)} \in \NB(i, \fp)$ für alle $i\in \first{m}$ gilt.
	Dabei nennt man $\fp$ einen \emph{Walras-Preisvektor} und $\fx$ eine von $\fp$ induzierte \emph{Walras-Allokation}.
\end{definition}

\begin{bemerkung}
	Im Vergleich zu~\cite{PaesLeme2018} wird hier das Angebot $s_j$ positiv statt nicht-negativ gewählt und es werden mindestens zwei Käufer vorausgesetzt, um den Beweis von Lemma~\ref{prices-bounded} bzw. von~\cite[Lemma 4]{PaesLeme2018} zu ermöglichen.
	Die ausgeschlossenen Fälle sind jedoch uninteressant.
\end{bemerkung}

In einem Walras-Gleichgewicht gibt es also für keinen der Käufer eine für ihn bessere Allokation: Das Bündel, das dem Käufer zugeteilt wird, hat bei den gegebenen Preisen einen für ihn maximalen Nutzen.
Des Weiteren wird durch Gültigkeit der Allokation in einem solchen Gleichgewicht sichergestellt, dass das Angebot und die Nachfrage des gesamten Marktes übereinstimmen: Es bleiben also weder Güter übrig, noch wird die Nachfrage irgendeines Käufers nicht gedeckt.

Darüber hinaus gelten hier die sogenannten Wohlfahrtstheoreme aus der Ökonomik:
Das erste Wohlfahrtstheorem besagt, dass jedes Gleichgewicht bei vollkommenen Wettbewerb, wie er hier unter den Käufern möglich ist, das soziale Wohl maximiert.
Das zweite Wohlfahrtstheorem sagt aus, dass jede optimale Allokation bei Walras-Preisen ein Gleichgewicht erzeugt.

\begin{lemma}[Erstes und zweites Wohlfahrtstheorem]
	Ist $(\fx, \fp)$ ein Gleichgewicht, so ist $\fx$ eine optimale Allokation.
	Ist $\fy$ eine beliebige optimale Allokation, so bildet auch $(\fy, \fp)$ ein Gleichgewicht.
\end{lemma}
\begin{proof}
	Seien $(\fx, \fp)$ ein Gleichgewicht und $\fy$ eine gültige Allokation.
	Für alle $i\in\first{m}$ gilt wegen $x^{(i)} \in \NB(i,\fp)$ dann $v_i(x^{(i)}) - \fp \bcdot x^{(i)} \geq v_i(y^{(i)}) - \fp \bcdot y^{(i)}$.
	Mit der Gültigkeit der Allokationen $\sum_{i\in\first{m}} x^{(i)} = \sum_{i\in\first{m}} y^{(i)} = \fs$ folgere man
	\[
		\sum_{i\in\first{m}} v_i(x^{(i)}) \geq
		\sum_{i\in\first{m}} \left( v_i(y^{(i)}) - \fp \bcdot (y^{(i)} - x^{(i)}) \right)
		= \sum_{i\in\first{m}} v_i(y^{(i)}) - \fp \bcdot (\fs - \fs) = \sum_{i\in\first{m}} v_i(y^{(i)}).
	\]
	Insbesondere ist $\fx$ also eine optimale Allokation.
	
	Ist nun $\fy$ ebenfalls optimal, gilt 
	$ \sum_{i\in\first{m}} v_i(x^{i}) = \sum_{i\in\first{m}} v_i(y^{(i)}) $ und mit der Gültigkeit der Allokationen lässt sich
	\[
		\sum_{i\in\first{m}} \left( v_i(x^{i}) - \fp \bcdot x^{(i)} \right)
		= \sum_{i\in\first{m}} \left( v_i(y^{(i)}) - \fp \bcdot y^{(i)} \right)
	\]
	folgern.
	Da $x^{(i)}\in\NB(i, \fp)$ für die Summanden bereits $v_i(x^{(i)}) - \fp \bcdot  x^{(i)} \geq v_i(y^{(i)}) - \fp \bcdot y^{(i)}$ impliziert, müssen diese bereits exakt übereinstimmen.
	Dadurch folgt auch $y^{(i)}\in\NB(i, \fp)$ für alle $i\in\first{m}$, sodass $\fy$ ebenfalls eine Walras-Allokation zu Preisen $\fp$ ist.
\end{proof}

