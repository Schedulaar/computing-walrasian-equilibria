\section{Brutto-Substituts-Bewertungen}\label{sec-gross-substitutes}

In diesem Abschnitt werden die Bewertungsfunktionen der Käufer eingeschränkt.
In vielen Marktmodellen werden nur \glqq ähnliche\grqq\ Güter betrachtet.
Diesen unterstellt man, dass bei steigenden Preisen des einen Guts ein anderes Gut dieses substituieren kann.
Insbesondere geht man davon aus, dass, sollten die Preise anderer Güter steigen, die Nachfrage der Güter mit gleichbleibendem Preis nicht sinkt.

\iffalse
\begin{definition}[Diskret-Konkave Funktion]
	Eine Funktion $v: \ffirst{\fs}^n \rightarrow \Z$ heißt \emph{diskret-konkav}, falls lokale Minima auch global minimal sind, also falls für alle Preise $\fp\in\R^n$ und Bündel $x\in\ffirst{\fs}^n$ mit \begin{align*}
	&v(x) \geq \max_{i : x_i>0} v(x - e_i) + p_i, \qquad
	&v(x) \geq \max_{j:x_j < s_j} v(x + e_j) - p_j, \\
	&v(x) \geq \max_{\substack{i: x_i>0 \\ j: x_j < s_j}} v(x + e_i - e_j) - p_i + p-j
	\end{align*}
	bereits $x\in\NB(v, \fp)$ gilt.
\end{definition}
\fi

\begin{definition}[Brutto-Substituts-Bewertung]
Eine Bewertungsfunktion $v: \ffirst{s} \rightarrow \Z$ heißt \emph{Brutto-Substituts-Bewertung} (engl. gross substitutes valuation), falls es für Preise $\fp,\fp' \in\R^n$ mit $\fp' \geq \fp$ und für ein nachgefragtes Bündel $x\in\NB(v, \fp)$ zu Preisen $\fp$ ein nachgefragtes Bündel $y\in\NB(v,\fp')$ zu Preisen $\fp'$ existiert, welches $y_j \geq x_j$ für alle $j\in\first{n}$ mit $p_j = p_j'$ erfüllt.
\end{definition}

Einige wichtige Resultate über Brutto-Substituts-Bewertungen wurden nur für den Fall gezeigt, dass jedes jedes Gut genau einmal angeboten wird, also für $s_j = 1$ für alle $j\in\first{n}$.
Die Theorie von Brutto-Substituts-Bewertungen wurde für diesen Fall maßgeblich von Kelso und Crawford in~\cite{KelsoCrawford} entwickelt.
Man kann den Fall mehrfachen Angebots einzelner Güter darauf herunterbrechen, indem man jedes Vorkommen eines Guts unabhängig bepreisen lässt:
\newcommand{\tild}[1]{\widetilde{#1}}
\begin{definition}[Unabhängige Brutto-Substituts-Bewertung]
	Eine Bewertungsfunktion $v: \ffirst{s} \rightarrow \Z$ heißt \emph{unabhängige Brutto-Substituts-Bewertung}, falls \[
	\tild{v}: \ffirst{1}^{N_\fs} \rightarrow \Z,
	\quad (x_{i,j})_{(i,j)\in N_\fs} \mapsto v\left( \left( \sum_{j\in\first{s_i}} x_{i,j} \right)_{i\in\first{n}} \right)
	\]
	eine Brutto-Substituts-Bewertung ist, wobei $N_\fs \coloneqq \{ (i,j) \mid i\in\first{n}, j\in\first{s_i} \}$ gilt.
\end{definition}
\begin{proposition}
	Jede unabhängige Brutto-Substituts-Funktion ist eine Brutto-Substituts-Funktion.
	Es gibt aber Brutto-Substituts-Funktionen, die nicht unabhängig sind.
\end{proposition}
\begin{proof}
	Für Preise $\fp,\fp'\in\R^n$ mit $\fp' \geq \fp$ definiere man $\tild{\fp}\coloneqq (p_i)_{i\in\first{n},j\in\first{s_i}}$ und analog $\tild{\fp'}\coloneqq (p_i')_{i\in\first{n},j\in\first{s_i}}$.
	Für ein Bündel $x\in\NB(v,\fp)$ ist $\tild{x}$ mit $\tild{x}_{i,j} = 1$ für $j\in\first{x_i}$ und $\tild{x}_{i,j}= 0$ für $j > x_i$ in $\NB(\tild{v}, \tild{\fp})$ enthalten.
	Da $\tild{v}$ eine Brutto-Substituts-Bewertung ist, existiert ein Bündel $\tild{y}$ in $\NB(\tild{v}, \tild{\fp'})$ mit $\tild{y}_{i,j} \geq \tild{x}_{i,j}$ für alle $(i,j)\in N_\fs$ mit $p_i = p_i'$.
	Dementsprechend ist $y\coloneqq(\sum_{j\in\first{s_i}} \tild{y}_{i,j})_{i\in\first{n}}$ in $\NB(v, \fp')$ und es gilt $y_i \geq x_i$ für alle $i\in\first{n}$, sodass $v$ die Brutto-Substituts-Eigenschaft erfüllt.
	
	Nun ein Beispiel einer nicht unabhängigen Brutto-Substituts-Funktion: Wir betrachten einen Markt mit nur einem Gut mit $s=2$ und der Bewertungsfunktion $v(k)\coloneqq k^2$, welche automatisch eine Brutto-Substituts-Funktion ist.
	Setzt man Preise $\tild{p}_j \coloneqq 2$ für $j\in\first{2}$ fest,
	so ist der Nutzen beim Kauf beider Einheiten maximal; es gilt $(1,1)\in \NB(\tild{v}, \tild{\fp})$.
	Erhöht man den Preis der ersten Einheit auf $\tild{p}'_1 \coloneqq 3$ und behält den Preis der zweiten Einheit bei, so ist $(0,0)$ das einzige nachgefragte Bündel.
	Daher ist $\tild{v}$ keine Brutto-Substituts-Funktion.
\end{proof}

\begin{bemerkung}
	In~\cite{PaesLeme2018} wurden Brutto-Substituts-Bewertungen für allgemeine Angebote $\ffirst{\fs}$ direkt als unabhängige Brutto-Substituts-Bewertungen definiert.
	In~\cite{AkiyoshiShioura2015} wird diese stärkere Version als \glqq strong gross-substitute-condition $(SS)$\grqq\ bezeichnet.
\end{bemerkung}

\subsection{$\mnath$-Konkave und Matroidale Funktionen}\label{section-m-concavity}

Eine elementare Charakterisierung von Brutto-Substituts-Bewertungen findet sich in der diskreten Analysis.
So entspricht das Konzept der $\mnath$-Konkavität, also einer diskreten Version von Konkavität, in gewisser Weise der Brutto-Substituts-Eigenschaft.

Dazu wird hier die Beschreibung der Eigenschaft von Murota aus~\cite[Theorem~6.2 bzw.~Abschnitt~11.3]{Murota2003} eingeführt.

\begin{definition}
	Es bezeichne $\dom(v)\coloneqq \{ z \mid v(z)\in\R \}$ die \emph{Domäne}, also den reellwertigen Bereich, einer Funktion $v:\Z^n \rightarrow \R \cup \{ -\infty\}$.
\end{definition}

In dieser Arbeit werden meist Funktionen $v: D\rightarrow \R$ mit $D\subseteq \Z^n$ betrachtet.
Um die folgende Definition anwenden zu können, werden solche Funktionen $v$ hier als Funktionen $v: \Z^n\rightarrow \R \cup \{ -\infty\}$ interpretiert, wobei $v(x) = -\infty$ für $x\notin D$ gilt.
Hier gilt $D = \dom(v)$.

\begin{definition}[$\mnath$-Konkavität]
	Eine Funktion $v:\Z^n \rightarrow \R \cup \{-\infty\}$ mit $\dom(v) \neq \emptyset$ heißt \emph{$\mnath$-konkav} (\glqq M-natürlich konkav\grqq), falls \[
	v(x) + v(y) \leq
	\max \left\{
		v(x - e_i) + v(y + e_i),
		\max_{j: x_j < y_j} v(x - e_i + e_j) + v(y + e_i - e_j)
	\right\}
	\] für alle $x,y\in\dom(f)$ und $i\in\first{n}$ mit $x_i > y_i$ gilt.
	Hier gelte $\max(\emptyset)=-\infty$.
\end{definition}


Murota gibt in~\cite[Theorem~6.24 bzw. Abschnitt~11.3]{Murota2003} eine äquivalente Beschreibung von $\mnath$-konkaven Funktionen, die für die Anwendung hier sehr nützlich erscheint:

\begin{theorem}\label{thm-char-m-concave}
	Eine Funktion $v:\Z^n \rightarrow \R \cup \{ -\infty \}$ ist genau dann \emph{M}$^\natural$-konkav, wenn für alle $\fp\in\R^n$ und $x,y\in\dom(v)\neq\emptyset$ die Ungleichung $v(x) - \fp\bcdot x < v(y) - \fp \bcdot y$ bereits \[ 
		v(x) - \fp\bcdot x < 
			\max_{i:\, x_i > y_i \,  \vee \, i = 0} \,\,
				\max_{j:\, x_j < y_j \, \vee \, j=0} \,\,
					v(x - e_i + e_j) - \fp \bcdot (x - e_i + e_j)
	\]
	impliziert.
\end{theorem}

Ein wichtiges Resultat, das die Brutto-Substituts-Eigenschaft mit $\mnath$-Konkavität in Beziehung stellt, liefert~\cite[Theorem 2.1]{Fujishige2003}:

\begin{theorem}\label{thm-fujishige-gs-iff-concave}
	Eine Bewertungsfunktion $\tild{v}:\ffirst{1}^n\rightarrow \Z$ erfüllt genau dann die Brutto-Substituts-Eigenschaft, wenn sie $\mnath$-konkav ist.
\end{theorem}

Dies impliziert eine wichtige Eigenschaft unabhängiger Brutto-Substituts-Bewertungen:

\begin{korollar}\label{cor-independent-local-global}
	Ist $v$ eine unabhängige Brutto-Substituts-Funktion $v:\ffirst{\fs} \rightarrow \Z$, so sind für alle Preise $\fp$ lokale Maxima von $x\mapsto v(x) - \fp \bcdot x$ bereits global maximal.
	Dabei heißt ein Bündel $x$ lokales Maximum von $f$, falls $f(x)\geq \max_{i,j\in\ffirst{n}} f(x - e_i + e_j)$ gilt.
\end{korollar}
\begin{proof}
	Nach Theorem~\ref{thm-fujishige-gs-iff-concave} ist $\tild{v}$ $\mnath$-konkav.
	Seien Bündel $x,y\in\ffirst{\fs}$ und Preise $\fp\in\R^n$ mit $v(x)\geq \max_{i,j\in\ffirst{n}} v(x - e_i + e_j) + \fp \bcdot(-e_i + e_j)$ gegeben und man nehme an, es gelte $v(x) - \fp \bcdot x < v(y) - \fp\bcdot y$.
	Man definiere das Bündel $\tild{x}\in\ffirst{1}^{N_\fs}$ mit $\tild{x}_{i,j} \coloneqq 1$ für $j\in\first{x_i}$ und $\tild{x}_{i,j} \coloneqq 0$ für $x_i < j \leq s_i$ sowie analog das Bündel $\tild{y}$ und die Preise $\tild{p}_{i,j}\coloneqq p_i$ für alle $i\in\first{n}$ und $j\in\first{s_i}$.
	Nach Theorem~\ref{thm-char-m-concave} existieren $i,j\in N_\fs \cup \{ 0 \}$ mit \[
	v(\tild{x}) - \tild{\fp} \bcdot \tild{x} < \tild{v}(\tild{x} - e_i + e_j) - \tild{\fp}\bcdot (\tild{x} - e_i + e_j) .\]
	Entsprechend existieren $k,l\in\first{n}\cup\{ 0 \}$ mit $ v(x) - \fp\bcdot x < v(x - e_k + e_l) - \fp \bcdot (x - e_k + e_l)$, was der lokalen Maximalität von $x$ widerspricht.
\end{proof}
\begin{bemerkung}
	Nach~\cite[Theorem~11.5]{Murota2003} erfüllt jede $\mnath$-konkave Funktion die Brutto-Substituts-Eigenschaft.
	Die umgekehrte Aussage gilt nur, falls die Funktion konkav-er\-weiter\-bar ist.
	Ist die Domäne $\dom(v)\subseteq \Z^n_+$ einer konkav-erweiterbaren Funktion $v$ beschränkt, so ist $v$ nach~\cite[Theorem~4.1]{AkiyoshiShioura2015} sogar genau dann unabhängige Brutto-Substituts-Bewertung, wenn sie $\mnath$-konkav ist.
	Falls unabhängige Brutto-Substituts-Bewer\-tungen mit Domäne $\ffirst{\fs}$ bereits automatisch konkav-erweiter\-bar wären, so würde im Szenario dieser Arbeit bereits Äquivalenz gelten.
	Dies bleibt an dieser Stelle jedoch offen.
\end{bemerkung}

Wie sich zeigt, ist $\mnath$-Konkavität nicht die einzige interessante Charakterisierung von Brutto-Substituts-Funktionen.
So lässt sich auch das Konzept von sogeannten matroidalen Funktionen wiederfinden:
\begin{definition}[Matroidale Funktion]
	Eine Funktion $v:\ffirst{\fs}^n\rightarrow \Z$ heißt \emph{matroidal}, falls
	der Greedy-Algorithmus für alle Preise $\fp\in\R^n$ ein Bündel aus $\NB(v, \fp)$ berechnet.
	\begin{description}
		\item[Greedy-Algorithmus:] Initialisiere das Bündel $x$ mit $x\leftarrow \zero$.
		Solange ein $i\in\first{n}$ existiert, das $x_i < s_i$ und $v(x + e_i) - p_i > v(x)$ erfüllt,
		aktualisiere $x$ mit $x\leftarrow x + e_{i^*}$, wobei $i^*$ ein Index mit $x_{i^*}<s_{i^*}$ sei, der $v(x + e_{i^*}) - p_{i^*}$ maximiert.
	\end{description}
\end{definition}

In~\cite[Theorem~3.2]{PaesLeme2017} wird nun folgende Äquivalenz bewiesen:
\begin{theorem}\label{thm-matroidal-iff-single-unit-gs}
	Eine Funktion $\tild{v}:\ffirst{1}^n \rightarrow \Z$ ist genau dann matroidal, wenn sie eine Brutto-Substituts-Bewertung ist.
\end{theorem}
Dieses Ergebnis lässt sich auf unabhängige Brutto-Substituts-Bewertungen erweitern.
Außerdem kann man in diesem Fall die Simulation eines Aggregierte-Nachfrage-Orakels durch ein Wert-Orakel deutlich beschleunigen:
Allgemein benötigt man für eine Abfrage des Aggregierte-Nachfrage-Orakels nach Abschnitt~\ref{section-market-access} $m\cdot\abs{\ffirst{\fs}}$ Abfragen des Wert-Orakels.
\begin{korollar}\label{cor-indep-gs-matroidal}
	Eine unabhängige Brutto-Substituts-Bewertung ist matroidal.
	Insbesondere kann man mit $mn^2S$ Abfragen eines Wert-Orakels einen aggregierten  Nachfrage-Vektor berechnen, falls alle Käufer eine unabhängige Brutto-Substituts-Bewertung haben.
\end{korollar}
\begin{proof}
	Für eine unabhängige Brutto-Substituts-Funktion $v:\ffirst{\fs} \rightarrow \Z$ ist $\tild{v}$ nach Theorem~\ref{thm-matroidal-iff-single-unit-gs} matroidal.
	Zu Preisen $\fp$ kann also mit dem Greedy-Algorithmus ein Vektor $\tild{x}\in\NB(\tild{v}, \tild{\fp})$ mit $\tild{\fp} \coloneqq (p_i)_{(i,j)\in N_\fs}$ berechnet werden.
	Dieser kann mit $x=(\sum_{j\in\first{s_i}} x_j)_{i\in\first{n}}$ in ein Nachfragebündel aus $\NB(v, \fp)$ transformiert werden.
	Es ist leicht zu sehen, dass der Greedy-Algorithmus -- führt man ihn auf der Eingabe $(v,\fp)$ aus -- diese Aggregation bereits während der Berechnung durchführt und das gleiche Ergebnis ausgibt.
	
	Um also ein nachgefragtes Bündel $\dem_i(\fp)$ zu berechnen, benötigt der Greedy-Algorithmus auf der Eingabe $(v, \fp)$ bis zu $\sum_{i\in\first{n}} s_i \leq nS$ Durchläufe der Schleife, welche selbst jeweils bis zu $n$ Auswertungen von $v$, also bis zu $n$ Abfragen des Wert-Orakels, tätigt.
	Für einen aggregierten Nachfrage-Vektor $d(\fp) = \sum_{i\in\first{m}} d_i(\fp)$ werden also maximal $mn^2S$ Abfragen benötigt.
\end{proof}


\subsection{Auswirkung auf Walras-Preise}

Mit den Resultaten aus Abschnitt~\ref{section-m-concavity} lässt sich nun folgende Eigenschaft über die Menge der Walras-Preise zeigen, falls alle Käufer eine unabhängige Brutto-Substituts-Bewertung haben:

\begin{theorem}\label{thm-walras-prices-integral-polytope}
	Haben alle Käufer unabhängige Brutto-Substituts-Bewertungen, so sind alle Ecken des Zulässigkeitsbereichs von~\eqref{TDP} ganzzahlig.
	Insbesondere ist in dem Fall die Menge der Walras-Preise ein ganzzahliges Polytop.
\end{theorem}
Der Beweis dieses Theorems stellt eine zulässige Lösung durch Rundung als Konvexkombination ganzzahliger Lösungen dar.
Dafür sind folgende Notation und Proposition hilfreich:
\begin{notation}
	Der Nachkommaanteil einer Zahl $a\in\R$ sei notiert als $\fraction(a)\coloneqq a - \floor{a}$.
\end{notation}
\begin{proposition}\label{prop-integral-polytop-helper}
	Seien $p_1,p_2\in\R$ gegeben und sei $\theta$ zufällig aus dem Intervall $[0,1]$ gewählt.
	Setzt man $\hat{p_i}\coloneqq \ceil{p_i}$ für $\fraction(p_i)> \theta$ und $\hat{p_i}\coloneqq \floor{p_i}$ für $\fraction(p_i) \leq \theta$, so erfüllt der Erwartungswert $\E[\hat{p}_i] = p_i$ für $i\in\first{2}$.
	Außerdem gilt $\hat{p}_1 - \hat{p}_2 \in \{ \floor{p_1 - p_2}, \ceil{p_1 - p_2} \}$.
\end{proposition}
\begin{proof}
	Ist $p_i$ ganzzahlig, so gilt $\hat{p_i} = p_i$.
	Für $p_i\notin\Z$ gilt $\ceil{p_i} = \floor{p_i} + 1$.
	Die Wahrscheinlichkeit für $\hat{p_i} = \ceil{p_i}$ ist $\fraction(p_i)$ und die Wahrscheinlichkeit für $\hat{p_i} = \floor{p_i}$ ist $(1-\fraction(p_i))$.
	Dementsprechend gilt für den Erwartungswert: \begin{align*}
	\E(\hat{p_i})
	&= \fraction(p_i) \cdot \ceil{p_i} + (1-\fraction(p_i))\cdot \floor{p_i}
	= \fraction(p_i)\cdot (\floor{p_i} + 1) + (1-\fraction(p_i)) \floor{p_i} \\
	&= \fraction(p_i) + \floor{p_i} = p_i.
	\end{align*}
	
	Die zweite Aussage lässt sich in drei Fälle aufgeteilt zeigen:
	\begin{description}
		\item[1. Fall:] Die Zahlen $p_1$ und $p_2$ werden beide auf- oder beide abgerundet.
		Bei Aufrundung gilt $\hat{p_1} - \hat{p_2} = \ceil{p_1 - p_2}$ für $\fraction(p_1) \geq \fraction(p_2)$ und $\hat{p_1} - \hat{p_2} = \floor{p_1 - p_2}$ sonst.
		Bei Abrundung ist dies umgekehrt.
		\item[2. Fall:] Es gilt $\fraction(p_1) \leq \theta < \fraction(p_2)$.
		Dann ist $p_2$ nicht ganzzahlig und es gilt die Gleichung $\hat{p_1} - \hat{p_2} = \floor{p_1} - \ceil{p_2} = \floor{p_1} - \floor{p_2} - 1 = \floor{p_1 - p_2}$.
		\item[3. Fall:] Es gilt $\fraction(p_2) \leq \theta < \fraction(p_1)$. Analog gilt $\hat{p_1} - \hat{p_2} = \floor{p_1} + 1 - \floor{p_2} = \ceil{p_1 - p_2}$.
	\end{description}
	\vspace{-0.5em}
\end{proof}
%Diese Aussagen lassen nun den folgenden Beweis zu:
\begin{proof}[Beweis von Theorem~\ref{thm-walras-prices-integral-polytope}]
	Sei $(u, \fp)$ ein zulässiger Punkt von~\eqref{TDP}.
	Das heißt es gilt $u \geq \sum_{i\in\first{m}} ( v(x^{(i)}) - \fp \bcdot x^{(i)})$ für alle $\fx = (x^{(i)})_{i\in\first{m}}\in\ffirst{\fs}^m$.
	Sei nun $\fx = (x^{(i)})_{i\in\first{m}}$ eine Allokation aus $\arg\max_{\fx\in\ffirst{\fs}^m} \sum_{i\in\first{m}} (v(x^{(i)}) - \fp \bcdot x^{(i)})$ und man definiere die Differenz $w \coloneqq u - \sum_{i\in\first{m}} (v_i(x^{(i)}) - \fp \bcdot x^{(i)})$.
	Man beachte, dass $w$ nicht negativ ist.
	
	Nun wird die folgende Zufallsverteilung definiert:
	Es wird ein Wert $\theta$ zufällig aus dem Intervall $[0,1]$ entnommen.
	Wie in Proposition~\ref{prop-integral-polytop-helper} wird $\hat{p_i}$ auf $\ceil{p_i}$ gesetzt, falls $\fraction(p_i)$ größer als $\theta$ ist, und sonst auf $\floor{p_i}$ für alle $i\in\first{m}$.
	Analog wird $\hat{w}$ definiert.
	Schließlich wird $\hat{u}\coloneqq\hat{w}+ \sum_{i\in\first{m}} (v_i(x^{(i)}) - \hat{\fp} \bcdot x^{(i)})$ gesetzt.
	Nach Proposition~\ref{prop-integral-polytop-helper} gilt mit der Linearität des Erwartungswerts auch $\E[(\hat{u},\hat{\fp})] = (u, \fp)$.
	Dabei ist der Erwartungswert tatsächlich eine Konvexkombination endlich vieler, genauer maximal $\abs{ \{ \fraction(w) \} \cup \{ \fraction(p_i) \mid i\in\first{n} \} } + 1$ vieler ganzzahliger Vektoren.
	Es genügt also zu zeigen, dass jeder dieser Vektoren für~\eqref{TDP} zulässig ist.
	
	Seien also ein $\theta\in [0,1]$ fest und $\hat{\fp}$ sowie $\hat{w}$ die entsprechend $\theta$ gerundeten Werte von $\fp$ und $w$.
	Es wird gezeigt, dass $x^{(i)}$ für alle $i\in\first{m}$ ein nachgefragtes Bündel von Käufer $i$ zu Preisen $\hat{\fp}$ ist.
	Mit $\hat{u}\geq \sum_{i\in\first{m}}(v_i(x^{(i)}) - \hat{\fp}\bcdot x^{(i)})$ wegen $\hat{w}\geq0$ folgt dann die Behauptung.
	
	Sei ein $i\in\first{m}$ gegeben. Wegen $x^{(i)}\in\NB(v_i, \fp)$ gilt $v_i(x^{(i)}) \geq v_i(x^{(i)} + e_j - e_k) - \fp \bcdot (e_j - e_k)$ für alle $j,k\in\ffirst{n}$.
	Diese Ungleichung bleibt bei Rundung von $\fp$ zu $\hat{\fp}$ erhalten, weil $\hat{\fp}\bcdot (e_j - e_k)$ nach Proposition~\ref{prop-integral-polytop-helper} in $\{ \floor{\fp\bcdot(e_j - e_k)}, \ceil{\fp\bcdot(e_j - e_k)} \}$ liegt und die rest\-lichen Terme der Ungleichung ganzzahlig sind.
	Daher ist $x^{(i)}$ ein lokales Maximum und nach Korollar~\ref{cor-independent-local-global} aufgrund der unabhängigen Brutto-Substituts-Eigenschaft von $v_i$ auch globales Maximum von $x\mapsto v(x) - \hat{\fp} \bcdot x$.
	Somit gilt $x^{(i)}\in\NB(v_i, \hat{\fp})$.
\end{proof}
\begin{theorem}
	Sei ein Markt mit unabhängigen Brutto-Substituts-Bewertungen und $\interior(K)\neq \emptyset$ für die Menge $K$ der Walras-Preise gegeben.
	Dann kann man entweder mit $\bigO(n^2\log(M))$ Abfragen des Aggregierte-Nachfrage-Orakels einen Walras-Preisvektor oder mit $\bigO(n^4 m S\log(M))$ Abfragen des Wert-Orakels ein Walras-Gleichgewicht bestimmen.
\end{theorem}
\begin{proof}
	Hier kann man den gleichen Beweis wie in Theorem~\ref{thm-compute-walras-with-ellipsoid} führen; jedoch kann hier statt der Abschätzung der Nenner durch Lemma~\ref{lemma-prices-bounded-S} die Ganzzahligkeit der Ecken von $K$ nach Theorem~\ref{thm-walras-prices-integral-polytope} genutzt werden.
	Daher kann $R=1$ gewählt werden.
	
	Nutzt man statt des Aggregierte-Nachfrage-Orakels ein Wert-Orakel, so kann man nach Korollar~\ref{cor-indep-gs-matroidal} das gleiche bei $mn^2S$ so vielen Abfragen erreichen.
	Allerdings hat dies den Vorteil, dass aus folgendem Grund sogar eine Walras-Allokation berechnet wird:
	Das Trennorakel liefert bei der Ellipsoid-Methode schließlich die Meldung \glqq$\fp\in K$\grqq\ für Preise $\fp\in\R^n$, was nach dem Beweis von Theorem~\ref{thm-compute-walras-with-ellipsoid} nur erfolgt, wenn $\dem(\fp) - \fs = 0$ gilt.
	Da der Nachfrage-Vektor als Summe $\sum_{i\in\first{m}} \dem_i(\fp)$ berechnet wurde, bildet $(\dem_i(\fp))_{i\in\first{m}}$ eine Walras-Allokation.
\end{proof}