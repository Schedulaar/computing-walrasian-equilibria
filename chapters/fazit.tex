\section{Fazit und Ausblick}

Die Autoren Leme und Wong haben in ihrer Arbeit einen Einblick in die Berechnung von Walras-Gleichgewichten gegeben.
So haben sie gezeigt, dass man Walras-Preise mit Hilfe eines Subgradientenverfahrens exakt und effizient mit einem Aggregierte-Nachfrage-Orakel berechnen kann.
Außerdem haben sie gezeigt, dass die Walras-Preise bei Brutto-Substituts-Bewertungen ein ganzzahliges Polytop darstellen, was wiederum ermöglicht, die Berechnung zu beschleunigen.

Die Autoren beschränken sich nicht nur auf die hier erwähnte Subgradientenmethode, sondern geben in~\cite[Abschnitt~7]{PaesLeme2018} auch einen kombinatorischen Algorithmus zur Bestimmung von Walras-Gleichgewichten bei monotonen
 Brutto-Substituts-Bewertungen.
 Auch hier spielen die Zusammenhänge zur $\mnath$-Konkavität von Brutto-Substituts-Funktionen eine wichtige Rolle: So basiert dieser Algorithmus auf denen von Murota aus~\cite[Abschnitt~10.4]{Murota2003}, die wiederum selbst auf Algorithmen zu Flüssen in Netzwerken aufbauen.