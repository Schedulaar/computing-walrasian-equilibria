\section{Berechnung von Gleichgewichten}

\todo{Blabla}

\subsection{Informationszugang zum Markt}

Bei der Berechnung eines Marktgleichgewichts müssen auf irgendeine Weise Informationen erhoben werden.
Dieser Informationszugang kann dann von Algorithmen als Schnittstelle zur Berechnung eines Gleichgewichts verwendet werden.
Die Algorithmen werden also unter der Annahme entwickelt, dass ein Orakel existiert, welches eine solche Schnittstelle zum Markt implementiert.

Da der Detailgrad der erfassbaren Informationen bei vielen Marktsituationen unterschiedlich ist, unterscheiden Leme und Wong in~\cite{PaesLeme2018} drei verschiedene Modelle:
\begin{itemize}
	\item \emph{Die Mikroskopische Sicht:} Hier kann der Wert einzelner Bündel für jeden Käufer abgefragt werden.
	Das sogenannte \emph{Wert-Orakel} bestimmt also anhand eines Käufers $i\in\first{m}$ und eines Bündels $x\in\ffirst{\fs}$ den Wert $v_i(x)$.
	\item \emph{Die Agenten-Sicht:} In diesem Modell besteht der Zugang zum Markt daraus, von jedem Käufer ein von ihm nachgefragtes Bündel bei gegebenen Preisen abfragen zu können.
	Sollten mehrere Bündel für den Käufer gleichwertig sein, so wird ein beliebiges dieser Bündel ausgegeben.
	Das \emph{Nachfrage-Orakel} berechnet also anhand eines Preisvektors $\fp\in\R^n$ und eines Käufers $i\in\first{m}$ ein Bündel $\dem_i(\fp)\in\NB(i, \fp)$.
	\item \emph{Die Makroskopische Sicht:} Hier können keine Informationen einzelner Käufer abgefragt werden, sondern nur noch auf die aggregierte Nachfrage bei gegebenen Preisen.
	Das heißt, das sogenannte \emph{Aggregierte-Nachfrage-Orakel} liefert bei Eingabe eines Preisvektors $\fp\in\R^n$ einen Nachfrage-Vektor $\dem(\fp)\in\Z_{\geq0}^n$, für den nachgefragte Bündel $x^{(i)} \in \NB(i,\fp)$ mit $\dem(p)=\sum_{i\in\first{m}} x^{(i)}$ existieren.
\end{itemize}

Es sei erwähnt, dass man anhand eines Wert-Orakels ein Nachfrage-Orakel konstruieren kann:
Sind Preise $\fp$ sowie ein Käufer $i$ gegeben, kann für jedes Bündel $x$ der Nutzen $u_i(x; \fp)$ durch je eine Abfrage des Wert-Orakels berechnet werden und ein Bündel ausgegeben werden, welches diesen Wert maximiert.
Für eine Abfrage des Nachfrage-Orakels entstehen dann $\ffirst{\fs}$ Abfragen des Wert-Orakels.

Genauso lässt sich aus einem Nachfrage-Orakel auch ein Aggregierte-Nachfrage-Orakel gewinnen: So kann man mit $m$ Abfragen des Nachfrage-Orakels für alle $i\in\{m\}$ ein nachgefragtes Bündel erhalten, welche summiert den aggregierten Nachfrage-Vektor ergeben.

\todo{Im Folgenden wird ein Algorithmus diskutiert, der das Aggregierte-Nachfrage-Modell verwendet, um Walras-Gleichgewichte für allgemeine Bewertungsfunktionen zu berechnen.}

\subsection{Darstellung als Lineares Optimierungsproblem}

Man betrachte die Relaxation der Bestimmung von optimalen Allokationen als lineares Programm:
\begin{align*}
	\tag{P}\label{LP}
	&\max_{z} \sum_{i\in\first{m}, x\in\ffirst{\fs}} v_i(x) \cdot z_{i, x} \\[5pt]
	\text{udN.} \quad & \sum_{x\in\ffirst{\fs}} z_{i, x} = 1 & \text{für alle $i\in\first{m}$}\\[5pt]
	& \sum_{i\in\first{m}, x\in\ffirst{\fs}} x_j \cdot z_{i,x} = s_j & \text{für alle $j\in\first{n}$} \\[5pt]
	& z_{i, x} \geq 0 &\text{für alle $i\in\first{m}, x\in\ffirst{\fs}$}
\end{align*}
Führt man für die ersten $m$ Bedingungen die Variablen $(u_i)_{i\in\first{m}}$ und für die nächsten $n$ Bedingungen die Variablen $(p_j)_{j\in\first{n}}$ ein, erhält man folgendes duale Programm:
\begin{align*}
\tag{D}\label{DP}
&\min_{\fp,\fu} \sum_{i\in\first{m}} u_i + \fp \bcdot \fs \\[5pt]
\text{udN.} \quad &  u_i \geq v_i(x) - \fp \bcdot x & \text{für alle $i\in\first{m}, x\in\ffirst{\fs}$}
\end{align*}
\begin{lemma}
	Ein Walras-Gleichgewicht existert genau dann, wenn~\eqref{LP} eine ganzzahlige Optimallösung hat.
	Ist dies der Fall, so ist die Menge der Walras-Preise gerade die Menge der Optimallösungen von~\eqref{DP} projiziert auf die $\fp$-Koordinaten.
\end{lemma}
\begin{proof}
	Zunächst bemerke man, dass die zulässigen ganzzahligen Lösungen von~\eqref{LP} gerade den gültigen Allokationen entsprechen:
	Ist $z$ ganzzahlig und zulässig, so gibt es für alle $i\in\first{m}$ wegen den ersten Bedingungen genau ein Bündel $x^{(i)}\in\ffirst{\fs}$ mit $z_{i,x^{(i)}} = 1$.
	Aus den nächsten $n$ Bedingung folgt dann die Gültigkeit der Allokation $(x^{(i)})_{i\in\first{m}}$.
	Ausgehend von einer Allokation $\fx$ setzt man $z_{i,y} = 1$, falls $y = x^{(i)}$ gilt, und $z_{i,y}=0$ sonst, um zu einer ganzzahligen Lösung von~\eqref{LP} zu gelangen.
	
	Angenommen, es existiere ein Walras-Gleichgewicht $(\fx, \fp)$.
	Transformiert man die Allokation wie oben zu $(z_{i,x})_{i,x}$ und setzt man zusätzlich $u_i = \max_{x\in\ffirst{s}} v_i(x) - \fp \bcdot x$, erhält man eine primal und eine dual zulässige Lösung mit gleichem Zielfunktionswert:
	Es gilt mit $x^{(i)}\in\NB(i, \fp)$ und der Gültigkeit von $\fx$:
	\begin{align*}
		\sum_{i\in\first{m}} u_i + \fp \bcdot \fs
		&= \sum_{i\in\first{m}} \left( \max_{x\in\ffirst{\fs}} v_i(x) - \fp \bcdot x \right) + \fp \bcdot s = \sum_{i\in\first{m}} \left( v_i(x^{(i)}) - \fp\bcdot x^{(i)} \right) + \fp \bcdot \fs \\
		&= \sum_{i\in\first{m}} v_i(x^{(i)}) = \sum_{i\in\first{m}, x\in\ffirst{\fs}} v_i(x) \cdot z_{i,x}.
	\end{align*}
	Aufgrund schwacher Dualität ist $(z_{i,x})_{i,x}$ eine Optimallösung von~\eqref{LP}.
	
	Gibt es umgekehrt eine ganzzahlige Optimallösung, so kann diese wie oben zu einer gültigen Allokation $\fx$ transformiert werden.
	Aufgrund starker Dualität gibt es auch eine Optimallösung $(\fp, \fu)$ von~\eqref{DP} mit Zielfunktionswert $\sum_{i\in\first{m}} v_i(x^{(i)})$.
	Die Optimalität von $(\fp, \fu)$ impliziert bereits $u_i = \max_{x\in\ffirst{\fs}} v_i(x) - \fp \bcdot x$ für alle $i\in\first{m}$.
	Subtrahiert man $\fp \bcdot \fs$ vom Zielfunktionswert erhält man 
	\[
	\sum_{i\in\first{m}} \left( v_i(x^{(i)}) - \fp \bcdot x^{(i)} \right) = \sum_{i\in\first{m}} \max_{x\in\ffirst{\fs}} v_i(x) - \fp \bcdot x,
	\]
	wodurch $x^{(i)}\in\NB(i, \fp)$ für alle $i\in\first{m}$ folgt.
	Daher ist $\fx$ eine Walras-Allokation induziert durch die Preise $\fp$.
	
	Der zweite Teil der Aussage folgt aus dem Beweis des ersten Teils.
\end{proof}

\todo{Hier der Fehler vom Paper + Gegenbeispiel}

\subsection{Konvexe Darstellung und Subgradienten}

Für die Berechnung von Walras-Preisen genügt es also -- im Falle ihrer Existenz -- einen Minimierer von~\eqref{DP} zu ermitteln.
Dabei können die Variablen $(u_i)_{i\in\first{m}}$ vermieden werden, indem man zu einem konvexen Minimierungsproblem wechselt:
\begin{definition}
	Seien ein Angebot $\fs$ sowie Käufer mit Bewertungen $(v_i)_{i\in\first{m}}$ gegeben. Die \emph{Marktpotenzialfunktion} ist definiert als
	\[ f:\R^n \rightarrow \R,\quad \fp \mapsto \sum_{i\in\first{m}} \left( \max_{x\in\ffirst{\fs}} v_i(x) - \fp \bcdot x \right) + \fp \bcdot \fs. \]
\end{definition}
Die Minimierer dieser Marktpotenzialfunktion sind gerade alle Optimallösungen von~\eqref{DP} projiziert auf die $\fp$-Koordinaten.
Da das Maximum und die Summe konvexer Funktionen wieder konvex sind, ist $f$ ebenfalls konvex.

Eine Möglichkeit, konvexe Funktionen zu optimieren, bildet die sogenannte Subgradienten-Methode.
Daher wird zunächst der Begriff eines Subgradients eingeführt:
\begin{definition}[Subgradient]
	Sei $f:\R^n \rightarrow \R$ eine konvexe Funktion.
	Man nennt $g\in\R^n$ einen \emph{Subgradient von $f$ an $p\in\R^n$}, falls $f(x) \geq f(p) + g\bcdot (x - p)$ für alle $x\in\R^n$ gilt.
\end{definition}

Zwar ist es schwierig mit dem Aggregierte-Nachfrage-Orakel Auswertungen von $f$ selbst zu machen, jedoch lassen sich sehr wohl Subgradienten von $f$ damit ermitteln:
\begin{lemma}
	Für alle $p\in\R^n$ ist $\fs - \dem(\fp)$ ein Subgradient von $f$ in $\fp$.
\end{lemma}
\begin{proof}
	Sei $\dem(\fp) = \sum_{i\in[m]} x^{(i)}$ die aggregierte Nachfrage zum Preis $\fp$.
	Da $f$ eine Summe konvexer Funktionen ist, ist nach~\cite[Theorem~1.12]{Shor1985} jede Summe von Subgradienten der einzelnen Summanden in $\fp$ ein Subgradient von $f$ in $\fp$.
	Weiter ist nach~\cite[Theorem~1.13]{Shor1985} für das Maximum konvexer Funktionen jeder Subgradient einer Funktion, die in $\fp$ das Maximum annimmt, ein Subgradient der Maximumsfunktion in $\fp$.
	Des Weiteren ist für eine differenzierbare Funktion der einzige Subgradient der Gradient der Funktion.
	
	Dies heißt, da $x^{(i)}\in \arg\max_{x\in\ffirst{\fs}} v_i(x) - \fp \bcdot x$ für alle $i\in\first{m}$ gilt, sind $-x^{(i)}$ ein Subgradient von $\fp \mapsto \max_{x\in\ffirst{\fs}} v_i(x) - \fp \bcdot x$ für alle $i\in\first{m}$ und $\fs - \dem(\fp)$ ein Subgradient von $f$ in $\fp$.
\end{proof}

\begin{lemma}\label{prices-bounded}
	Ist $\fx$ eine Walras-Allokation, so sind für $M\coloneqq \max_{i\in\first{m}, x\in\ffirst{\fs}} \abs{v_i(x)}$ alle Walras-Preise in $[-2M, 2M]^n$ enthalten.
\end{lemma}
\begin{proof}
	Seien $\fp$ ein Walras-Preisvektor und $j\in\first{n}$ gegeben.
	Der Preis $p_j$ übersteigt $2M$ nicht, denn sonst würde kein Käufer $j$ nachfragen:
	Da $s_j$ positiv ist, gibt es einen Käufer $i$ mit $x^{(i)}_j > 0$, für den $x^{(i)} \in\NB(i, \fp)$ folgende Ungleichung impliziert, wobei $e_j$ der $j$-te Einheitsvektor ist:
	\[
	v_i(x^{(i)}) - \fp \bcdot x^{(i)} \geq v_i(x^{(i)} - e_j) - \fp \bcdot ( x^{(i)} - e_j ).
	 \]
	 Es folgt also $2M \geq v_i(x^{(i)}) - v_i(x^{(i)} - e_j) \geq p_j$.
	 Umgekehrt ist $p_j$ mindestens $-2M$, denn sonst würden alle Käufer jeweils das gesamte Angebot von $j$ nachfragen:
	 Angenommen, es gilt $p_j < -2M$, so folgt $x^{(i)}_j = s_j$ für alle $i\in\first{m}$, denn für $x^{(i)}_j < s_j$ folgt der Widerspruch
	 \[ 
	 	v_i(x^{i})  - \fp \bcdot x^{(i)} < v_i(x^{(i)}) - \fp \bcdot (x^{(i)} + e_j) - 2M \leq v_i(x^{(i)} + e_j) - \fp \bcdot (x^{(i)} + e_j).
	 \]
	 Da es mindestens zwei Käufer gibt, folgt schließlich $p_j \geq -2M$.
\end{proof}

\begin{lemma}
	Die Menge der Walras-Preise ist ein Polytop, dessen Ecken von der Form $p_j = a_j/b_j$ mit $a_j,b_j\in\Z$ und
	$\abs{b_j} \leq (Sn)^n$
	für $S = \max_{j\in\first{n}} s_j$ sind.
\end{lemma}
\begin{proof}
	Ergänzt man~\eqref{DP} um Schlupfvariablen $\fy=(y_{i,x})_{i,x}$, so erhält man Bedingungen der Form $ u_i + \fp \bcdot x + y_{i,x} = v_i(x) $ für alle $i\in\first{m}, x\in\ffirst{\fs}$.
	Sei $A$ die Matrix, die diese Nebenbedingungen in $A \cdot (\fu, \fp, \fy)^T = (v_i(x))_{i,x}$ zusammenfasst.
	
	
\end{proof}