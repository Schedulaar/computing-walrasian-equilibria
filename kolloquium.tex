
\documentclass[envcountsect, 10pt]{beamer}
\usetheme[headheight=0em, logo=none, themepath=beamercolorthemefibeamer, nofonts]{fibeamer}
\usepackage{beamercolorthemefibeamer}

\usepackage[utf8]{inputenc} % for input encoding
\usepackage[german]{babel} % for german localization
\usepackage{colonequals} % for :=
\usepackage{graphicx} % for \includegraphics
\usepackage{wrapfig} % for floating images to the right
\usepackage{tikz}
\usepackage{esvect}
\usepackage{tabto}
\usepackage{enumitem}
\usepackage{mathabx}
\usepackage{stmaryrd}
\usepackage{mathtools}
\usepackage{import}
\usepackage{framed}
\usepackage{tikz}


\usepackage{xcolor}
\usepackage{transparent}


\usetikzlibrary{shapes,snakes}
\usetikzlibrary{arrows.meta}
\usetikzlibrary{calc}

%-------------------------------------------------------------------------------
% Hilfreiche Befehle
%-------------------------------------------------------------------------------
\newcommand{\betrag}[1]{\lvert #1 \rvert}	        % Betrag
\providecommand*{\Lfloor}{\left\lfloor}                 % gro\ss{}es Abrunden
\providecommand*{\Rfloor}{\right\rfloor}                % gro\ss{}es Abrunden
\providecommand*{\Floor}[1]{\Lfloor #1 \Rfloor}         % gro\ss{}es ganzes Abrunden
\providecommand*{\Ceil}[1]{\left\lceil #1 \right\rceil} % gro\ss{}es ganzes Aufrunden

\newcommand{\Z}{\mathbb{Z}}
\newcommand{\N}{\mathbb{N}}
\newcommand{\R}{\mathbb{R}}
\newcommand{\Q}{\mathbb{Q}}
\providecommand*{\Lfloor}{\left\lfloor}                 % gro\ss{}es Abrunden
\providecommand*{\Rfloor}{\right\rfloor}                % gro\ss{}es Abrunden
\providecommand*{\floor}[1]{\Lfloor #1 \Rfloor}         % gro\ss{}es ganzes Abrunden
\providecommand*{\ceil}[1]{\left\lceil #1 \right\rceil} % gro\ss{}es ganzes Aufrunden
\newcommand{\firstNumbers}[1]{[#1]}
\newcommand{\transpose}{^\intercal}
\newcommand{\subjectTo}{\textbf{s.t.}}
\newcommand{\MIPR}{MIP\textsuperscript{*}}
\newcommand{\MIPI}{MIP}
\newcommand{\oBdA}{oBdA.}
\newcommand{\rang}{\operatorname{rang}}
\newcommand{\norm}[1]{\left\lVert#1\right\rVert_2}
\newcommand{\todo}[1]{{\color{red}{\emph{TODO: }}#1}}
\newcommand{\one}{\mathbbm{1}}
\newcommand{\eq}[1]{{\operatorname{eq}(#1)}}
\newcommand{\co}[1]{\operatorname{co}(#1)}

\setbeamertemplate{theorems}[numbered]
\newtheorem{conjecture}[theorem]{Vermutung}
\newtheorem{notation}[theorem]{Notation}
\newtheorem{korollar}[theorem]{Korollar}
\newtheorem{beispiel}[theorem]{Beispiel}
\newtheorem{proposition}[theorem]{Proposition}

\definecolor{darkblue}{HTML}{00446B}
\definecolor{darkgreen}{HTML}{006B44}

\newcommand{\bigO}{\mathcal{O}}
\DeclarePairedDelimiter\abs{\lvert}{\rvert}
\newcommand{\size}[1]{\langle#1\rangle}
\newcommand{\coloniff}{\vcentcolon\Longle ftrightarrow}

\newcommand{\E}{\mathbb{E}}

\DeclareMathOperator{\e}{ex}
\DeclareMathOperator{\ma}{mate}
\DeclareMathOperator{\Ex}{Ex}
\DeclareMathOperator{\SW}{SW}
\DeclareMathOperator{\NB}{D}
\DeclareMathOperator{\dem}{d}
\DeclareMathOperator{\dom}{dom}
\DeclareMathOperator{\vol}{vol}
\DeclareMathOperator{\interior}{int}
\DeclareMathOperator{\fraction}{frac}
\DeclareMathOperator{\FA}{\mathcal{A}}

\newcommand{\first}[1]{\left[ #1 \right]}
\newcommand{\ffirst}[1]{\left\llbracket #1 \right\rrbracket}
\newcommand{\zero}{\mathbf{0}}
\newcommand{\fp}{\mathbf{p}}
\newcommand{\fx}{\mathbf{x}}
\newcommand{\fy}{\mathbf{y}}
\newcommand{\fs}{\mathbf{s}}
\newcommand{\fu}{\mathbf{u}}
\newcommand{\fa}{\mathbf{a}}
\newcommand{\fr}{\mathbf{r}}
\newcommand{\mnath}{\textrm{\textup{M}}^\natural}

\makeatletter
\newcommand*\bcdot{\mathpalette\bcdot@{.5}}
\newcommand*\bcdot@[2]{\mathbin{\vcenter{\hbox{\scalebox{#2}{$\m@th#1\bullet$}}}}}
\makeatother


\makeatletter
\makeatother
\useoutertheme{infolines}
\newenvironment{noheadline}{
	\setbeamertemplate{headline}{}
}{}
\newenvironment{nofootline}{
\setbeamertemplate{footline}{}
}{}

\setbeamersize{text margin left=2.7em, text margin right=2.7em}
\setbeamertemplate{frametitle}{\insertframetitle}
\setbeamercolor{block body}{fg=black!90}

\setbeamerfont{title}{size=\huge}
\setbeamertemplate{institute}{\insertinstitute}

\AtBeginSection{
	\begin{noheadline}
		\begin{frame}
		\vfill
		\centering
		\begin{beamercolorbox}[sep=8pt,center]{title}
			\usebeamerfont{title}\insertsectionhead\par%
		\end{beamercolorbox}
		\vfill
	\end{frame}
	\end{noheadline}

}

\makeatletter
\setbeamertemplate{footline}{%
	\color{darkblue}% to color the progressbar
	\small
	\hfill{\insertframenumber\hspace{1.5em}}
	\vspace{1em}


	\hspace*{-\beamer@leftmargin}%
	\rule{\beamer@leftmargin}{1pt}%
	\rlap{\rule{\dimexpr\numexpr0\insertframenumber\dimexpr
			\textwidth\relax/\numexpr0\inserttotalframenumber}{1pt}}
	% next 'empty' line is mandatory!

	\vspace{0\baselineskip}
	{}
}


\setbeamertemplate{bibliography item}{\insertbiblabel}

\title{\LARGE Berechnung von Walras-Gleichgewichten}
\subtitle{Seminar zur Optimierung und Spieltheorie}
\author{~\\Michael Markl \\ 14. Mai 2020}
\date{23.01.2020}
\renewcommand{\[}{
	\setlength\abovedisplayskip{0.5ex}
	\setlength{\belowdisplayskip}{0.5ex}
	\setlength{\abovedisplayshortskip}{0.5ex}
	\setlength{\belowdisplayshortskip}{0.5ex}\begin{equation*}}

\institute{Insitut für Mathematik der Universität Augsburg\\Diskrete Mathematik, Optimierung und Operations Research}

\input{kolloquium/setupfont}

\newcommand*\diff{\mathop{}\!\mathrm{d}}
%\setlist[enumerate]{topsep=0.5ex,itemsep=0ex,partopsep=0ex,parsep=0.8ex}


\beamertemplatenavigationsymbolsempty

\begin{document}

	\setcounter{framenumber}{-1}

	\begin{nofootline}
		\frame{\titlepage}
	\end{nofootline}

	\begin{frame}{Gliederung}
		\tableofcontents
	\end{frame}

	\section{Walras-Gleichgewichte}
\subsection{Grundlegende Definitionen}

\begin{frame}{Grundlagen}
	\begin{notation}
		Für $k\in\Z_{\geq0}$ bezeichne $\first{k}\coloneqq \{ 1, 2, \dots, k \}$, $\ffirst{k}\coloneqq \{0, 1, \dots, k \}$.
		
		\pause
		Für $\fs=(s_1, \dots, s_n)\in\Z_{\geq0}^n$ bezeichne $\ffirst{\fs}\coloneqq \prod_{j\in\first{n}} \ffirst{s_j}$.
	\end{notation}
	
	\pause
	\begin{definition}[Markt]
		Ein Markt besteht aus:
		\begin{itemize}[label=\color{darkblue}$\bullet$]
			\item einer Menge $\first{n}$ von Gütern sowie einem Angebot $s_j\in\Z_{>0}$ von jedem Gut $j\in\first{n}$, \pause
			\item einer Menge $\first{m}$ von $m\geq 2$ Käufern mit je einer Bewertungsfunktion $v_i:\ffirst{\fs} \rightarrow \Z$ mit $v(\zero) = 0$ für $i\in\first{m}$.
		\end{itemize}
	\end{definition}
	\pause
	\begin{definition}[Nachfragebereich]
		Bei Preisen $\fp\in\R^n$ ist $u_i(x; \fp)\coloneqq v_i(x) - \fp \bcdot x$ der \emph{Nutzen eines Bündels} $x\in\ffirst{\fs}$ für einen Käufer $i\in\first{m}$.
		\pause
		Der \emph{Nachfragebereich} eines Käufers $i$ bei Preisen $\fp$ ist dann $\NB(v_i, \fp)\coloneqq \arg\max_{x\in\ffirst{\fs}} u_i(x; \fp)$.
	\end{definition}
\end{frame}

\begin{frame}{Grundlagen}
	\begin{definition}[Allokation]
		Eine \emph{Allokation} $\fx\coloneqq(x^{(i)})_{i\in\first{m}}$ weist jedem Käufer $i\in\first{m}$ ein Bündel $x^{(i)}\in\ffirst{\fs}$ zu.
		\pause
		
		Eine Allokation $\fx$ heißt \emph{gültig}, falls $\sum_{i\in\first{m}}x^{(i)} = \fs$ gilt.
		\pause 
		
		Das \emph{soziale Wohl} einer gültigen Allokation ist $\SW(\fx)\coloneqq \sum_{i\in\first{m}} v_i(x^{(i)})$.
		
		Gültige Allokationen mit maximalem sozialen Wohl heißen \emph{optimal}.
	\end{definition}
	
	\pause
	\begin{definition}[Walras-Gleichgewicht]
		Ein Paar $(\fx, \fp)$ aus einer gültigen Allokation $\fx$ und einem Preisvektor $\fp\in\R^n$ heißt \emph{(Walras-)Gleichgewicht}, falls $x^{(i)}\in\NB(v_i, \fp)$ für alle $i\in\first{m}$ gilt.
		Dabei nennt man $\fx$ eine \emph{Walras-Allokation zu Walras-Preisen} $\fp$.
	\end{definition}

	\pause
	\begin{lemma}[Wohlfahrtstheoreme]
		Ist $(\fx, \fp)$ ein Gleichgewicht, so ist $\fx$ eine optimale Allokation.
		
		Ist $\fy$ eine optimale Allokation, so bildet auch $(\fy, \fp)$ ein Gleichgewicht.
	\end{lemma}
\end{frame}



\subsection{Darstellung als lineares Optimierungsproblem}
\begin{frame}
\begin{lemma}[Darstellung als lineares Optimierungsproblem]
	\onslide<2->{Es existiert genau dann ein Gleichgewicht, wenn~\eqref{LP} eine ganzzahlige Optimallösung hat.
	Ist das der Fall, so sind die Walras-Preise gerade die Optimallösungen von~\eqref{DP} projiziert auf die $\fp$-Variablen.}
	\vspace{5pt}
	\hrule
	\begin{align*}
	\tag{P}\label{LP}
	&\max_{z} \sum_{i\in\first{m}, x\in\ffirst{\fs}} v_i(x) \cdot z_{i, x} \\
	\text{udN.} \quad & \sum_{x\in\ffirst{\fs}} z_{i, x} = 1 & \text{für alle $i\in\first{m}$}\\
	& \sum_{i\in\first{m}, x\in\ffirst{\fs}} x_j \cdot z_{i,x} = s_j & \text{für alle $j\in\first{n}$} \\
	& z_{i, x} \geq 0 &\text{für alle $i\in\first{m}, x\in\ffirst{\fs}$}
	\end{align*}
	\hrule
	\begin{align*}
	\tag{D}\label{DP}
	&\min_{\fp,\fu} \sum_{i\in\first{m}} u_i + \fp \bcdot \fs \\
	\text{udN.} \quad &  u_i \geq v_i(x) - \fp \bcdot x & \text{für alle $i\in\first{m}, x\in\ffirst{\fs}$}
	\end{align*}\vspace{-1.5em}
\end{lemma}
\end{frame}

	
	
\section{Berechnung von Walras-Preisen}
\subsection{Informationszugang zum Markt}
\begin{frame}{Informationszugang zum Markt}
	Man unterscheidet die drei Sichtweisen:
	\begin{itemize}[label=\color{darkblue}$\bullet$]
		\pause\item \emph{Die Mikroskopische Sicht:} Ein \emph{Wert-Orakel} bestimmt anhand eines Käufers $i\in\first{m}$ und eines Bündels $x\in\ffirst{\fs}$ den Wert $v_i(x)$.
		\pause\item \emph{Die Agenten-Sicht:} Ein \emph{Nachfrage-Orakel} bestimmt anhand eines Käufers $i\in\first{m}$ und eines Preisvektors $\fp\in\R^n$ ein Bündel $x\in\NB(v_i, \fp)$.
		\pause\item \emph{Die Makroskopische Sicht:} Ein \emph{Aggregierte-Nachfrage-Orakel} bestimmt anhand eines Preisvektors $\fp\in\R^n$ eine aggregierte Nachfrage $\dem(\fp)$, für den $x^{(i)}\in\NB(v_i, \fp)$ für alle $i\in\first{m}$ existieren mit $\dem(\fp) = \sum_{i\in\first{m}} x^{(i)}$.
	\end{itemize}

\vspace{1em}

\pause
	\parbox{\textwidth}{\emph{Ziel:} Berechnung von Walras-Preisen mit dem Aggregierte-Nachfrage-Orakel.}
	
\vspace{1em}

\pause
	\parbox{\textwidth}{
		\emph{Vorgehen:} Berechnung mittels Ellipsoidmethode. Voraussetzungen:
		\begin{itemize}[label=\color{darkblue}$\bullet$]
			\item Beschränkung der Menge der Walras-Preise
			\item Konstruktion eines Trennorakels
			\item Abschätzungen für Laufzeitanalyse
		\end{itemize}
	}
\end{frame}

\subsection{Berechnung mittels Ellipsoidmethode}
\begin{frame}{Beschränkung der Walras-Preise}
\begin{lemma}
	Für $M\coloneqq \max_{i\in\first{m},x\in\ffirst{\fs}} \abs{v_i(x)}$ sind alle Walras-Preise in $[-2M, 2M]^n$ enthalten.
\end{lemma}
\end{frame}

\begin{frame}[t]{Konstruktion eines Trennorakels}
\pause\begin{definition}[Subgradient]
	Seien $f:\R^n \rightarrow \R$ eine konvexe Funktion.
	Man nennt $g\in\R^n$ einen \emph{Subgradient von $f$ an $x_0$}, falls $f(x)\geq f(x_0) + g \bcdot (x - x_0)$ für alle $x\in\R^n$ gilt.
\end{definition}
\def\svgwidth{\columnwidth}
	\import{kolloquium/}{subgradient.pdf_tex}
\end{frame}

\begin{frame}[t]{Konstruktion eines Trennorakels}
\addtocounter{theorem}{-1}
\vspace{0.2em}
\begin{definition}[Subgradient]
	Seien $f:\R^n \rightarrow \R$ eine konvexe Funktion.
	Man nennt $g\in\R^n$ einen \emph{Subgradient von $f$ an $x_0$}, falls $f(x)\geq f(x_0) + g \bcdot (x - x_0)$ für alle $x\in\R^n$ gilt.
\end{definition}
\vspace{-0.4em}
\begin{definition}[Marktpotenzialfunktion]
Die \emph{Marktpotenzialfunktion} ist definiert als
\[ f: \R^n \rightarrow \R, \quad \fp \mapsto \sum_{i\in\first{m}} \left( 
\max_{x\in\ffirst{\fs}} v_i(x) - \fp \bcdot x
\right) + \fp\bcdot \fs \]
\end{definition}
\vspace{-0.4em}
\pause \begin{lemma}
Für alle $\fp\in\R^n$ ist $\fs - \dem(\fp)$ ein Subgradient von $f$ an $\fp$.
\end{lemma}
\vspace{-0.4em}
\pause \begin{definition}[Trennorakel]
	Ein Trennorakel für eine konvexe Menge $K$ gibt gegeben eines Punktes $\fp\in\R^n$ entweder die Meldung \glqq $\fp\in K$\grqq\ oder eine Halbebene aus, die $\fp$ von $K$ trennt, also einen Vektor $g\in\R^n$ mit $g\bcdot \fp \leq g \bcdot \fp^*$ für alle $\fp^*\in K$.
\end{definition}
\end{frame}

\begin{frame}{Abschätzungen zur Analyse der Laufzeit}
	\pause
	\vspace{-1em}
	\begin{align*}
		\tag{D}
		&\min_{\fp,\fu} \sum_{i\in\first{m}} u_i + \fp \bcdot \fs \\
		\text{udN.} \quad &  u_i \geq v_i(x) - \fp \bcdot x & \text{für alle $i\in\first{m}, x\in\ffirst{\fs}$}
	\end{align*}
	\begin{lemma}[Darstellung als LP]
		Das folgende Programm ist äquivalent zu~\eqref{DP}:
		\vspace{5pt}
		\hrule
		\begin{align*}
		\tag{TD}\label{TDP}
		&\min_{\fp, u} u + \fp \bcdot \fs \\
		\text{udN.}\quad& u \geq \sum_{i\in\first{m}} \left( v_i(x^{(i)}) - \fp \bcdot x^{(i)} \right) & \text{für alle $\fx = (x^{(i)})_{i\in\first{m}}\in\ffirst{\fs}^m$}
		\end{align*}
	\end{lemma}
	\pause \begin{lemma}
		Die Walras-Preise bilden ein Polytop, dessen Ecken $\fp$ von der Form $p_j=a_j/b$ mit $a_j,b\in\Z$ und $\abs{b}\leq (n+1)!\, (mS)^n$ für $S\coloneqq\max_{j\in\first{n}} s_j$.
	\end{lemma}
\end{frame}

\begin{frame}{Berechnung von Walras-Preisen}
	\begin{theorem}[Ellipsoid-Methode]
		Sei ein Trennorakel einer konvexen Menge $K\subseteq B_r(\zero)$ gegeben.
		
		Mit $t$ Abfragen des Orakels lässt sich entweder ein Punkt $\fp\in K$ oder eine Ellipse $E$ mit $\vol(E) \leq \exp(-t/(2n+1)) \cdot \vol(B_r(\zero))$ und $K\subseteq E$ ermitteln.
	\end{theorem}
	\pause \begin{theorem}[Berechnung von Walras-Preisen]
		Gilt $\interior(K)\neq \emptyset$ für die Menge $K$ der Walras-Preise, kann man mit $\bigO(n^3\log(nmS) + n^2\log(M))$ Abfragen des Aggregierte-Nachfrage-Orakel einen Walras-Preisvektor bestimmen.
	\end{theorem}
\end{frame}
	
	\section{Brutto-Substituts-Bewertungen}
\newcommand{\tild}[1]{\widetile{#1}}
\subsection{Definition und Eigenschaften}
\begin{frame}{Brutto-Substituts-Bewertungen}
	\begin{definition}[Brutto-Substituts-Bewertung]
		Eine Bewertungsfunktion $v: \ffirst{\fs} \rightarrow \Z$ heißt \emph{Brutto-Substituts-Bewertung},
		falls für alle Preise $\fp,\fp'\in\R^n$ mit $\fp' \geq \fp$ und für alle Bündel $x\in\NB(v, \fp)$ ein Bündel $y\in\NB(v,\fp')$ existiert mit $y_i \geq x_i$ für alle $i$ mit $p_i' = p_i$.
		
		\pause\vspace{1em}
		
		\parbox{\textwidth}{Eine Bewertungsfunktion $v:\ffirst{\fs} \rightarrow \Z$ heißt \emph{unabhängige Brutto-Substituts-Bewertung}, falls die Funktion}
		\[
		\tilde{v}:\ffirst{1}^{N_\fs} \rightarrow \Z, \quad (x_{i,j})_{(i,j)\in N_\fs} \mapsto v \left( \left( \sum_{j\in\first{s_i}} x_{i,j} \right)_{i\in\first{n}} \right)
		\]
		Brutto-Substituts-Funktion ist.
		Dabei ist $N_\fs \coloneqq \{ (i,j) \mid i\in\first{n}, j\in\first{s_i} \}$.
	\end{definition}
\end{frame}


\begin{frame}{Brutto-Substituts-Bewertungen}
\begin{definition}[Brutto-Substituts-Bewertung]
Eine Bewertungsfunktion $v: \ffirst{\fs} \rightarrow \Z$ heißt \emph{Brutto-Substituts-Bewertung},
falls für alle Preise $\fp,\fp'\in\R^n$ mit $\fp' \geq \fp$ und für alle Bündel $x\in\NB(v, \fp)$ ein Bündel $y\in\NB(v,\fp')$ existiert mit $y_i \geq x_i$ für alle $i$ mit $p_i' = p_i$.

\pause\vspace{1em}

\parbox{\textwidth}{Eine Bewertungsfunktion $v:\ffirst{\fs} \rightarrow \Z$ heißt \emph{unabhängige Brutto-Substituts-Bewertung}, falls die Funktion}
\[
\tilde{v}:\ffirst{1}^{N_\fs} \rightarrow \Z, \quad (x_{i,j})_{(i,j)\in N_\fs} \mapsto v \left( \left( \sum_{j\in\first{s_i}} x_{i,j} \right)_{i\in\first{n}} \right)
\]
Brutto-Substituts-Funktion ist.
Dabei ist $N_\fs \coloneqq \{ (i,j) \mid i\in\first{n}, j\in\first{s_i} \}$.
\end{definition}
\end{frame}

\subsection{Auswirkung auf Walras-Preise}
\begin{frame}{Auswirkung auf Walras-Preise}
\begin{theorem}[[FY03, Theorem 2.1]]
Seien eine unabhängige Brutto-Substituts-Bewertung $v: \ffirst{\fs} \rightarrow \Z$ sowie Preise $\fp$ gegeben und man definiere $u(x) \coloneqq v(x) - \fp \bcdot x$ für $x\in\ffirst{\fs}$.

Dann impliziert $u(x)\geq \max_{i,j\in\ffirst{n}} u(x - e_i + e_j)$ für alle $x\in\ffirst{\fs}$ bereits 
$u(x) \geq u(y)$ für alle $x\in\ffirst{\fs}$.
\end{theorem}

\begin{theorem}
	Haben alle Käufer unabhängige Brutto-Substituts-Bewertungen, bilden die Walras-Preise ein ganzzahliges Polytop.
\end{theorem}
Für eine Zahl $a\in\R$ sei $\fraction(a) \coloneqq a - \floor{a}$.
\begin{proposition}
	Seien $p_1,p_2\in\R$ und sei $\theta$ zufällig aus $[0,1]$ gewählt.
	Setzt man \[ \hat{p_i}\coloneqq \begin{cases}
		\ceil{p_i}, & \text{für $\fraction(p_i) > \theta$,} \\
		\floor{p_i}, & \text{für $\fraction(p_i) \leq \theta$,}
	\end{cases}  \]
	so ist $\E(\hat{p_i}) = p_i$ für $i\in\first{2}$ und es gilt $\hat{p_1} - \hat{p_2}\in\{ \floor{p_1 - p_2}, \ceil{p_1 - p_2}  \}$.
\end{proposition}
\end{frame}


\begin{frame}{Auswirkung auf Walras-Preise}
\begin{theorem}
	Haben alle Käufer unabhängige Brutto-Substituts-Bewertungen und gilt $\interior(K)\neq\emptyset$ für die Menge $K$ der Walras-Preise, so kann man mit $\bigO(n^2 \log(M))$ Abfragen des Aggregierte-Nachfrage-Orakels einen Walras-Preisvektor bestimmen.
	
	Alternativ lässt sich mit $\bigO(n^4 m S \log(M))$ Abfragen des Wert-Orakels ein Walras-Gleichgewicht bestimmen.
\end{theorem}
\end{frame}
	
	\begin{noheadline}
		\begin{frame}<presentation:0>[noframenumbering]
			\cite{PaesLeme2018}
			\cite{Fujishige2003}
		\end{frame}
	
		\begin{frame}{Literatur}
			\scriptsize
			\bibliographystyle{alphadin}
			\bibliography{literature/kolloquium}
		\end{frame}
	\end{noheadline}

\end{document}
