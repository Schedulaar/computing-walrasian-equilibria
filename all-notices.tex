\documentclass[paper=a4, 	% Seitenformat
		fontsize=11pt, 		% Schriftgr\"o\ss{}e
		abstracton, 	% mit Abstrakt
		headsepline, 	% Trennlinie f\"ur die Kopfzeile
		notitlepage	% keine extra Titelseite
		]{scrartcl}

%%%%%%%%%%%%%%%%%%%%%%%%%%%%%%%%%%%%%%%%%%%%%%%%%%%%%%%%%%%%%%%%%%%%%%%%%%%%%%%%%%%%%%%%%%%
% Zusammenfassung einiger nützlicher und Befehle
\input{header/rest}			    % restliche Befehle und Pakete
%-----------------------------------------------------------------------------------
% Kopf-Zeilen
%-----------------------------------------------------------------------------------

\usepackage[automark]{scrlayer-scrpage}	% Seiten-Stil f\"ur scrartcl
\pagestyle{scrheadings}		% Kopfzeilen nach scr-Standard		
\ifx\chapter\undefined 		% falls Kapitel nicht definiert sind
  \automark[subsection]{section}% Kopf- und Fusszeilen setzen
\else				% Kapitel sind definiert
  \automark[section]{chapter}	% Kopf- und Fusszeilen setzen
\fi

%-----------------------------------------------------------------------------------
%   Maske f\"ur \"Uberschrift 
%-----------------------------------------------------------------------------------
% Belegung der notwendigen Kommandos f\"ur die Titelseite
\newcommand{\autor}{Markl, Michael} 		% bearbeitender Student
\newcommand{\veranstaltung}{Seminar zur Optimierung und Spieltheorie} 	% Titel des ganzen Seminars
\newcommand{\matrikelnummer}{1474802}
\newcommand{\uni}{Institut f\"ur Mathematik der Universit\"at Augsburg} % Universit\"at
\newcommand{\lehrstuhl}{Diskrete Mathematik, Optimierung und Operations Research} % Lehrstuhl
\newcommand{\semester}{Wintersemester 2020/21}	% Winter- oder Sommersemester mit Jahr
\newcommand{\datum}{14.05.2020} 			% Datumsangabe
\newcommand{\thema}{Berechnung von Walras-Gleichgewichten}  		% Titel der Seminararbeit

\newcommand{\ownline}{\vspace{.7em}\hrule\vspace{.7em}} % horizontale Linie mit Abstand

\newcommand{\seminarkopf}{	% Befehl zum Erzeugen der Titelseite 
 \textsc{\autor}  \hfill{\datum} \\ 
\textbf{\veranstaltung} \\ 
\uni \\ 
\lehrstuhl \\
\semester \hfill{Matrikelnummer: \matrikelnummer}
\ownline 

\begin{center}
{\LARGE \textbf{\thema}}
\end{center}

\ownline
}			    % Befehle und Pakete f\"ur Titelseite
\input{header/theorem}			% Mathematische Befehle und Pakete
\input{header/referenz}			% Befehle und Pakete für Referenzen

%\newcommand{\todo}[1]{{\color{red} #1}}

\newcommand{\abs}[1]{\left\lvert #1 \right\rvert}	        % Betrag
\newcommand{\norm}[1]{\left\lVert #1 \right\rVert_2}		        % Norm
\providecommand*{\Lfloor}{\left\lfloor}                 % gro\ss{}es Abrunden
\providecommand*{\Rfloor}{\right\rfloor}                % gro\ss{}es Abrunden
\providecommand*{\floor}[1]{\Lfloor #1 \Rfloor}         % gro\ss{}es ganzes Abrunden
\providecommand*{\ceil}[1]{\left\lceil #1 \right\rceil} % gro\ss{}es ganzes Aufrunden

\newcommand{\R}{\mathbb{R}} 
\newcommand{\E}{\mathbb{E}}
\newcommand{\N}{\mathbb{N}} 

\DeclareMathOperator{\e}{ex}
\DeclareMathOperator{\ma}{mate}
\DeclareMathOperator{\Ex}{Ex}
\DeclareMathOperator{\SW}{SW}
\DeclareMathOperator{\NB}{D}
\DeclareMathOperator{\dem}{d}
\DeclareMathOperator{\dom}{dom}
\DeclareMathOperator{\vol}{vol}
\DeclareMathOperator{\interior}{int}
\DeclareMathOperator{\fraction}{frac}
\DeclareMathOperator{\FA}{\mathcal{A}}
\newcommand{\bigO}{\mathcal{O}}

\newcommand{\first}[1]{\left[ #1 \right]}
\newcommand{\ffirst}[1]{\left\llbracket #1 \right\rrbracket}
\newcommand{\Z}{\mathbb{Z}}
\newcommand{\zero}{\mathbf{0}}
\newcommand{\fp}{\mathbf{p}}
\newcommand{\fx}{\mathbf{x}}
\newcommand{\fy}{\mathbf{y}}
\newcommand{\fs}{\mathbf{s}}
\newcommand{\fu}{\mathbf{u}}
\newcommand{\fa}{\mathbf{a}}
\newcommand{\fr}{\mathbf{r}}
\newcommand{\mnath}{\textrm{\textup{M}}^\natural}

\newcommand{\todo}[1]{{\color{red}#1}}

\makeatletter
\newcommand*\bcdot{\mathpalette\bcdot@{.5}}
\newcommand*\bcdot@[2]{\mathbin{\vcenter{\hbox{\scalebox{#2}{$\m@th#1\bullet$}}}}}
\makeatother

%%%%%%%%%%%%%%%%%%%%%%%%%%%%%%%%%%%%%%%%%%%%%%%%%%%%%%%%%%%%%%%%%%%%%%%%%%%%%%%%%%%%%%%%%%%
%%%%%%%%%%%%%%%%%%%%%%%%%%%%%%%%%%%%%%%%%%%%%%%%%%%%%%%%%%%%%%%%%%%%%%%%%%%%%%%%%%%%%%%%%%%
% Start des Dokuments
\begin{document}		

\ownautorefnames		% Änderung einiger automatischen Texte von hyperref (wie in referenz.tex definiert)
\parindent0em 			% kein Einzug nach einer Leerzeile

%%%%%%%%%%%%%%%%%%%%%%%%%%%%%%%%%%%%%%%%%%%%%%%%%%%%%%%%%%%%%%%%%%%%%%%%%%%%%%%%%%%%%%%%%%%
% Titelseite
\thispagestyle{empty}		% leerer Seitenstil, also keine Seitennummer
\begin{titlepage}
\seminarkopf 			% Titelblatt (wie in kopf.tex definiert)
\begin{abstract} 
Diese Arbeit gibt einen Einblick in die Berechnung von Walras-Gleichgewichten bei unteilbaren Gütern mit mehrfachem Angebot einzelner Güter.
Es wird nach einer formalen  Definition von Walras-Gleichgewichten zunächst beleuchtet, welche Eigenschaften Walras-Gleichgewichte im Allgemeinen haben und wann solche existieren.
Unter Benutzung eines makroskopischen Markt-Orakels wird anschließend die Methodik der Subgradientenverfahren genutzt, um Walras-Preise exakt zu berechnen, falls das Polytop dieser Preise volldimensional ist.
Haben die Käufer sogenannte Brutto-Substituts-Bewertungen, lässt sich das Berechnungsverfahren beschleunigen, da in diesem Fall die Walras-Preise ein ganzzahliges Polytop bilden.
Dabei spielen die verschiedenen Charakterisierungen von Brutto-Substituts-Bewertungen  als $\mnath$-konkave und als matroidale Funktionen eine zentrale Rolle.
\end{abstract}
\end{titlepage}

%%%%%%%%%%%%%%%%%%%%%%%%%%%%%%%%%%%%%%%%%%%%%%%%%%%%%%%%%%%%%%%%%%%%%%%%%%%%%%%%
% Inhaltsverzeichnis
\thispagestyle{empty}	
\tableofcontents		% Inhaltsverzeichnis
%\listoffigures			% Abbildungsverzeichnis (eventuell einfügen)
%\listoftables			% Tabellenverzeichnis (eventuell einfügen)
\setcounter{page}{0}% Eigentlicher Inhalt beginnt auf Seite 1
\clearpage          % neue Seite für eigentlichen Inhalt
%%%%%%%%%%%%%%%%%%%%%%%%%%%%%%%%%%%%%%%%%%%%%%%%%%%%%%%%%%%%%%%%%%%%%%%%%%%%%%%%
% Eigentlicher Inhalt der Seminararbeit; die einzelnen Teile werden hier (aus Gründen der Übersichtlichkeit) über \input{file} eingebunden


\section{Einführung}

\todo{Walras-Gleichgewichte rocken.}
\section{Walras-Gleichgewicht}

\newcommand{\first}[1]{\left[ #1 \right]}
\newcommand{\ffirst}[1]{\left\llbracket #1 \right\rrbracket}
\newcommand{\Z}{\mathbb{Z}}
\newcommand{\zero}{\mathbf{0}}
\newcommand{\fp}{\mathbf{p}}
\newcommand{\fx}{\mathbf{x}}
\newcommand{\fy}{\mathbf{y}}
\newcommand{\fs}{\mathbf{s}}
\newcommand{\fu}{\mathbf{u}}


\makeatletter
\newcommand*\bcdot{\mathpalette\bcdot@{.5}}
\newcommand*\bcdot@[2]{\mathbin{\vcenter{\hbox{\scalebox{#2}{$\m@th#1\bullet$}}}}}
\makeatother

Um Walras-Gleichgewichte formal einführen zu können, werden zunächst einige grundlegende Begriffe und Notationen erklärt.

\begin{notation}
	Für $k\in\Z_{\geq 0}$ sei $\first{k}\coloneqq \{ 1, 2, \dots, k \}$ die Menge der ersten $k$ natürlichen Zahlen; um die $0$ hinzuzunehmen, schreibt man $\ffirst{k}\coloneqq \{0,1,\dots, k \}$.
	Für einen Vektor $\fs = (s_1, \dots, s_n)\in\Z_{\geq0}^n$ definiert man $\ffirst{\fs}\coloneqq \Pi_{j\in\first{n}} \ffirst{s_j}$.
\end{notation}

Das Modell des Marktes besteht hier aus einer Menge $\first{m}$ von $m\geq 2$ \emph{Käufern}, einer Menge $\first{n}$ von \emph{Gütern} sowie aus einem \emph{Angebot} $s_j\in\Z_{>0}$ für jedes Gut $j\in\first{n}$.

Jedem Käufer $i\in\first{m}$ ist eine Bewertungsfunktion $v_i:\Z^n_{\geq 0} \rightarrow \Z$ zugeordnet, die einem $\emph{Bündel}$, also einer Multimenge an Gütern aus $\first{m}$, einen ganzzahligen Wert zuschreibt, wobei $v_i(\zero) = 0$ gilt.

Sind ein \emph{Preisvektor} $\fp\in\R^n$ und ein Bündel $x\in\Z^{m}_{\geq0}$ gegeben, bezeichnet $u_i(x;\fp) \coloneqq v_i(x) - \fp \bcdot x$ den Nutzen von Bündel $x$ bei Preisen $\fp$ für Käufer $i$.
Hierbei ist $\fp \cdot x$ das Skalarprodukt von $\fp$ und $x$.

\begin{definition}[Allokation]
	Eine \emph{Allokation} $\fx\coloneqq (x^{(i)})_{i\in\first{m}}$ weist jedem Käufer $i\in\first{m}$ ein Bündel $x^{(i)} \in\Z^n_{\geq0}$ zu.
	Ist ein Angebotsvektor $\fs\in\Z^m_{\geq0}$ gegeben, nennt man eine Allokation \emph{gültig}, falls sie genau das Angebot verteilt, das heißt, falls $\sum_{i\in\first{m}} x^{(i)} = \fs$ gilt.
	Alle gültigen Allokationen werden in der Menge $\FA$ gesammelt.
	
	Das \emph{soziale Wohl} einer gültigen Allokation $\fx$ ist definiert als $\SW(\fx) \coloneqq \sum_{i\in\first{m}} v_i(x^{(i)})$.
	Eine gültige Allokation mit maximalem sozialen Wohl wird \emph{optimale Allokation} genannt.
\end{definition}

\begin{definition}[Nachfragebereich]
	Man nennt die Menge $\NB(v,\fp)$ der Bündel, die den Nutzen unter einer Bewertungsfunktion $v:\Z^n_{\geq 0} \rightarrow \Z$ bei Preisen $\fp$ maximiert, den Nachfragebereich von $v$ bei Preisen $\fp$.
	Es ist also $ \NB(v, \fp) \coloneqq \arg\max_{x\in\ffirst{\fs}} v(x) - \fp \bcdot x$.
	Für den Nachfragebereich eines Käufers $i\in\first{m}$ wird abkürzend $\NB(i,p):=\NB(v_i, p)$ geschrieben.
\end{definition}

\begin{definition}[Walras-Gleichgewicht]
	Ein Paar $(\fx, \fp)$ bestehend aus einer gültigen Allokation $\fx$ und einem Preisvektor $\fp$ heißt \emph{(Walras-)Gleichgewicht}, falls jedem Käufer ein Bündel aus seinem Nachfragebereich zugewiesen wird, falls also $x^{(i)} \in \NB(i, \fp)$ für alle $i\in \first{m}$ gilt.
	Dabei nennt man $\fp$ einen \emph{Walras-Preisvektor} und $\fx$ eine von $\fp$ induzierte \emph{Walras-Allokation}.
\end{definition}

\begin{bemerkung}
	Im Vergleich zu~\cite{PaesLeme2018} wird hier das Angebot $s_j$ positiv statt nicht-negativ gewählt und es werden mindestens zwei Käufer vorausgesetzt, um den Beweis von Lemma~\ref{prices-bounded} bzw. von~\cite[Lemma 4]{PaesLeme2018} zu ermöglichen.
	Die ausgeschlossenen Fälle sind jedoch uninteressant.
\end{bemerkung}

In einem Walras-Gleichgewicht gibt es also für keinen der Käufer eine für ihn bessere Allokation: Das Bündel, das dem Käufer zugeteilt wird, hat bei den gegebenen Preisen einen für ihn maximalen Nutzen.
Des Weiteren wird durch Gültigkeit der Allokation in einem solchen Gleichgewicht sichergestellt, dass das Angebot und die Nachfrage des gesamten Marktes übereinstimmen: Es bleiben also weder Güter übrig, noch wird die Nachfrage irgendeines Käufers nicht gedeckt.

Darüber hinaus gelten hier die sogenannten Wohlfahrtstheoreme aus der Ökonomik:
Das erste Wohlfahrtstheorem besagt, dass jedes Gleichgewicht bei vollkommenen Wettbewerb, wie er hier unter den Käufern möglich ist, das soziale Wohl maximiert.
Das zweite Wohlfahrtstheorem sagt aus, dass jede optimale Allokation bei Walras-Preisen ein Gleichgewicht erzeugt.

\begin{lemma}[Erstes und zweites Wohlfahrtstheorem]
	Ist $(\fx, \fp)$ ein Gleichgewicht, so ist $\fx$ eine optimale Allokation.
	Ist $\fy$ eine beliebige optimale Allokation, so bildet auch $(\fy, \fp)$ ein Gleichgewicht.
\end{lemma}
\begin{proof}
	Seien $(\fx, \fp)$ ein Gleichgewicht und $\fy$ eine gültige Allokation.
	Für alle $i\in\first{m}$ gilt wegen $x^{(i)} \in \NB(i,\fp)$ dann $v_i(x^{(i)}) - \fp \bcdot x^{(i)} \geq v_i(y^{(i)}) - \fp \bcdot y^{(i)}$.
	Mit der Gültigkeit der Allokationen $\sum_{i\in\first{m}} x^{(i)} = \sum_{i\in\first{m}} y^{(i)} = \fs$ folgere man
	\[
		\sum_{i\in\first{m}} v_i(x^{(i)}) \geq
		\sum_{i\in\first{m}} \left( v_i(y^{(i)}) - \fp \bcdot (y^{(i)} - x^{(i)}) \right)
		= \sum_{i\in\first{m}} v_i(y^{(i)}) - \fp \bcdot (\fs - \fs) = \sum_{i\in\first{m}} v_i(y^{(i)}).
	\]
	Insbesondere ist $\fx$ also eine optimale Allokation.
	
	Ist nun $\fy$ ebenfalls optimal, gilt 
	$ \sum_{i\in\first{m}} v_i(x^{i}) = \sum_{i\in\first{m}} v_i(y^{(i)}) $ und mit der Gültigkeit der Allokationen lässt sich
	\[
		\sum_{i\in\first{m}} \left( v_i(x^{i}) - \fp \bcdot x^{(i)} \right)
		= \sum_{i\in\first{m}} \left( v_i(y^{(i)}) - \fp \bcdot y^{(i)} \right)
	\]
	folgern.
	Da $x^{(i)}\in\NB(i, \fp)$ für die Summanden bereits $v_i(x^{(i)}) - \fp \bcdot  x^{(i)} \geq v_i(y^{(i)}) - \fp \bcdot y^{(i)}$ impliziert, müssen diese bereits exakt übereinstimmen.
	Dadurch folgt auch $y^{(i)}\in\NB(i, \fp)$ für alle $i\in\first{m}$, sodass $\fy$ ebenfalls eine Walras-Allokation zu Preisen $\fp$ ist.
\end{proof}


\section{Berechnung von Gleichgewichten}

\todo{Blabla}

\subsection{Informationszugang zum Markt}

Bei der Berechnung eines Marktgleichgewichts müssen auf irgendeine Weise Informationen erhoben werden.
Dieser Informationszugang kann dann von Algorithmen als Schnittstelle zur Berechnung eines Gleichgewichts verwendet werden.
Die Algorithmen werden also unter der Annahme entwickelt, dass ein Orakel existiert, welches eine solche Schnittstelle zum Markt implementiert.

Da der Detailgrad der erfassbaren Informationen bei vielen Marktsituationen unterschiedlich ist, unterscheiden Leme und Wong in~\cite{PaesLeme2018} drei verschiedene Modelle:
\begin{itemize}
	\item \emph{Die Mikroskopische Sicht:} Hier kann der Wert einzelner Bündel für jeden Käufer abgefragt werden.
	Das sogenannte \emph{Wert-Orakel} bestimmt also anhand eines Käufers $i\in\first{m}$ und eines Bündels $x\in\ffirst{\fs}$ den Wert $v_i(x)$.
	\item \emph{Die Agenten-Sicht:} In diesem Modell besteht der Zugang zum Markt daraus, von jedem Käufer ein von ihm nachgefragtes Bündel bei gegebenen Preisen abfragen zu können.
	Sollten mehrere Bündel für den Käufer gleichwertig sein, so wird ein beliebiges dieser Bündel ausgegeben.
	Das \emph{Nachfrage-Orakel} berechnet also anhand eines Preisvektors $\fp\in\R^n$ und eines Käufers $i\in\first{m}$ ein Bündel $\dem_i(\fp)\in\NB(i, \fp)$.
	\item \emph{Die Makroskopische Sicht:} Hier können keine Informationen einzelner Käufer abgefragt werden, sondern nur noch auf die aggregierte Nachfrage bei gegebenen Preisen.
	Das heißt, das sogenannte \emph{Aggregierte-Nachfrage-Orakel} liefert bei Eingabe eines Preisvektors $\fp\in\R^n$ einen Nachfrage-Vektor $\dem(\fp)\in\Z_{\geq0}^n$, für den nachgefragte Bündel $x^{(i)} \in \NB(i,\fp)$ mit $\dem(p)=\sum_{i\in\first{m}} x^{(i)}$ existieren.
\end{itemize}

Es sei erwähnt, dass man anhand eines Wert-Orakels ein Nachfrage-Orakel konstruieren kann:
Sind Preise $\fp$ sowie ein Käufer $i$ gegeben, kann für jedes Bündel $x$ der Nutzen $u_i(x; \fp)$ durch je eine Abfrage des Wert-Orakels berechnet werden und ein Bündel ausgegeben werden, welches diesen Wert maximiert.
Für eine Abfrage des Nachfrage-Orakels entstehen dann $\ffirst{\fs}$ Abfragen des Wert-Orakels.

Genauso lässt sich aus einem Nachfrage-Orakel auch ein Aggregierte-Nachfrage-Orakel gewinnen: So kann man mit $m$ Abfragen des Nachfrage-Orakels für alle $i\in\{m\}$ ein nachgefragtes Bündel erhalten, welche summiert den aggregierten Nachfrage-Vektor ergeben.

\todo{Im Folgenden wird ein Algorithmus diskutiert, der das Aggregierte-Nachfrage-Modell verwendet, um Walras-Gleichgewichte für allgemeine Bewertungsfunktionen zu berechnen.}

\subsection{Darstellung als Lineares Optimierungsproblem}

Man betrachte die Relaxation der Bestimmung von optimalen Allokationen als lineares Programm:
\begin{align*}
	\tag{P}\label{LP}
	&\max_{z} \sum_{i\in\first{m}, x\in\ffirst{\fs}} v_i(x) \cdot z_{i, x} \\[5pt]
	\text{udN.} \quad & \sum_{x\in\ffirst{\fs}} z_{i, x} = 1 & \text{für alle $i\in\first{m}$}\\[5pt]
	& \sum_{i\in\first{m}, x\in\ffirst{\fs}} x_j \cdot z_{i,x} = s_j & \text{für alle $j\in\first{n}$} \\[5pt]
	& z_{i, x} \geq 0 &\text{für alle $i\in\first{m}, x\in\ffirst{\fs}$}
\end{align*}
Führt man für die ersten $m$ Bedingungen die Variablen $(u_i)_{i\in\first{m}}$ und für die nächsten $n$ Bedingungen die Variablen $(p_j)_{j\in\first{n}}$ ein, erhält man folgendes duale Programm:
\begin{align*}
\tag{D}\label{DP}
&\min_{\fp,\fu} \sum_{i\in\first{m}} u_i + \fp \bcdot \fs \\[5pt]
\text{udN.} \quad &  u_i \geq v_i(x) - \fp \bcdot x & \text{für alle $i\in\first{m}, x\in\ffirst{\fs}$}
\end{align*}
\begin{lemma}
	Ein Walras-Gleichgewicht existert genau dann, wenn~\eqref{LP} eine ganzzahlige Optimallösung hat.
	Ist dies der Fall, so ist die Menge der Walras-Preise gerade die Menge der Optimallösungen von~\eqref{DP} projiziert auf die $\fp$-Koordinaten.
\end{lemma}
\begin{proof}
	Zunächst bemerke man, dass die zulässigen ganzzahligen Lösungen von~\eqref{LP} gerade den gültigen Allokationen entsprechen:
	Ist $z$ ganzzahlig und zulässig, so gibt es für alle $i\in\first{m}$ wegen den ersten Bedingungen genau ein Bündel $x^{(i)}\in\ffirst{\fs}$ mit $z_{i,x^{(i)}} = 1$.
	Aus den nächsten $n$ Bedingung folgt dann die Gültigkeit der Allokation $(x^{(i)})_{i\in\first{m}}$.
	Ausgehend von einer Allokation $\fx$ setzt man $z_{i,y} = 1$, falls $y = x^{(i)}$ gilt, und $z_{i,y}=0$ sonst, um zu einer ganzzahligen Lösung von~\eqref{LP} zu gelangen.
	
	Angenommen, es existiere ein Walras-Gleichgewicht $(\fx, \fp)$.
	Transformiert man die Allokation wie oben zu $(z_{i,x})_{i,x}$ und setzt man zusätzlich $u_i = \max_{x\in\ffirst{s}} v_i(x) - \fp \bcdot x$, erhält man eine primal und eine dual zulässige Lösung mit gleichem Zielfunktionswert:
	Es gilt mit $x^{(i)}\in\NB(i, \fp)$ und der Gültigkeit von $\fx$:
	\begin{align*}
		\sum_{i\in\first{m}} u_i + \fp \bcdot \fs
		&= \sum_{i\in\first{m}} \left( \max_{x\in\ffirst{\fs}} v_i(x) - \fp \bcdot x \right) + \fp \bcdot s = \sum_{i\in\first{m}} \left( v_i(x^{(i)}) - \fp\bcdot x^{(i)} \right) + \fp \bcdot \fs \\
		&= \sum_{i\in\first{m}} v_i(x^{(i)}) = \sum_{i\in\first{m}, x\in\ffirst{\fs}} v_i(x) \cdot z_{i,x}.
	\end{align*}
	Aufgrund schwacher Dualität ist $(z_{i,x})_{i,x}$ eine Optimallösung von~\eqref{LP}.
	
	Gibt es umgekehrt eine ganzzahlige Optimallösung, so kann diese wie oben zu einer gültigen Allokation $\fx$ transformiert werden.
	Aufgrund starker Dualität gibt es auch eine Optimallösung $(\fp, \fu)$ von~\eqref{DP} mit Zielfunktionswert $\sum_{i\in\first{m}} v_i(x^{(i)})$.
	Die Optimalität von $(\fp, \fu)$ impliziert bereits $u_i = \max_{x\in\ffirst{\fs}} v_i(x) - \fp \bcdot x$ für alle $i\in\first{m}$.
	Subtrahiert man $\fp \bcdot \fs$ vom Zielfunktionswert erhält man 
	\[
	\sum_{i\in\first{m}} \left( v_i(x^{(i)}) - \fp \bcdot x^{(i)} \right) = \sum_{i\in\first{m}} \max_{x\in\ffirst{\fs}} v_i(x) - \fp \bcdot x,
	\]
	wodurch $x^{(i)}\in\NB(i, \fp)$ für alle $i\in\first{m}$ folgt.
	Daher ist $\fx$ eine Walras-Allokation induziert durch die Preise $\fp$.
	
	Der zweite Teil der Aussage folgt aus dem Beweis des ersten Teils.
\end{proof}

\todo{Hier der Fehler vom Paper + Gegenbeispiel}

\subsection{Konvexe Darstellung und Subgradienten}

Für die Berechnung von Walras-Preisen genügt es also -- im Falle ihrer Existenz -- einen Minimierer von~\eqref{DP} zu ermitteln.
Dabei können die Variablen $(u_i)_{i\in\first{m}}$ vermieden werden, indem man zu einem konvexen Minimierungsproblem wechselt:
\begin{definition}
	Seien ein Angebot $\fs$ sowie Käufer mit Bewertungen $(v_i)_{i\in\first{m}}$ gegeben. Die \emph{Marktpotenzialfunktion} ist definiert als
	\[ f:\R^n \rightarrow \R,\quad \fp \mapsto \sum_{i\in\first{m}} \left( \max_{x\in\ffirst{\fs}} v_i(x) - \fp \bcdot x \right) + \fp \bcdot \fs. \]
\end{definition}
Die Minimierer dieser Marktpotenzialfunktion sind gerade alle Optimallösungen von~\eqref{DP} projiziert auf die $\fp$-Koordinaten.
Da das Maximum und die Summe konvexer Funktionen wieder konvex sind, ist $f$ ebenfalls konvex.

Eine Möglichkeit, konvexe Funktionen zu optimieren, bildet die sogenannte Subgradienten-Methode.
Daher wird zunächst der Begriff eines Subgradients eingeführt:
\begin{definition}[Subgradient]
	Sei $f:\R^n \rightarrow \R$ eine konvexe Funktion.
	Man nennt $g\in\R^n$ einen \emph{Subgradient von $f$ an $p\in\R^n$}, falls $f(x) \geq f(p) + g\bcdot (x - p)$ für alle $x\in\R^n$ gilt.
\end{definition}

Zwar ist es schwierig mit dem Aggregierte-Nachfrage-Orakel Auswertungen von $f$ selbst zu machen, jedoch lassen sich sehr wohl Subgradienten von $f$ damit ermitteln:
\begin{lemma}
	Für alle $p\in\R^n$ ist $\fs - \dem(\fp)$ ein Subgradient von $f$ in $\fp$.
\end{lemma}
\begin{proof}
	Sei $\dem(\fp) = \sum_{i\in[m]} x^{(i)}$ die aggregierte Nachfrage zum Preis $\fp$.
	Da $f$ eine Summe konvexer Funktionen ist, ist nach~\cite[Theorem~1.12]{Shor1985} jede Summe von Subgradienten der einzelnen Summanden in $\fp$ ein Subgradient von $f$ in $\fp$.
	Weiter ist nach~\cite[Theorem~1.13]{Shor1985} für das Maximum konvexer Funktionen jeder Subgradient einer Funktion, die in $\fp$ das Maximum annimmt, ein Subgradient der Maximumsfunktion in $\fp$.
	Des Weiteren ist für eine differenzierbare Funktion der einzige Subgradient der Gradient der Funktion.
	
	Dies heißt, da $x^{(i)}\in \arg\max_{x\in\ffirst{\fs}} v_i(x) - \fp \bcdot x$ für alle $i\in\first{m}$ gilt, sind $-x^{(i)}$ ein Subgradient von $\fp \mapsto \max_{x\in\ffirst{\fs}} v_i(x) - \fp \bcdot x$ für alle $i\in\first{m}$ und $\fs - \dem(\fp)$ ein Subgradient von $f$ in $\fp$.
\end{proof}

\begin{lemma}\label{prices-bounded}
	Ist $\fx$ eine Walras-Allokation, so sind für $M\coloneqq \max_{i\in\first{m}, x\in\ffirst{\fs}} \abs{v_i(x)}$ alle Walras-Preise in $[-2M, 2M]^n$ enthalten.
\end{lemma}
\begin{proof}
	Seien $\fp$ ein Walras-Preisvektor und $j\in\first{n}$ gegeben.
	Der Preis $p_j$ übersteigt $2M$ nicht, denn sonst würde kein Käufer $j$ nachfragen:
	Da $s_j$ positiv ist, gibt es einen Käufer $i$ mit $x^{(i)}_j > 0$, für den $x^{(i)} \in\NB(i, \fp)$ folgende Ungleichung impliziert, wobei $e_j$ der $j$-te Einheitsvektor ist:
	\[
	v_i(x^{(i)}) - \fp \bcdot x^{(i)} \geq v_i(x^{(i)} - e_j) - \fp \bcdot ( x^{(i)} - e_j ).
	 \]
	 Es folgt also $2M \geq v_i(x^{(i)}) - v_i(x^{(i)} - e_j) \geq p_j$.
	 Umgekehrt ist $p_j$ mindestens $-2M$, denn sonst würden alle Käufer jeweils das gesamte Angebot von $j$ nachfragen:
	 Angenommen, es gilt $p_j < -2M$, so folgt $x^{(i)}_j = s_j$ für alle $i\in\first{m}$, denn für $x^{(i)}_j < s_j$ folgt der Widerspruch
	 \[ 
	 	v_i(x^{i})  - \fp \bcdot x^{(i)} < v_i(x^{(i)}) - \fp \bcdot (x^{(i)} + e_j) - 2M \leq v_i(x^{(i)} + e_j) - \fp \bcdot (x^{(i)} + e_j).
	 \]
	 Da es mindestens zwei Käufer gibt, folgt schließlich $p_j \geq -2M$.
\end{proof}

\begin{lemma}
	Die Menge der Walras-Preise ist ein Polytop, dessen Ecken von der Form $p_j = a_j/b_j$ mit $a_j,b_j\in\Z$ und
	$\abs{b_j} \leq (Sn)^n$
	für $S = \max_{j\in\first{n}} s_j$ sind.
\end{lemma}
\begin{proof}
	Ergänzt man~\eqref{DP} um Schlupfvariablen $\fy=(y_{i,x})_{i,x}$, so erhält man Bedingungen der Form $ u_i + \fp \bcdot x + y_{i,x} = v_i(x) $ für alle $i\in\first{m}, x\in\ffirst{\fs}$.
	Sei $A$ die Matrix, die diese Nebenbedingungen in $A \cdot (\fu, \fp, \fy)^T = (v_i(x))_{i,x}$ zusammenfasst.
	
	
\end{proof}
\section{Brutto-Substituts-Bewertungen}

\todo{Kelso and Crawford's definition of gross substitutes}

\iffalse
\begin{definition}[Diskret-Konkave Funktion]
	Eine Funktion $v: \ffirst{\fs}^n \rightarrow \Z$ heißt \emph{diskret-konkav}, falls lokale Minima auch global minimal sind, also falls für alle Preise $\fp\in\R^n$ und Bündel $x\in\ffirst{\fs}^n$ mit \begin{align*}
	&v(x) \geq \max_{i : x_i>0} v(x - e_i) + p_i, \qquad
	&v(x) \geq \max_{j:x_j < s_j} v(x + e_j) - p_j, \\
	&v(x) \geq \max_{\substack{i: x_i>0 \\ j: x_j < s_j}} v(x + e_i - e_j) - p_i + p-j
	\end{align*}
	bereits $x\in\NB(v, \fp)$ gilt.
\end{definition}
\fi

\begin{definition}[Brutto-Substituts-Bewertung]
Eine Bewertungsfunktion $v: \ffirst{s} \rightarrow \Z$ heißt \emph{Brutto-Substituts-Bewertung} (engl. gross substitutes valuation), falls es für Preise $\fp,\fp' \in\R^n$ mit $\fp' \geq \fp$ und für ein nachgefragtes Bündel $x\in\NB(v, \fp)$ zu Preisen $\fp$ ein nachgefragtes Bündel $y\in\NB(v,\fp')$ zu Preisen $\fp'$ existiert, welches $y_j \geq x_j$ für alle $j\in\first{n}$ mit $p_j = p_j'$ erfüllt.
\end{definition}

Einige wichtige Resultate über Brutto-Substituts-Bewertungen wurden nur für den Fall gezeigt, dass jedes jedes Gut genau einmal angeboten wird, also für $s_j = 1$ für alle $j\in\first{n}$.
Man kann den Fall mehrfachen Angebots einzelner Güter darauf herunterbrechen, indem man jedes Vorkommen eines Guts unabhängig bepreisen lässt:
\newcommand{\tild}[1]{\widetilde{#1}}
\begin{definition}[Unabhängige Brutto-Substituts-Bewertung]
	Eine Bewertungsfunktion $v: \ffirst{s} \rightarrow \Z$ heißt \emph{unabhängige Brutto-Substituts-Bewertung}, falls \[
	\tild{v}: \ffirst{1}^{\sum_{i\in\first{n}}s_i} \rightarrow \Z,
	\quad (x_{i,j})_{i\in\first{n}, j\in\first{s_i}} \mapsto v\left( \left( \sum_{j\in\first{s_i}} x_{i,j} \right)_{i\in\first{n}} \right)
	\]
	eine Brutto-Substituts-Bewertung ist.
\end{definition}
\begin{proposition}
	Jede unabhängige Brutto-Substituts-Funktion ist eine Brutto-Substituts-Funktion.
	Es gibt aber Brutto-Substituts-Funktionen, die nicht unabhängig sind.
\end{proposition}
\begin{proof}
	Für Preise $\fp,\fp'\in\R^n$ mit $\fp' \geq \fp$ definiere man $\tild{\fp}\coloneqq (p_i)_{i\in\first{n},j\in\first{s_i}}$ und analog $\tild{\fp'}\coloneqq (p_i')_{i\in\first{n},j\in\first{s_i}}$.
	Für ein Bündel $x\in\NB(v,\fp)$ ist $\tild{x}$ mit $\tild{x}_{i,j} = 1$ für $i\in\first{n}$ und $j\leq x_i$ in $\NB(\tild{v}, \tild{\fp})$ enthalten.
	Es existiert ein Bündel $\tild{y}$ mit $\tild{y}_{i,j} \geq \tild{x}_{i,j}$ für alle $i,j$ mit $p_i \geq p_i'$ und $j\in\first{s_i}$ in $\NB(\tild{v}, \tild{\fp'})$.
	Dementsprechend ist $y\coloneqq(\sum_{j\in\first{s_i}} \tild{y}_{i,j})_{i\in\first{n}}$ in $\NB(v, \fp')$ und es gilt $y_i \geq x_i$ für alle $i\in\first{n}$, sodass $v$ die Brutto-Substituts-Eigenschaft erfüllt.
	
	Nun ein Beispiel einer nicht unabhängigen Brutto-Substituts-Funktion: Wir betrachten einen Markt mit nur einem Gut mit $s=2$ und der Bewertungsfunktion $v(k)\coloneqq k^2$, welche automatisch eine Brutto-Substituts-Funktion ist.
	Setzt man Preise $\tild{p}_j \coloneqq 2$ für $j\in\first{2}$ fest,
	so ist der Nutzen beim Kauf beider Einheiten maximal; es gilt $(1,1)\in \NB(\tild{v}, \tild{\fp})$.
	Erhöht man den Preis der ersten Einheit auf $\tild{p}'_1 \coloneqq 3$ und behält den Preis der zweiten Einheit bei, so ist $(0,0)$ das einzige nachgefragte Bündel.
	Daher ist $\tilde{v}$ keine Brutto-Substituts-Funktion.
\end{proof}

\subsection{$\mnath$-Konkave Funktion}

\todo{Nach Murato Charakterisierung Theorem 6.2}

\begin{definition}[$\mnath$-Konkave Funktion]
	Eine Funktion $v:\Z^n \rightarrow \R \cup \{-\infty\}$, welche $\dom(v) \coloneqq \{ z\in\Z^n \mid v(z) \in \R \}\neq \emptyset$ erfüllt, heißt \emph{$\mnath$-konkav} (\glqq M-natürlich konkav\grqq), falls \[
	v(x) + v(y) \leq
	\max \left\{
		v(x - e_i) + v(y + e_i),
		\max_{j: x_j < y_j} v(x - e_i + e_j) + v(y + e_i - e_j)
	\right\}
	\] für alle $x,y\in\dom(f)$ und $i\in\first{n}$ mit $x_i > y_i$ gilt.
	Hier gelte $\max(\emptyset)=-\infty$.
	
	Eine Funktion $v:D \rightarrow \Z^n$ mit $D\subseteq \Z^n$ heißt \emph{$\mnath$-konkav}, falls ihre Erweiterung mit $v(x)\coloneqq-\infty$ für $x\in\Z^n\setminus D$ bereits $\mnath$-konkav ist.
\end{definition}

\todo{Nach Murato Charakterisierung Theorem 6.24 bzw. Abschnitt 11.3}

\begin{theorem}[\cite{}]\label{thm-char-m-concave}
	Eine Funktion $v:\Z^n \rightarrow \R \cup \{ -\infty \}$ ist genau dann \emph{M}$^\natural$-konkav, wenn für alle $\fp\in\R^n$ und $x,y\in\dom(v)\neq\emptyset$ die Ungleichung $v(x) - \fp\bcdot x < v(y) - \fp \bcdot y$ bereits \[ 
		v(x) - \fp\bcdot x < 
			\max_{i:\, x_i > y_i \,  \vee \, i = 0} \,\,
				\max_{j:\, x_j < y_j \, \vee \, j=0} \,\,
					v(x - e_i + e_j) - \fp \bcdot (x - e_i + e_j)
	\]
	impliziert. Dabei ist $e_0$ der Nullvektor.
\end{theorem}
\begin{korollar}\label{cor-concave-local-global}
	Es ist $v: \Z^n \rightarrow \R \cup \{ -\infty \}$ genau dann \emph{M}$^\natural$-konkav, wenn für alle $\fp\in\R^n$ lokale Maxima von $x \mapsto v(x) - \fp \bcdot x$ bereits global maximal sind, also 
	falls für alle $\fp\in\R^n$ und $x\in\dom(v)$ mit $
	v(x)\geq \max_{i,j\in\ffirst{n}} v(x - e_i + e_j) + p_i - p_j
	$
	bereits $v(x) - \fp \bcdot x \geq v(y) - \fp \bcdot y$ für alle $y\in\dom(v)$ gilt.
\end{korollar}
\begin{proof}
	Es gelte $v(x) \geq max_{i,j\in\ffirst{n}} v(x - e_i + e_j) + p_i - p_j$ und man nehme zusätzlich $v(x) - \fp\bcdot x < v(y) - \fp \bcdot y$ an.
	Es ist $y\in\dom(v)$ und nach Theorem~\ref{thm-char-m-concave} gibt es $i,j\in\ffirst{n}$ mit $v(x) - \fp \bcdot x < v(x - e_i + e_j) - \fp\bcdot (x - e_i + e_j)$, was der lokalen Maximalität widerspricht.
\end{proof}

Ein wichtiges Resultat, das die Brutto-Substituts-Eigenschaft mit $\mnath$-Konkavität in Beziehung stellt, liefert~\cite[Theorem 2.1]{Fujishige2003}:

\begin{theorem}\label{thm-fujishige-gs-iff-concave}
	Eine Bewertungsfunktion $\tild{v}:\ffirst{1}^n\rightarrow \Z$ erfüllt genau dann die Brutto-Substituts-Eigenschaft, wenn sie $\mnath$-konkav ist.
\end{theorem}
\begin{korollar}
	Jede unabhängige Brutto-Substituts-Funktion $v:\ffirst{\fs} \rightarrow \Z$ ist $\mnath$-konkav.
\end{korollar}
\todo{Domäne erklären}
\begin{proof}
	Nach Theorem~\ref{thm-fujishige-gs-iff-concave} ist $\tild{v}$ schon $\mnath$-konkav.
	Man zeige die notwendigen Voraussetzungen in~\ref{cor-concave-local-global} für $v$:
	Seien $x\in\ffirst{\fs}$ und $p\in\R^n$ mit $v(x)\geq \max_{i,j\in\ffirst{n}} v(x - e_i + e_j) + p_i - p_j$ gegeben.
	Man definiere das Bündel $\tild{x}\in\ffirst{1}^{\sum_{i\in\first{n}} s_i}$ mit $\tild{x}_{i,j} \coloneqq 1$ für $j\in\first{x_i}$ und $\tild{x}_{i,j} \coloneqq 0$ für $x_i < j \leq s_i$ sowie die Preise $\tild{p}_{i,j}\coloneqq p_i$ für alle $i\in\first{n},j\in\first{s_i}$.
	Dann
	gilt \[
	\tild{v}(\tild{x} - e_k + e_l) + \tild{p}_k - \tild{p}_l \leq \tild{v}(\tild{x})
	\] für alle $k, l \in \{ (i,j) \mid i\in\first{n}, j\in\first{s_i} \}\cup\{ 0 \}$, da diese Ungleichung nur von Gütern $i\in\first{n}$ und nicht von $j\in\first{s_i}$ abhängt und sich dadurch die Voraussetzungen an $x$ ausnutzen lassen.
	Aufgrund der $\mnath$-Konkavität von $\tild{v}$ gilt $\tild{x}\in\NB(\tild{v}, \tild{\fp})$ nach Korollar~\ref{cor-concave-local-global}. Daraus folgt auch $x\in\NB(v,\fp)$.
\end{proof}
\begin{bemerkung}
	Nach~\cite[Theorem~11.4]{Murota2003} erfüllt jede $\mnath$-konkave Funktion die Brutto-Substituts-Eigenschaft.
	Ob diese auch unabhängig ist, bleibt an dieser Stelle offen.
\end{bemerkung}

\subsection{Auswirkung auf Walras-Preise}

\clearpage          % neue Seite für Literaturverzeichnis

%%%%%%%%%%%%%%%%%%%%%%%%%%%%%%%%%%%%%%%%%%%%%%%%%%%%%%%%%%%%%%%%%%%%%%%%%%%%%%%%
% Literaturverzeichnis
\nocite*  % Nicht zitierte Quellen werden auch ins Literaturverzeichnis aufgenommen
\thispagestyle{empty}
\bibliography{literature/seminararbeit}  % Literaturverzeichnis liegt in der Datei seminararbeit

%%%%%%%%%%%%%%%%%%%%%%%%%%%%%%%%%%%%%%%%%%%%%%%%%%%%%%%%%%%%%%%%%%%%%%%%%%%%%%%%
% Ende des Dokuments
\end{document}			
