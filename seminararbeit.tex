\documentclass[paper=a4, 	% Seitenformat
		fontsize=11pt, 		% Schriftgr\"o\ss{}e
		abstracton, 	% mit Abstrakt
		headsepline, 	% Trennlinie f\"ur die Kopfzeile
		notitlepage	% keine extra Titelseite
		]{scrartcl}

%%%%%%%%%%%%%%%%%%%%%%%%%%%%%%%%%%%%%%%%%%%%%%%%%%%%%%%%%%%%%%%%%%%%%%%%%%%%%%%%%%%%%%%%%%%
% Zusammenfassung einiger nützlicher und Befehle
\usepackage{array}		% erweiterte Tabellen

% Schriftzeichen, Format
\usepackage{latexsym}		% Latex-Symbole
\usepackage[utf8]{inputenc}	% Eingabekodierungen
\usepackage[english,ngerman]{babel}	% Mehrsprachenumgebung

% Layout
\usepackage{geometry}                    % Seitenränder
\geometry{a4paper, top=30mm, bottom=30mm, left=30mm, right=30mm}
\addtolength{\footskip}{-0.5cm}          % Seitenzahlen höher setzen
\usepackage{xcolor}                      % Farben


% Tabellen und Listen
\usepackage{float}		        % Gleitobjekte 
\usepackage[flushright]{paralist}       % Bessere Behandlung der Auflistungen

% Bilder
\usepackage[final]{graphicx}            % Graphiken einbinden

\usepackage{caption}                    % Beschriftungen
\usepackage{subcaption}                 % Beschriftungen f\"ur Unterteilung

\usepackage{pst-all}                    % Zeichnungen in Latex (kein pdflatex)
\usepackage{pstricks-add}               % zus\"atzliches von pstricks
\usepackage{pst-3dplot}                 % dreidimensionale Zeichnungen
\usepackage{pst-eucl}                   % euklidisches Paket
			    % restliche Befehle und Pakete
%-----------------------------------------------------------------------------------
% Kopf-Zeilen
%-----------------------------------------------------------------------------------

\usepackage[automark]{scrlayer-scrpage}	% Seiten-Stil f\"ur scrartcl
\pagestyle{scrheadings}		% Kopfzeilen nach scr-Standard		
\ifx\chapter\undefined 		% falls Kapitel nicht definiert sind
  \automark[subsection]{section}% Kopf- und Fusszeilen setzen
\else				% Kapitel sind definiert
  \automark[section]{chapter}	% Kopf- und Fusszeilen setzen
\fi

%-----------------------------------------------------------------------------------
%   Maske f\"ur \"Uberschrift 
%-----------------------------------------------------------------------------------
% Belegung der notwendigen Kommandos f\"ur die Titelseite
\newcommand{\autor}{Markl, Michael} 		% bearbeitender Student
\newcommand{\veranstaltung}{Seminar zur Optimierung und Spieltheorie} 	% Titel des ganzen Seminars
\newcommand{\matrikelnummer}{1474802}
\newcommand{\uni}{Institut f\"ur Mathematik der Universit\"at Augsburg} % Universit\"at
\newcommand{\lehrstuhl}{Diskrete Mathematik, Optimierung und Operations Research} % Lehrstuhl
\newcommand{\semester}{Wintersemester 2020/21}	% Winter- oder Sommersemester mit Jahr
\newcommand{\datum}{14.05.2020} 			% Datumsangabe
\newcommand{\thema}{Berechnung von Walras-Gleichgewichten}  		% Titel der Seminararbeit

\newcommand{\ownline}{\vspace{.7em}\hrule\vspace{.7em}} % horizontale Linie mit Abstand

\newcommand{\seminarkopf}{	% Befehl zum Erzeugen der Titelseite 
 \textsc{\autor}  \hfill{\datum} \\ 
\textbf{\veranstaltung} \\ 
\uni \\ 
\lehrstuhl \\
\semester \hfill{Matrikelnummer: \matrikelnummer}
\ownline 

\begin{center}
{\LARGE \textbf{\thema}}
\end{center}

\ownline
}			    % Befehle und Pakete f\"ur Titelseite
% Mathematische Zeichens\"atze und Umgebungen
\usepackage{amsfonts, amssymb}	% Definition einer Liste mathematischer Fontbefehle und Symbole
\usepackage[intlimits,		% Integralgrenzen \"uber und unter dem Integral
	    sumlimits]		% Summationsgrenzen \"uber und unter der Summe
           {amsmath}		% mathematische Verbesserungen
\usepackage{amsthm}		% spezielle theorem Stile
\usepackage{mathtools}
\usepackage{aliascnt} 
\usepackage{stmaryrd}


%-----------------------------------------------------------------------------------
% Hilfreiche Befehle
%-----------------------------------------------------------------------------------
\newcommand{\betrag}[1]{\lvert #1 \rvert}	        % Betrag
\newcommand{\norm}[1]{\lVert #1 \rVert}		        % Norm
\providecommand*{\Lfloor}{\left\lfloor}                 % gro\ss{}es Abrunden
\providecommand*{\Rfloor}{\right\rfloor}                % gro\ss{}es Abrunden
\providecommand*{\Floor}[1]{\Lfloor #1 \Rfloor}         % gro\ss{}es ganzes Abrunden
\providecommand*{\Ceil}[1]{\left\lceil #1 \right\rceil} % gro\ss{}es ganzes Aufrunden

\newcommand{\R}{\mathbb{R}} 
\newcommand{\N}{\mathbb{N}} 

\DeclareMathOperator{\e}{ex}
\DeclareMathOperator{\ma}{mate}
\DeclareMathOperator{\Ex}{Ex}

%-----------------------------------------------------------------------------------
%   Befehle f\"ur Nummerierung der Ergebnisse
%   fortlaufend innerhalb eines Abschnittes
%-----------------------------------------------------------------------------------
\theoremstyle{plain}            % normaler Stil
\newtheorem{theorem}{Theorem}[section]
% Lemma
\newaliascnt{lemma}{theorem}  
\newtheorem{lemma}[lemma]{Lemma}  
\aliascntresetthe{lemma}  
% Satz
\newaliascnt{satz}{theorem}  
\newtheorem{satz}[satz]{Satz} 
\aliascntresetthe{satz}
% Korollar
\newaliascnt{korollar}{theorem}  
\newtheorem{korollar}[korollar]{Korollar} 
\aliascntresetthe{korollar}
% Proposition
\newaliascnt{proposition}{theorem}  
\newtheorem{proposition}[proposition]{Proposition} 
\aliascntresetthe{proposition}
%-----------------------------------------------------------------------------------
\theoremstyle{definition}	% Definitionsstil
% Definition
\newaliascnt{definition}{theorem}  
\newtheorem{definition}[definition]{Definition} 
\aliascntresetthe{definition}
% Beispiel
\newaliascnt{beispiel} {theorem}  
\newtheorem{beispiel}[beispiel]{Beispiel} 
\aliascntresetthe{beispiel} 
% Problem
\newaliascnt{problem}{theorem}  
\newtheorem{problem}[problem]{Problem} 
\aliascntresetthe{problem}
% Algorithmus
\newaliascnt{algorithmus}{theorem}  
\newtheorem{algorithmus}[algorithmus]{Algorithmus} 
\aliascntresetthe{algorithmus} 
%-----------------------------------------------------------------------------------
\theoremstyle{remark}		% Bemerkungsstil
% Bemerkung
\newaliascnt{bemerkung}{theorem}  
\newtheorem{bemerkung}[bemerkung]{Bemerkung}  
\aliascntresetthe{bemerkung} 
% Vermutung
\newaliascnt{vermutung}{theorem}  
\newtheorem{vermutung}[vermutung]{Vermutung}  
\aliascntresetthe{vermutung} 
% Notation
\newaliascnt{notation}{theorem}  
\newtheorem{notation}[notation]{Notation} 
\aliascntresetthe{notation}

%-----------------------------------------------------------------------------------
% automatische Referenzen mit interaktiven Text
%-----------------------------------------------------------------------------------

% Texte
\newcommand{\theoremautorefname}{Theorem}
\newcommand{\lemmaautorefname}{Lemma}
\newcommand{\satzautorefname}{Satz}
\newcommand{\korollarautorefname}{Korollar}
\newcommand{\propositionautorefname}{Proposition}

\newcommand{\definitionautorefname}{Definition}
\newcommand{\beispielautorefname}{Beispiel}
\newcommand{\problemautorefname}{Problem}
\newcommand{\algorithmusautorefname}{Algorithmus}

\newcommand{\bemerkungautorefname}{Bemerkung}
\newcommand{\vermutungautorefname}{Vermutung}
\newcommand{\notationautorefname}{Notation}

%-----------------------------------------------------------------------------------
% Nummerierung der Gleichungen innerhalb der obersten Ebene
%-----------------------------------------------------------------------------------
\ifx\chapter\undefined 			% Kapitel sind definiert
  \numberwithin{equation}{section}	% Gleichungsnummern in Section
\else					% Kapitel sind nicht definiert
  \numberwithin{equation}{chapter}	% Gleichungsnummern in Kapiteln
\fi

			% Mathematische Befehle und Pakete

% Literatur-Bibliothek
\bibliographystyle{alphadin}               % deutscher Bibliotheksstil

% Interaktive Referenzen, und PDF-Keys
\usepackage{xr-hyper}  
\usepackage[pagebackref,                % R\"uckreferenz im Literaturverzeichnis
           pdftex,                      % Treiber f\"ur ps zu pdf ; f\"ur direkt nach pdf: pdftex
           ]{hyperref}

% Erweiterte Einstellungen zu hyperref

\hypersetup{
        breaklinks=true,        % zu lange Links unterbrechen
        colorlinks=true,        % F\"arben von Referenzen
        citecolor=black,        % Farbe der Zitate
        linkcolor=black,        % Farbe der Links
        extension=pdf,          % Externe Dokumente k\"onnen eingebunden werden.
        ngerman,		
	pdfview=FitH,
	pdfstartview=FitH,		
	bookmarksnumbered=true, % Anzeige der Abschnittsnummern	% pdf-Titel
	pdfauthor={\autor}% pdf-Autor
}

% Namen f\"ur Referenzen 

\newcommand{\ownautorefnames}{
  \renewcommand{\sectionautorefname}{Kapitel}
  \renewcommand{\subsectionautorefname}{Unterkapitel}
  \renewcommand{\subsubsectionautorefname}{\subsectionautorefname}
  \renewcommand{\appendixautorefname}{Anhang}
  \renewcommand{\figureautorefname}{Abbildung}
}

% R\"uckreferenzentext zur Literatur
\def\bibandname{und}%
\renewcommand*{\backref}[1]{}
\renewcommand*{\backrefalt}[4]{%
\ifcase #1 %
 (Nicht zitiert, also Erg\"anzungsliteratur.)%
\or
 (Zitiert auf Seite #2.)%
\else
 (Zitiert auf den Seiten #2.)%
\fi
}
\renewcommand{\backreftwosep}{ und~} % seperate 2 pages
\renewcommand{\backreflastsep}{ und~} % seperate last of longer 

\numberwithin{figure}{section}	% Abbildungsnummern in Section
			% Befehle und Pakete für Referenzen

\newcommand{\todo}[1]{{\color{red} #1}}

\newcommand{\abs}[1]{\left\lvert #1 \right\rvert}	        % Betrag
\newcommand{\norm}[1]{\left\lVert #1 \right\rVert_2}		        % Norm
\providecommand*{\Lfloor}{\left\lfloor}                 % gro\ss{}es Abrunden
\providecommand*{\Rfloor}{\right\rfloor}                % gro\ss{}es Abrunden
\providecommand*{\floor}[1]{\Lfloor #1 \Rfloor}         % gro\ss{}es ganzes Abrunden
\providecommand*{\ceil}[1]{\left\lceil #1 \right\rceil} % gro\ss{}es ganzes Aufrunden

\newcommand{\R}{\mathbb{R}} 
\newcommand{\E}{\mathbb{E}}
\newcommand{\N}{\mathbb{N}} 

\DeclareMathOperator{\e}{ex}
\DeclareMathOperator{\ma}{mate}
\DeclareMathOperator{\Ex}{Ex}
\DeclareMathOperator{\SW}{SW}
\DeclareMathOperator{\NB}{D}
\DeclareMathOperator{\dem}{d}
\DeclareMathOperator{\dom}{dom}
\DeclareMathOperator{\vol}{vol}
\DeclareMathOperator{\interior}{int}
\DeclareMathOperator{\fraction}{frac}
\DeclareMathOperator{\FA}{\mathcal{A}}
\newcommand{\bigO}{\mathcal{O}}

\newcommand{\first}[1]{\left[ #1 \right]}
\newcommand{\ffirst}[1]{\left\llbracket #1 \right\rrbracket}
\newcommand{\Z}{\mathbb{Z}}
\newcommand{\zero}{\mathbf{0}}
\newcommand{\fp}{\mathbf{p}}
\newcommand{\fx}{\mathbf{x}}
\newcommand{\fy}{\mathbf{y}}
\newcommand{\fs}{\mathbf{s}}
\newcommand{\fu}{\mathbf{u}}
\newcommand{\fa}{\mathbf{a}}
\newcommand{\fr}{\mathbf{r}}
\newcommand{\mnath}{\textrm{\textup{M}}^\natural}

\makeatletter
\newcommand*\bcdot{\mathpalette\bcdot@{.5}}
\newcommand*\bcdot@[2]{\mathbin{\vcenter{\hbox{\scalebox{#2}{$\m@th#1\bullet$}}}}}
\makeatother

%%%%%%%%%%%%%%%%%%%%%%%%%%%%%%%%%%%%%%%%%%%%%%%%%%%%%%%%%%%%%%%%%%%%%%%%%%%%%%%%%%%%%%%%%%%
%%%%%%%%%%%%%%%%%%%%%%%%%%%%%%%%%%%%%%%%%%%%%%%%%%%%%%%%%%%%%%%%%%%%%%%%%%%%%%%%%%%%%%%%%%%
% Start des Dokuments
\begin{document}		

\ownautorefnames		% Änderung einiger automatischen Texte von hyperref (wie in referenz.tex definiert)
\parindent0em 			% kein Einzug nach einer Leerzeile

%%%%%%%%%%%%%%%%%%%%%%%%%%%%%%%%%%%%%%%%%%%%%%%%%%%%%%%%%%%%%%%%%%%%%%%%%%%%%%%%%%%%%%%%%%%
% Titelseite
\thispagestyle{empty}		% leerer Seitenstil, also keine Seitennummer
\begin{titlepage}
\seminarkopf 			% Titelblatt (wie in kopf.tex definiert)
\begin{abstract} 
\todo{Hier sollte der Inhalt der Seminararbeit in ein paar Sätzen zusammengefasst werden. }
\end{abstract}
\end{titlepage}

%%%%%%%%%%%%%%%%%%%%%%%%%%%%%%%%%%%%%%%%%%%%%%%%%%%%%%%%%%%%%%%%%%%%%%%%%%%%%%%%%%%%%%%%%%%
% Inhaltsverzeichnis
\thispagestyle{empty}	
\tableofcontents		% Inhaltsverzeichnis
%\listoffigures			% Abbildungsverzeichnis (eventuell einfügen)
%\listoftables			% Tabellenverzeichnis (eventuell einfügen)
\setcounter{page}{0}% Eigentlicher Inhalt beginnt auf Seite 1
\clearpage          % neue Seite für eigentlichen Inhalt
%%%%%%%%%%%%%%%%%%%%%%%%%%%%%%%%%%%%%%%%%%%%%%%%%%%%%%%%%%%%%%%%%%%%%%%%%%%%%%%%%%%%%%%%%%%
% Eigentlicher Inhalt der Seminararbeit; die einzelnen Teile werden hier (aus Gründen der Übersichtlichkeit) über \input{file} eingebunden


\section{Einführung}

\todo{Walras-Gleichgewichte rocken.}
\section{Walras-Gleichgewicht}

Um Walras-Gleichgewichte formal einführen zu können, werden zunächst einige grundlegende Begriffe und Notationen erklärt.

\begin{notation}
	Für $k\in\Z_{\geq 0}$ sei $\first{k}\coloneqq \{ 1, 2, \dots, k \}$ die Menge der ersten $k$ natürlichen Zahlen; um die $0$ hinzuzunehmen, schreibt man $\ffirst{k}\coloneqq \{0,1,\dots, k \}$.
	Für einen Vektor $\fs = (s_1, \dots, s_n)\in\Z_{\geq0}^n$ definiert man $\ffirst{\fs}\coloneqq \Pi_{j\in\first{n}} \ffirst{s_j}$.
	
	Des Weiteren bezeichne $e_i$ den $i$-ten Einheitsvektoren für $i\in\N$ und $e_0 \coloneqq \zero$ den Nullvektor. 
\end{notation}

Das Modell des Marktes besteht hier aus einer Menge $\first{m}$ von $m\geq 2$ \emph{Käufern}, einer Menge $\first{n}$ von \emph{Gütern} sowie aus einem \emph{Angebot} $s_j\in\Z_{>0}$ für jedes Gut $j\in\first{n}$.

Jedem Käufer $i\in\first{m}$ ist eine Bewertungsfunktion $v_i:\Z^n_{\geq 0} \rightarrow \Z$ zugeordnet, die einem $\emph{Bündel}$, also einer Multimenge an Gütern aus $\first{m}$, einen ganzzahligen Wert zuschreibt, wobei $v_i(\zero) = 0$ gilt.

Sind ein \emph{Preisvektor} $\fp\in\R^n$ und ein Bündel $x\in\Z^{m}_{\geq0}$ gegeben, bezeichnet $u_i(x;\fp) \coloneqq v_i(x) - \fp \bcdot x$ den Nutzen von Bündel $x$ bei Preisen $\fp$ für Käufer $i$.
Hierbei ist $\fp \cdot x$ das Skalarprodukt von $\fp$ und $x$.

\begin{definition}[Allokation]
	Eine \emph{Allokation} $\fx\coloneqq (x^{(i)})_{i\in\first{m}}$ weist jedem Käufer $i\in\first{m}$ ein Bündel $x^{(i)} \in\Z^n_{\geq0}$ zu.
	Ist ein Angebotsvektor $\fs\in\Z^m_{\geq0}$ gegeben, nennt man eine Allokation \emph{gültig}, falls sie genau das Angebot verteilt, das heißt, falls $\sum_{i\in\first{m}} x^{(i)} = \fs$ gilt.
	Alle gültigen Allokationen werden in der Menge $\FA$ gesammelt.
	
	Das \emph{soziale Wohl} einer gültigen Allokation $\fx$ ist definiert als $\SW(\fx) \coloneqq \sum_{i\in\first{m}} v_i(x^{(i)})$.
	Eine gültige Allokation mit maximalem sozialen Wohl wird \emph{optimale Allokation} genannt.
\end{definition}

\begin{definition}[Nachfragebereich]
	Man nennt die Menge $\NB(v,\fp)$ der Bündel, die den Nutzen unter einer Bewertungsfunktion $v:\Z^n_{\geq 0} \rightarrow \Z$ bei Preisen $\fp$ maximiert, den Nachfragebereich von $v$ bei Preisen $\fp$.
	Es ist also $ \NB(v, \fp) \coloneqq \arg\max_{x\in\ffirst{\fs}} v(x) - \fp \bcdot x$.
	Für den Nachfragebereich eines Käufers $i\in\first{m}$ wird abkürzend $\NB(i,p):=\NB(v_i, p)$ geschrieben.
\end{definition}

\begin{definition}[Walras-Gleichgewicht]
	Ein Paar $(\fx, \fp)$ bestehend aus einer gültigen Allokation $\fx$ und einem Preisvektor $\fp$ heißt \emph{(Walras-)Gleichgewicht}, falls jedem Käufer ein Bündel aus seinem Nachfragebereich zugewiesen wird, falls also $x^{(i)} \in \NB(i, \fp)$ für alle $i\in \first{m}$ gilt.
	Dabei nennt man $\fp$ einen \emph{Walras-Preisvektor} und $\fx$ eine von $\fp$ induzierte \emph{Walras-Allokation}.
\end{definition}

\begin{bemerkung}
	Im Vergleich zu~\cite{PaesLeme2018} wird hier das Angebot $s_j$ positiv statt nicht-negativ gewählt und es werden mindestens zwei Käufer vorausgesetzt, um den Beweis von Lemma~\ref{lemma-prices-bounded-M} bzw. von~\cite[Lemma 4]{PaesLeme2018} zu ermöglichen.
	Die ausgeschlossenen Fälle sind jedoch uninteressant.
\end{bemerkung}

In einem Walras-Gleichgewicht gibt es also für keinen der Käufer eine für ihn bessere Allokation: Das Bündel, das dem Käufer zugeteilt wird, hat bei den gegebenen Preisen einen für ihn maximalen Nutzen.
Des Weiteren wird durch Gültigkeit der Allokation in einem solchen Gleichgewicht sichergestellt, dass das Angebot und die Nachfrage des gesamten Marktes übereinstimmen: Es bleiben also weder Güter übrig, noch wird die Nachfrage irgendeines Käufers nicht gedeckt.

Darüber hinaus gelten hier die sogenannten Wohlfahrtstheoreme aus der Ökonomik:
Das erste Wohlfahrtstheorem besagt, dass jedes Gleichgewicht bei vollkommenen Wettbewerb, wie er hier unter den Käufern möglich ist, das soziale Wohl maximiert.
Das zweite Wohlfahrtstheorem sagt aus, dass jede optimale Allokation bei Walras-Preisen ein Gleichgewicht erzeugt.

\begin{lemma}[Erstes und zweites Wohlfahrtstheorem]
	Ist $(\fx, \fp)$ ein Gleichgewicht, so ist $\fx$ eine optimale Allokation.
	Ist $\fy$ eine beliebige optimale Allokation, so bildet auch $(\fy, \fp)$ ein Gleichgewicht.
\end{lemma}
\begin{proof}
	Seien $(\fx, \fp)$ ein Gleichgewicht und $\fy$ eine gültige Allokation.
	Für alle $i\in\first{m}$ gilt wegen $x^{(i)} \in \NB(i,\fp)$ dann $v_i(x^{(i)}) - \fp \bcdot x^{(i)} \geq v_i(y^{(i)}) - \fp \bcdot y^{(i)}$.
	Mit der Gültigkeit der Allokationen $\sum_{i\in\first{m}} x^{(i)} = \sum_{i\in\first{m}} y^{(i)} = \fs$ folgere man
	\[
		\sum_{i\in\first{m}} v_i(x^{(i)}) \geq
		\sum_{i\in\first{m}} \left( v_i(y^{(i)}) - \fp \bcdot (y^{(i)} - x^{(i)}) \right)
		= \sum_{i\in\first{m}} v_i(y^{(i)}) - \fp \bcdot (\fs - \fs) = \sum_{i\in\first{m}} v_i(y^{(i)}).
	\]
	Insbesondere ist $\fx$ also eine optimale Allokation.
	
	Ist nun $\fy$ ebenfalls optimal, gilt 
	$ \sum_{i\in\first{m}} v_i(x^{i}) = \sum_{i\in\first{m}} v_i(y^{(i)}) $ und mit der Gültigkeit der Allokationen lässt sich
	\[
		\sum_{i\in\first{m}} \left( v_i(x^{i}) - \fp \bcdot x^{(i)} \right)
		= \sum_{i\in\first{m}} \left( v_i(y^{(i)}) - \fp \bcdot y^{(i)} \right)
	\]
	folgern.
	Da $x^{(i)}\in\NB(i, \fp)$ für die Summanden bereits $v_i(x^{(i)}) - \fp \bcdot  x^{(i)} \geq v_i(y^{(i)}) - \fp \bcdot y^{(i)}$ impliziert, müssen diese bereits exakt übereinstimmen.
	Dadurch folgt auch $y^{(i)}\in\NB(i, \fp)$ für alle $i\in\first{m}$, sodass $\fy$ ebenfalls eine Walras-Allokation zu Preisen $\fp$ ist.
\end{proof}


\section{Berechnung von Gleichgewichten}

\todo{Blabla}

\subsection{Informationszugang zum Markt}

Bei der Berechnung eines Marktgleichgewichts müssen auf irgendeine Weise Informationen erhoben werden.
Dieser Informationszugang kann dann von Algorithmen als Schnittstelle zur Berechnung eines Gleichgewichts verwendet werden.
Die Algorithmen werden also unter der Annahme entwickelt, dass ein Orakel existiert, welches eine solche Schnittstelle zum Markt implementiert.

Da der Detailgrad der erfassbaren Informationen bei vielen Marktsituationen unterschiedlich ist, unterscheiden Leme und Wong in~\cite{PaesLeme2018} drei verschiedene Modelle:
\begin{itemize}
	\item \emph{Die Mikroskopische Sicht:} Hier kann der Wert einzelner Bündel für jeden Käufer abgefragt werden.
	Das sogenannte \emph{Wert-Orakel} bestimmt also anhand eines Käufers $i\in\first{m}$ und eines Bündels $x\in\ffirst{\fs}$ den Wert $v_i(x)$.
	\item \emph{Die Agenten-Sicht:} In diesem Modell besteht der Zugang zum Markt daraus, von jedem Käufer ein von ihm nachgefragtes Bündel bei gegebenen Preisen abfragen zu können.
	Sollten mehrere Bündel für den Käufer gleichwertig sein, so wird ein beliebiges dieser Bündel ausgegeben.
	Das \emph{Nachfrage-Orakel} berechnet also anhand eines Preisvektors $\fp\in\R^n$ und eines Käufers $i\in\first{m}$ ein Bündel $\dem_i(\fp)\in\NB(i, \fp)$.
	\item \emph{Die Makroskopische Sicht:} Hier können keine Informationen einzelner Käufer abgefragt werden, sondern nur noch auf die aggregierte Nachfrage bei gegebenen Preisen.
	Das heißt, das sogenannte \emph{Aggregierte-Nachfrage-Orakel} liefert bei Eingabe eines Preisvektors $\fp\in\R^n$ einen Nachfrage-Vektor $\dem(\fp)\in\Z_{\geq0}^n$, für den nachgefragte Bündel $x^{(i)} \in \NB(i,\fp)$ mit $\dem(p)=\sum_{i\in\first{m}} x^{(i)}$ existieren.
\end{itemize}

Es sei erwähnt, dass man anhand eines Wert-Orakels ein Nachfrage-Orakel konstruieren kann:
Sind Preise $\fp$ sowie ein Käufer $i$ gegeben, kann für jedes Bündel $x$ der Nutzen $u_i(x; \fp)$ durch je eine Abfrage des Wert-Orakels berechnet werden und ein Bündel ausgegeben werden, welches diesen Wert maximiert.
Für eine Abfrage des Nachfrage-Orakels entstehen dann $\ffirst{\fs}$ Abfragen des Wert-Orakels.

Genauso lässt sich aus einem Nachfrage-Orakel auch ein Aggregierte-Nachfrage-Orakel gewinnen: So kann man mit $m$ Abfragen des Nachfrage-Orakels für alle $i\in\{m\}$ ein nachgefragtes Bündel erhalten, welche summiert den aggregierten Nachfrage-Vektor ergeben.

\todo{Im Folgenden wird ein Algorithmus diskutiert, der das Aggregierte-Nachfrage-Modell verwendet, um Walras-Preise für allgemeine Bewertungsfunktionen zu berechnen.}

\subsection{Darstellung als Lineares Optimierungsproblem}

Man betrachte die Relaxation der Bestimmung von optimalen Allokationen als lineares Programm:
\begin{align*}
	\tag{P}\label{LP}
	&\max_{z} \sum_{i\in\first{m}, x\in\ffirst{\fs}} v_i(x) \cdot z_{i, x} \\[5pt]
	\text{udN.} \quad & \sum_{x\in\ffirst{\fs}} z_{i, x} = 1 & \text{für alle $i\in\first{m}$}\\[5pt]
	& \sum_{i\in\first{m}, x\in\ffirst{\fs}} x_j \cdot z_{i,x} = s_j & \text{für alle $j\in\first{n}$} \\[5pt]
	& z_{i, x} \geq 0 &\text{für alle $i\in\first{m}, x\in\ffirst{\fs}$}
\end{align*}
Führt man für die ersten $m$ Bedingungen die Variablen $(u_i)_{i\in\first{m}}$ und für die nächsten $n$ Bedingungen die Variablen $(p_j)_{j\in\first{n}}$ ein, erhält man folgendes duale Programm:
\begin{align*}
\tag{D}\label{DP}
&\min_{\fp,\fu} \sum_{i\in\first{m}} u_i + \fp \bcdot \fs \\[5pt]
\text{udN.} \quad &  u_i \geq v_i(x) - \fp \bcdot x & \text{für alle $i\in\first{m}, x\in\ffirst{\fs}$}
\end{align*}
\begin{lemma}
	Ein Walras-Gleichgewicht existert genau dann, wenn~\eqref{LP} eine ganzzahlige Optimallösung hat.
	Ist dies der Fall, so ist die Menge der Walras-Preise gerade die Menge der Optimallösungen von~\eqref{DP} projiziert auf die $\fp$-Koordinaten.
\end{lemma}
\begin{proof}
	Zunächst bemerke man, dass die zulässigen ganzzahligen Lösungen von~\eqref{LP} gerade den gültigen Allokationen entsprechen:
	Ist $z$ ganzzahlig und zulässig, so gibt es für alle $i\in\first{m}$ wegen den ersten Bedingungen genau ein Bündel $x^{(i)}\in\ffirst{\fs}$ mit $z_{i,x^{(i)}} = 1$.
	Aus den nächsten $n$ Bedingung folgt dann die Gültigkeit der Allokation $(x^{(i)})_{i\in\first{m}}$.
	Ausgehend von einer Allokation $\fx$ setzt man $z_{i,y} = 1$, falls $y = x^{(i)}$ gilt, und $z_{i,y}=0$ sonst, um zu einer ganzzahligen Lösung von~\eqref{LP} zu gelangen.
	
	Angenommen, es existiere ein Walras-Gleichgewicht $(\fx, \fp)$.
	Transformiert man die Allokation wie oben zu $(z_{i,x})_{i,x}$ und setzt man zusätzlich $u_i = \max_{x\in\ffirst{s}} v_i(x) - \fp \bcdot x$, erhält man eine primal und eine dual zulässige Lösung mit gleichem Zielfunktionswert:
	Es gilt mit $x^{(i)}\in\NB(i, \fp)$ und der Gültigkeit von $\fx$:
	\begin{align*}
		\sum_{i\in\first{m}} u_i + \fp \bcdot \fs
		&= \sum_{i\in\first{m}} \left( \max_{x\in\ffirst{\fs}} v_i(x) - \fp \bcdot x \right) + \fp \bcdot s = \sum_{i\in\first{m}} \left( v_i(x^{(i)}) - \fp\bcdot x^{(i)} \right) + \fp \bcdot \fs \\
		&= \sum_{i\in\first{m}} v_i(x^{(i)}) = \sum_{i\in\first{m}, x\in\ffirst{\fs}} v_i(x) \cdot z_{i,x}.
	\end{align*}
	Aufgrund schwacher Dualität ist $(z_{i,x})_{i,x}$ eine Optimallösung von~\eqref{LP}.
	
	Gibt es umgekehrt eine ganzzahlige Optimallösung, so kann diese wie oben zu einer gültigen Allokation $\fx$ transformiert werden.
	Aufgrund starker Dualität gibt es auch eine Optimallösung $(\fp, \fu)$ von~\eqref{DP} mit Zielfunktionswert $\sum_{i\in\first{m}} v_i(x^{(i)})$.
	Die Optimalität von $(\fp, \fu)$ impliziert bereits $u_i = \max_{x\in\ffirst{\fs}} v_i(x) - \fp \bcdot x$ für alle $i\in\first{m}$.
	Subtrahiert man $\fp \bcdot \fs$ vom Zielfunktionswert erhält man 
	\[
	\sum_{i\in\first{m}} \left( v_i(x^{(i)}) - \fp \bcdot x^{(i)} \right) = \sum_{i\in\first{m}} \max_{x\in\ffirst{\fs}} v_i(x) - \fp \bcdot x,
	\]
	wodurch $x^{(i)}\in\NB(i, \fp)$ für alle $i\in\first{m}$ folgt.
	Daher ist $\fx$ eine Walras-Allokation induziert durch die Preise $\fp$.
	
	Der zweite Teil der Aussage folgt aus dem Beweis des ersten Teils.
\end{proof}

\begin{lemma}
	Das folgende transformierte Programm ist äquivalent zu~\eqref{DP}: \emph{
	\begin{align*}
		\tag{TD}\label{TDP}
		&\min_{\fp, u} u + \fp \bcdot \fs \\[5pt]
	\text{udN.}\quad& u \geq \sum_{i\in\first{m}} \left( v_i(x^{(i)}) - \fp \bcdot x^{(i)} \right) & \text{für alle $\fx = (x^{(i)})_{i\in\first{m}}\in\ffirst{\fs}^m$}
	\end{align*}
}
\end{lemma}
\begin{proof}
	Für eine zulässige Lösung $(\fp, \fu)$ von~\eqref{DP} ist $(\fp, u)$ mit $u\coloneqq \sum_{i\in\first{m}} u_i$ eine zulässige Lösung von~\eqref{TDP} mit gleichem Zielfunktionswert.
	Umgekehrt ist für eine zulässige Lösung $(\fp, u)$ von~\eqref{TDP} mit $u_i\coloneqq \max_{x\in\ffirst{\fs}} v_i(x) - \fp \bcdot x$ auch $(\fp, \fu)$ zulässig für~\eqref{DP} mit einem Wert von höchstens $u+\fp\bcdot \fs$, da die folgende Ungleichung erfüllt ist: \[
		\sum_{i\in\first{m}} u_i = \sum_{i\in\first{m}} \max_{x\in\ffirst{\fs}} \left( v_i(x) - \fp \bcdot x \right) = \max_{\fx\in\ffirst{\fs}^m} \sum_{i\in\first{m}} \left( v_i(x^{(i)}) - \fp \bcdot x^{(i)} \right) \leq u.
	\]
\end{proof}
\begin{bemerkung} 
	In~\cite{PaesLeme2018} wird statt der Nebenbedingung in~\eqref{TDP} die Nebenbedingung $u \geq \sum_{i\in\first{m}} v_i(x) - \fp \bcdot x$ für alle $x\in\ffirst{\fs}$ betrachtet.
	Das entsprechende Problem ist jedoch nicht äquivalent zu~\eqref{DP}:
	Betrachtet man einen Markt mit nur einem Gut mit Angebot $s=1$ und zwei Käufern, deren Bewertungsfunktionen durch $v_1(0)=v_2(0)=0$ und $v_1(1)=1$ sowie $v_2(1) = -1$ gegeben sind, haben die Probleme unterschiedliche Optimalwerte.
\end{bemerkung}

\subsection{Konvexe Darstellung und Subgradienten}

Für die Berechnung von Walras-Preisen genügt es also -- im Falle ihrer Existenz -- einen Minimierer von~\eqref{DP} zu ermitteln.
Dabei können die Variablen $(u_i)_{i\in\first{m}}$ vermieden werden, indem man zu einem konvexen Minimierungsproblem wechselt:
\begin{definition}
	Seien ein Angebot $\fs$ sowie Käufer mit Bewertungen $(v_i)_{i\in\first{m}}$ gegeben. Die \emph{Marktpotenzialfunktion} ist definiert als
	\[ f:\R^n \rightarrow \R,\quad \fp \mapsto \sum_{i\in\first{m}} \left( \max_{x\in\ffirst{\fs}} v_i(x) - \fp \bcdot x \right) + \fp \bcdot \fs. \]
\end{definition}
Die Minimierer dieser Marktpotenzialfunktion sind gerade alle Optimallösungen von~\eqref{DP} projiziert auf die $\fp$-Koordinaten.
Da das Maximum und die Summe konvexer Funktionen wieder konvex sind, ist $f$ ebenfalls konvex.

Eine Möglichkeit, konvexe Funktionen zu optimieren, bildet die sogenannte Subgradienten-Methode.
Daher wird zunächst der Begriff eines Subgradients eingeführt:
\begin{definition}[Subgradient]
	Sei $f:\R^n \rightarrow \R$ eine konvexe Funktion.
	Man nennt $g\in\R^n$ einen \emph{Subgradient von $f$ an $p\in\R^n$}, falls $f(x) \geq f(p) + g\bcdot (x - p)$ für alle $x\in\R^n$ gilt.
\end{definition}
Man bemerke, dass $\fp$ eine konvexe Funktion $f$ genau dann minimiert, wenn der Nullvektor ein Subgradient von $f$ an $\fp$ ist.
\begin{proposition}\label{prop-convex-lipschitz}
Sei $f: \R^n \rightarrow \R$ eine konvexe und bezüglich $\norm{\cdot}$ Lipschitz-stetige Funktion mit Lipschitz-Konstante $L$.
Dann gilt $\norm{g}\leq L$ für jeden Subgradienten $g$ von $f$ an jedem Punkt $p\in\R^n$.
\end{proposition}
\begin{proof}
Es gilt $\norm{g} = g \bcdot (g \cdot \norm{g}^{-1}) \leq  f(g \cdot \norm{g}^{-1} + p) - f(p) \leq L \cdot \norm{g}\cdot \norm{g}^{-1} = L$.
\end{proof}

Zwar ist es schwierig mit dem Aggregierte-Nachfrage-Orakel Auswertungen von $f$ selbst zu machen, jedoch lassen sich sehr wohl Subgradienten von $f$ damit ermitteln:
\begin{lemma}\label{lemma-subgradient}
	Für alle $p\in\R^n$ ist $\fs - \dem(\fp)$ ein Subgradient von $f$ an $\fp$.
\end{lemma}
\begin{proof}
	Sei $\dem(\fp) = \sum_{i\in\first{m}} x^{(i)}$ die aggregierte Nachfrage zum Preis $\fp$.
	Da $f$ eine Summe konvexer Funktionen ist, ist nach~\cite[Theorem~1.12]{Shor1985} jede Summe von Subgradienten der einzelnen Summanden in $\fp$ ein Subgradient von $f$ in $\fp$.
	Weiter ist nach~\cite[Theorem~1.13]{Shor1985} für das Maximum konvexer Funktionen jeder Subgradient einer Funktion, die in $\fp$ das Maximum annimmt, ein Subgradient der Maximumsfunktion in $\fp$.
	Des Weiteren ist für eine differenzierbare Funktion der einzige Subgradient der Gradient der Funktion.
	
	Dies heißt, da $x^{(i)}\in \arg\max_{x\in\ffirst{\fs}} v_i(x) - \fp \bcdot x$ für alle $i\in\first{m}$ gilt, sind $-x^{(i)}$ ein Subgradient von $\fp \mapsto \max_{x\in\ffirst{\fs}} v_i(x) - \fp \bcdot x$ für alle $i\in\first{m}$ und $\fs - \dem(\fp)$ ein Subgradient von $f$ an~$\fp$.
\end{proof}

\begin{lemma}\label{lemma-prices-bounded-M}
	Existiert eine Walras-Allokation $\fx$, so sind für $M\coloneqq \max_{i\in\first{m}, x\in\ffirst{\fs}} \abs{v_i(x)}$ alle Walras-Preise in $[-2M, 2M]^n$ enthalten.
\end{lemma}
\begin{proof}
	Seien $\fp$ ein Walras-Preisvektor und $j\in\first{n}$ gegeben.
	Der Preis $p_j$ übersteigt $2M$ nicht, denn sonst würde kein Käufer $j$ nachfragen:
	Da $s_j$ positiv ist, gibt es einen Käufer $i$ mit $x^{(i)}_j > 0$, für den $x^{(i)} \in\NB(i, \fp)$ folgende Ungleichung impliziert:
	\[
	v_i(x^{(i)}) - \fp \bcdot x^{(i)} \geq v_i(x^{(i)} - e_j) - \fp \bcdot ( x^{(i)} - e_j ).
	 \]
	 Es folgt also $2M \geq v_i(x^{(i)}) - v_i(x^{(i)} - e_j) \geq p_j$.
	 Umgekehrt ist $p_j$ mindestens $-2M$, denn sonst würden alle Käufer jeweils das gesamte Angebot von $j$ nachfragen:
	 Angenommen, es gilt $p_j < -2M$, so folgt $x^{(i)}_j = s_j$ für alle $i\in\first{m}$, denn für $x^{(i)}_j < s_j$ folgt der Widerspruch
	 \[ 
	 	v_i(x^{i})  - \fp \bcdot x^{(i)} < v_i(x^{(i)}) - \fp \bcdot (x^{(i)} + e_j) - 2M \leq v_i(x^{(i)} + e_j) - \fp \bcdot (x^{(i)} + e_j).
	 \]
	 Da es mindestens zwei Käufer gibt, folgt schließlich $p_j \geq -2M$.
\end{proof}

\begin{lemma}\label{lemma-prices-bounded-S}
	Die Menge der Walras-Preise ist ein Polytop, dessen Ecken $\fp$ von der Form $p_j = a_j/b$ mit $a_j,b\in\Z$ und
	$\abs{b} \leq (n+1)!\, (mS)^n$
	für $S = \max_{j\in\first{n}} s_j$ sind.
\end{lemma}
\begin{proof}
	Ergänzt man~\eqref{TDP} um Schlupfvariablen $\fy=(y_{\fx})_{\fx\in\ffirst{\fs}}$, so erhält man Bedingungen der Form $ u + \fp \bcdot \sum_{i\in\first{m}} x^{(i)} + y_{\fx} = \sum_{i\in\first{m}} v_i(x) $ für alle $i\in\first{m}, x\in\ffirst{\fs}$.
	Seien $A$ die Matrix und $b$ der Vektor, die diese Nebenbedingungen in $A \cdot (\fp, u, \fy)^T = b$ zusammenfasst.
	
	
	Aus den Grundlagen linearer Optimierung ist bekannt, dass beschränkte Optimierungsprobleme bei nichtleerem Zulässigkeitsbereich immer optimale Ecklösungen, also optimale Basislösungen, besitzen.
	Die Regel von Cramer besagt, dass optimale Basislösungen $(\fp, u, \fy)$ die Darstellung $p_j = \det(A_B^j) / \det(A_B)$ für $j\in\first{n}\cap B$ haben, wobei $A_B$ die Basisspalten von $A$ sind und $A_B^j$ aus $A_B$ durch Ersetzen der $j$-ten Spalte mit $b_B$ entsteht.
	
	Der Zähler ist also eine ganze Zahl.
	Sind Schlupfvariablen in der Basis $B$, so kann man $\det(A_B)$ jeweils nach diesen Spalten entwickeln, da $A_B$ hier nur aus Einheitsvektoren besteht.
	Es genügt also die Determinante einer $(n+1)\times(n+1)$-Submatrix mit den Spalten von $\fp$ und $u$ zu beschränken:
 	Die Einträge der Spalten von $\fp$ sind hier zwischen 0 und $m\cdot S$, sodass mit der Leibniz-Formel $\abs{\det(A_B)} \leq (n+1)!\, (m S)^{n}$ folgt.
\end{proof}
\begin{bemerkung}
	Im Vergleich zu~\cite[Lemma~5]{PaesLeme2018} ist in der Abschätzung hier die Anzahl der Käufer $m$ enthalten.
	Die Begründung der Schranke $n!\,(nS)^n$ im Beweis von~\cite[Lemma~5]{PaesLeme2018} nutzt ebenfalls die Cramersche Regel von Basislösungen des LPs mit dem Hinweis, die zugehörige Matrix enthalte nur Koeffizienten in $\{0, 1,\dots, S\}$.
	Tatsächlich enthält sie aber Koeffizienten in $\{0, 1, \dots, mS \}$.
	Es kann $\det(A_B)$ sogar den Wert $mS$ annehmen: Dazu betrachte man den Markt mit einem Gut mit $s=1$ und zwei Käufern.
	Besteht $B$ aus den durch $\fx_1 = (0,0)$ und $\fx_2= (1,1)$ induzierten Zeilen, gilt $\det(A_B) = 1\cdot 2 - 0 \cdot 1 = 2$.
	Dies ist zwar kein Gegenbeispiel, da dies nur für Bewertungen mit $v_1(1) = v_2(1)$ eine zulässige Basislösung ist und in diesem Fall $p = \det({A_B^p}) / 2 = (v_1(1) + v_2(2)) / 2$ ganzzahlig wäre.
	Jedoch lässt es Zweifel ob der Korrektheit von~\cite[Lemma~5]{PaesLeme2018} zu.
\end{bemerkung}



\begin{theorem}[Ellipsoid-Methode]
Seien eine konvexe Menge $K\subseteq B_r(\zero)\subseteq \R^n$ und ein Trennorakel gegeben, also ein Orakel, das für einen Punkt $\fp\in B_r(\zero)$ in einer Lauf\-zeit von $T$ entweder die Meldung \glqq\,$\fp\in K$\grqq\ oder eine Halbebene $H=\{ \fp \in\R^n \mid \fa \bcdot \fp \leq b \}$ mit $K\subseteq H$ ausgibt.

Dann lässt sich mit $t$ Abfragen des Orakels entweder ein Punkt $\fp\in K$ oder eine Ellipse mit Maximalvolumen $\exp({-t / ({2n + 1}})) \cdot \vol(B_r(\zero))$, welche die Menge $K$ enthält, ermitteln.
\end{theorem}
\begin{theorem}
	Gilt $\interior(K) \neq \emptyset$ für die Menge $K$ der Walras-Preise im $\R^n$, so kann man mit $\bigO(n\log(Mn) + n^2 \log(n m S))$ Abfragen des Aggregierte-Nachfrage-Orakels und Iterationen der Ellipsoidmethode ein Walras-Preisvektor bestimmen.
\end{theorem}
\begin{proof}
	Nach Lemma~\ref{lemma-prices-bounded-M} sind alle Walras-Preise in der Menge $[-2M, 2M]^n\subseteq B_{r}(\zero)$ mit $r = 2M\sqrt{n}$ enthalten.
	Als Trennorakel dient hier das Aggregierte-Nachfrage-Orakel, welches mit $g \coloneqq s - \dem(\fp_0)$ für $\fp_0\in\R^n$ nach Lemma~\ref{lemma-subgradient} einen Subgradienten von $f$ an $\fp_0$ ermitteln kann.
	Ist $g = \zero$, so kann \glqq$\fp_0 \in K$\grqq\ gemeldet werden.
	Sonst gilt für die Halbebene $H\coloneqq\{ \fp\in\R^n \mid g\bcdot \fp \leq g\bcdot \fp_0 \}$ dann $g\bcdot \fp^* \leq f(\fp^*) - f(\fp_0) + g \bcdot \fp_0 \leq g \bcdot \fp_0$ für alle $\fp^*\in K$, was $K\subseteq H$ impliziert.
	
	Da $\interior(K)$ nichtleer ist, existieren Ecken $\fp_0, \dots, \fp_n$ von $K$, sodass der Simplex $S$, welcher durch diese Ecken aufgespannt wird, positives Volumen hat.
	Dieses Volumen kann mit
	\[ \vol(S) = \abs{ \frac{1}{n!} \det\left(  \begin{array}{ccccc}
		1 & 1 & \cdots & 1 \\	
		\fp_0 & \fp_1 & \cdots & \fp_n
	\end{array}  \right)}  \eqqcolon \abs{ \frac{1}{n!} \det(A) } \]
	berechnet werden.
	Nach Lemma~\ref{lemma-prices-bounded-S} kann der Nenner der Einträge eines jeden Walras-Preisvektors durch $R = (n+1)!\ (mS)^n$ abgeschätzt werden.
	Multipliziert man jede Spalte mit seinem Nenner,
	erhält man eine ganzzahlige Matrix $\tilde{A}$ mit $\lvert \det(\tilde{A}) \rvert \neq 0$,
	welche die Ungleichung $\abs{ \det(A) } \geq R^{-(n+1)} \lvert \det(\tilde{A}) \rvert \geq R^{-(n+1)}$ erfüllt.
	Entsprechend gilt also $\vol(K) \geq \vol(S) \geq R^{-(n+1)} / n! \geq (nR)^{-(n+1)}$.
	
	Gibt die Ellipsoidmethode nach $t$ Iterationen eine Ellipse $E$ aus, welche die Bedingungen $\vol(E) \leq \exp(-t / ({2n + 1}))\cdot (4M \sqrt{n})^n$ und $K\subseteq E$ erfüllt, gilt mit $\vol(E) \geq \vol(K)$:
	\[ \exp(-t / ({2n + 1}))\cdot (4M\sqrt{n})^n \geq (nR)^{-(n+1)}
	~ \Leftrightarrow ~ \exp(t) \leq (4M\sqrt{n})^n \exp(2n+1) (nR)^{n+1} \]
	Insbesondere gilt also $t\in\bigO(n\log(Mn) + n\log(nR)) = \bigO(n\log(Mn) + n^2 \log(n m S))$.
	Daher ist die Ellipsoidmethode dazu gezwungen nach so vielen Iterationen einen Punkt $\fp\in K$ auszugeben.
\end{proof}

\todo{Zitat}
\begin{theorem}\label{thm-cutting-plane-method}
	Seien eine konvexe Menge $K\subseteq [-M,M]^n$ und ein Trennorakel gegeben, also ein Orakel, das für einen Punkt $\fp\in[-M,M]^n$ in einer Laufzeit von $T$ entweder die Meldung \glqq\,$\fp\in K$\grqq\ oder eine Halbebene $H=\{ \fp \in\R^n \mid \fa \bcdot \fp \leq b \}$ mit $K\subseteq H$ ausgibt.
	
	Dann lässt sich in $\bigO(nT\log(nM/\delta) + n^3 \log^{\bigO(1)}(nM)\log^2(nM/\delta))$ Zeit entweder ein Punkt $\fp\in K$ oder ein Polytop $P=\{ \fp\in\R^n \mid \fa_i \bcdot \fp \leq b_i, i\in\first{k} \}$ mit $k=\bigO(n)$ und den folgenden Eigenschaften ermitteln:
	\begin{itemize}
		\item Jede Bedingung $\fa_i \bcdot \fp \leq b_i$ von $P$ ist entweder von der Form $p_i \leq M$ oder $p_i\geq -M$ oder wurde vom Trennorakel erstellt, wobei $\norm{\fa_i}=1$ gilt.
		\item Das Polytop $P$ ist schmal, das heißt, es gibt einen Vektor $\fa$ mit $\norm{\fa}=1$ und 
	\[
	\max_{\fp\in P} \fa \bcdot \fp - \min_{\fp\in P} \fa \bcdot \fp = \bigO(n\delta \log(nM/\delta)).
	\]
		\item Der Algorithmus erstellt ein Zertifikat der Schmalheit, also eine Konvexkombination $t_1,\dots,t_k$ mit $t_i\geq0$ für $i\in\first{k}$ und $\sum_{i\in\first{k}}t_i = 1$, die die folgenden Schranken erfüllt:
		\[
		\norm{\, \sum_{i\in\first{k}} t_i \fa_i \, } = \bigO\left( \frac{\delta}{M} \sqrt{n} \log\left( \frac{M}{\delta} \right) \right)
		\qquad
		\text{und}
		\qquad
		\abs{\sum_{i\in\first{k}} t_i b_i} = \bigO\left(n\delta\log\left(\frac{M}{\delta}\right) \right).
		\]
	\end{itemize}
\end{theorem}
\todo{blaa}

\begin{proposition}\label{prop-trivial-bound-thm}
	Für ein Polytop, das aus dem Algorithmus von Theorem~\ref{thm-cutting-plane-method} hervorgeht, gilt für Punkte $\fp^*\in K$:
	\[
		\sum_{i\in\first{k}} t_i (b_i - \fa_i \bcdot \fp^*) \leq \bigO\left( n\delta\log\left(\frac{M}{\delta}\right) \right).
	\]
\end{proposition}
\begin{proof}
	Mit der Dreiecksungleichung, der Cauchy-Schwarzschen Ungleichung sowie den Beschränkungen aus Theorem~\ref{thm-cutting-plane-method} gilt:
	\begin{align*}
	\sum_{i\in\first{k}}  t_i (b_i - \fa_i \bcdot \fp^*)
	&= \sum_{i\in\first{k}} t_i b_i
	- \left( \sum_{i\in\first{k}} t_i \fa_i \right)\bcdot \fp^*
	\leq
	\abs{\sum_{i\in\first{k}} t_i b_i}
	+ \norm{\sum_{i\in\first{k}} t_i \fa_i} \cdot \norm{p^*}
	\\[5pt]
	&\leq \bigO\left( 
	n\delta \log\left(\frac{M}{\delta}\right)
	+ \frac{\delta}{M}\sqrt{n}\log\left( \frac{M}{\delta} \right)
	\cdot M \sqrt{n}
	\right)
	= \bigO\left(
	n\delta\log\left( \frac{M}{\delta} \right)
	\right)
	\end{align*}
\end{proof}

\begin{theorem}\label{thm-approximate-solution}
	Seien eine konvexe, $L$-Lipschitz-stetige Funktion $f:\R^n\rightarrow \R$ sowie ein Subgradienten-Orakel, welches für einen Punkt $\fp$ mit Laufzeit $T$ einen Subgradienten von $f$ in $p$ ausgibt, gegeben.
	Existiert ein Minimierer von $f$ in $[-M, M]^n$, kann ein Punkt $\bar{\fp}$ mit $f(\bar{\fp}) - \min_{\fp\in\R^n} f(p)\leq \varepsilon$ in $\bigO(n T \log(n M L / \varepsilon) + n^3 \log^{\bigO(1)} (nML) \log^2(nML/\varepsilon) )$ Zeit gefunden werden.
\end{theorem}
\begin{proof}
	Seien zunächst $M'\coloneqq n^{\bigO(1)}M$ und $K\coloneqq \arg\min_{\fp\in\R^n} f(\fp)\cap [-M', M']^n$ die konvexe, nichtleere Menge der Minimierer von $f$ in $[-M',M']^n$.
	Theorem~\ref{thm-cutting-plane-method} wird mit dem erweiterten Quader $[-M', M']^n$ und dem Subgradienten-Orakel angewandt:
	Dieses gibt für einen Punkt $\fp_0$ die Halbebene $H\coloneqq\{ \fp\in\R^n \mid g\bcdot \fp \leq g\bcdot \fp_0 \}$ aus, wobei $g$ ein Subgradient von $f$ in $\fp$ ist.
	Für alle $\fp^*\in K$ gilt dann $g\bcdot \fp^* \leq f(\fp^*) - f(\fp) + g \bcdot \fp \leq g \bcdot \fp$, was $K\subseteq H$ impliziert.
	
	Gibt der Algorithmus einen Punkt in $K$ aus, ist dieser ein Minimierer von $f$ und erfüllt die geforderte Schranke.
	Sonst erhält man ein Polytop $P= \{ \fp\in\R^n \mid \fa_i \bcdot \fp \leq b_i, i\in\first{k} \}$ mit den Eigenschaften aus Theorem~\ref{thm-cutting-plane-method}.
	
	Entstammen alle Ungleichungen des Polytops dem Orakel, so gelten $\fa_i = g_i\cdot \norm{g_i}^{-1}$ und $b_i = g_i\bcdot \fp_i \cdot \norm{g_i}^{-1}$ mit Punkten $\fp_i\in\R^n$ und Subgradienten $g_i$ von $f$ in $\fp_i$ für $i\in\first{k}$.
	Sei $\fp^*\in K$ ein Minimierer von $f$.
	Mit der gegebenen Konvexkombination $(t_i)_{i\in\first{k}}$ erhält man nun einen gesuchten Punkt durch $\bar{\fp} \coloneqq \sum_{i\in\first{k}} t_i \fp_i$:
	\begin{align*}
		f(\bar{\fp}) - f(\fp^*)
		&\leq \sum_{i\in\first{k}} t_i f(\fp_i) - f(\fp^*)
		= \sum_{i\in\first{k}} t_i (f(\fp_i) - f(\fp^*))
		\leq \sum_{i\in\first{k}} t_i g_i \bcdot (\fp_i - \fp^*) \\
		&= L\cdot \sum_{i\in\first{k}}  t_i \fa_i \bcdot (\fp_i - \fp^*)
		\leq \bigO\left(
			Ln\delta\log\left( \frac{M'}{\delta} \right)
		\right)
	\end{align*}
	Die erste Ungleichung folgt aufgrund der Konvexität; die zweite gilt nach Definition von Subgradienten.
	Die dritte gilt nach Proposition~\ref{prop-trivial-bound-thm}
	
	Man kann nun $\delta$ so klein wählen, dass $\varepsilon\leq\bigO(Ln\delta\log(M/\delta))$ gilt.
	\todo{Wie soll ich damit die Laufzeit erklären? Also auf was soll ich delta denn tatsächlich setzen damit das gewünschte raus kommt? Auf das s.u.?!}
	
	Sei nun $B$, die Menge der Indizes der Bedingungen des Polytops $P$ von der Form $p_i \leq M'$ oder $-p_i \leq M'$, nicht leer.
	Für alle $i\in B$ gilt dann wegen $\fa_i\in\{e_i, -e_i \}$ und $b_i=M'$ bereits $b_i - \fa_i\bcdot p^* \geq M' - \abs{p^*_i} \geq M' - M = \Omega(n^{\bigO(1)}M)$.
	Proposition~\ref{prop-trivial-bound-thm} impliziert nun
\[ 
	t_i \Omega(n^{\bigO(1)} M)
	\leq t_i (b_i - \fa_i \bcdot \fp^*)
	\leq \sum_{j\in\first{k}} t_j (b_j - \fa_j \bcdot \fp^*)
	\leq \bigO\left(  n\delta\log\left( \frac{M'}{\delta}  \right) \right),
\]
	wodurch man $t_i\leq \bigO(\delta / (n^{\bigO(1)} M)  \log(M'/\delta))$ folgern kann.
	Wählt man $\delta$ sehr klein, also $\delta=\bigO(\varepsilon/(Ln^{\bigO(1)} M^{\bigO(1)} ))$, \todo{ erhält man $\sum_{i\in B} t_i \leq \bigO(\varepsilon/(n^{\bigO(1)}) M^{\bigO(1)}) \leq 1/2$ und $\sum_{i\in\first{k}} t_i(b_i - \fa_i \bcdot \fp^*) \leq \bigO(\varepsilon / (Ln^{\bigO(1)}M^{\bigO(1)})) $  (?????)}.
	Entfernt man nun die Skalare aus $B$ und skaliert man die restliche Konvexkombination zu $t_i'\coloneqq t_i / (1 - \sum_{j\in B} t_j)$ für alle $i\in\first{k}\setminus B$, ergibt sich:
	\begin{align*}
		\sum_{i\in \first{k}\setminus B} t_i'(b_i - \fa_i \bcdot \fp^*)
		&\leq 2 \left( 1-\sum_{i\in B}t_i  \right) \left( \sum_{i\in \first{k}\setminus B} t_i' (b_i - \fa_i \bcdot \fp^*) \right)
		= 2 \sum_{i\in \first{k}\setminus B} t_i (b_i - \fa_i \bcdot \fp^*)\\
		&\leq 2\sum_{i\in\first{k}} t_i(b_i - \fa_i\bcdot \fp^*)
		\leq \bigO\left(\frac{\varepsilon}{Ln^{\bigO(1)}M^{\bigO(1)}}\right)
		\todo{\leq \frac{\varepsilon}{L}}
	\end{align*}
	Wendet man nun das Argument für den Fall, in dem ausschließlich Orakel-Bedingungen vorkommen, mit der obigen Ungleichung statt Proposition~\ref{prop-trivial-bound-thm} an, so erhält man das gewünschte Ergebnis.
	\todo{Aber in welcher Laufzeit?}
\end{proof}

Diese Methode der Optimierung konvexer Funktionen wird nun genutzt, um das vorgestellte Probleme sogar exakt zu lösen.
\todo{bla bei konvexen allgemein gehts nicht, bei linearen schon}
Dazu werden zunächst Störungen bei der Zielfunktion zugelassen, anschließend eine angenäherte Lösung mittels Theorem~\ref{thm-approximate-solution} berechnet und schließlich durch gezieltes Runden eine exakte Lösung ermittelt.

Die Störung erfolgt, indem das transformierte duale Problem~\eqref{TDP} durch einen Störungs\-vektor $\fr\in\R^n$ ergänzt wird:
\begin{align*}
	\label{PDP}\tag{PD}
	&\min_{\fp, u} u + \fp \bcdot (\fs + \fr)\\[5pt]
	\text{udN.}\quad
	& u\geq \sum_{i\in\first{m}} \left( v_i(x^{(i)}) -\fp \bcdot x^{(i)} \right)
	& \text{für alle $\fx=(x^{(i)})_{i\in\first{m}}\in\ffirst{\fs}^m$}
\end{align*}

\begin{lemma}
	Gilt \todo{ $r_i < (nS)^{-(2n+1)}$ } für alle $i\in\first{n}$, so ist eine optimale Lösung von~\eqref{PDP} auch optimal für~\eqref{TDP}.
\end{lemma}
\begin{proof}
	Es sei $\mathcal{C}$ die Menge der Ecken des Zulässigkeitsbereichs von~\eqref{TDP}.
	Die Einträge solcher Ecken sind nach Lemma~\ref{lemma-prices-bounded-S} von der Form $a_j / b$ mit $a,b\in\Z$ und \[
	\abs{b} \leq (n+1)!\, (mS)^n \leq ((n+1)mS)^{n+1} \eqqcolon q.
	\]
	Da die lineare Zielfunktion von~\eqref{TDP} nur ganzzahlige Koeffizienten hat, ist auch der Zielfunktionswert ein Bruch, dessen Nenner durch $q$ beschränkt ist.
	Der Zielfunktionswert $a/b$ einer optimalen Ecklösung unterscheidet sich also vom Wert $a'/b'$ einer nicht-optimalen Ecklösung mindestens um $\abs{a / b - a' / b'} = \abs{ (ab' - a'b) / (b b') } \geq 1/q^2$.
	
	\todo{Da $r_i < (nS^{-2n-2})$ für alle $i\in\first{n}$ gilt, beläuft sich die durch die Störung verursachte Differenz des Zielfunktionswerts auf weniger als $nS\cdot (nS)^{-2n-2} = (nS)^{-2n}$.
	Daher ist eine Optimalecklösung von~\eqref{PDP} immer auch eine Optimalecklösung von~\eqref{TDP}.
	
	VORSCHLAG:
	
	Da $r_i < (2qM)^{-1}$ für alle $i\in\first{n}$ gilt, beläuft sich die durch die Störung verursachte Differenz des Zielfunktionswerts nach Lemma~\ref{lemma-prices-bounded-M} auf weniger als $n \cdot (2M)\cdot (2qM)^{-1} \leq n/q \leq 1/q^2$.
	Daher ist eine Optimalecklösung von~\eqref{PDP} immer auch eine Optimalecklösung von~\eqref{TDP}.
}
\end{proof}

\section{Brutto-Substituts-Bewertungen}

In diesem Abschnitt werden die Bewertungsfunktionen der Käufer eingeschränkt.
In vielen Marktmodellen werden nur \glqq ähnliche\grqq\ Güter betrachtet.
Diesen unterstellt man, dass bei steigenden Preisen des einen Guts ein anderes Gut dieses substituieren kann.
Insbesondere geht man davon aus, dass, sollten die Preise anderer Güter steigen, die Nachfrage der Güter mit gleichbleibendem Preis nicht sinkt.

\iffalse
\begin{definition}[Diskret-Konkave Funktion]
	Eine Funktion $v: \ffirst{\fs}^n \rightarrow \Z$ heißt \emph{diskret-konkav}, falls lokale Minima auch global minimal sind, also falls für alle Preise $\fp\in\R^n$ und Bündel $x\in\ffirst{\fs}^n$ mit \begin{align*}
	&v(x) \geq \max_{i : x_i>0} v(x - e_i) + p_i, \qquad
	&v(x) \geq \max_{j:x_j < s_j} v(x + e_j) - p_j, \\
	&v(x) \geq \max_{\substack{i: x_i>0 \\ j: x_j < s_j}} v(x + e_i - e_j) - p_i + p-j
	\end{align*}
	bereits $x\in\NB(v, \fp)$ gilt.
\end{definition}
\fi

\begin{definition}[Brutto-Substituts-Bewertung]
Eine Bewertungsfunktion $v: \ffirst{s} \rightarrow \Z$ heißt \emph{Brutto-Substituts-Bewertung} (engl. gross substitutes valuation), falls es für Preise $\fp,\fp' \in\R^n$ mit $\fp' \geq \fp$ und für ein nachgefragtes Bündel $x\in\NB(v, \fp)$ zu Preisen $\fp$ ein nachgefragtes Bündel $y\in\NB(v,\fp')$ zu Preisen $\fp'$ existiert, welches $y_j \geq x_j$ für alle $j\in\first{n}$ mit $p_j = p_j'$ erfüllt.
\end{definition}

Einige wichtige Resultate über Brutto-Substituts-Bewertungen wurden nur für den Fall gezeigt, dass jedes jedes Gut genau einmal angeboten wird, also für $s_j = 1$ für alle $j\in\first{n}$.
Die Theorie von Brutto-Substituts-Bewertungen wurde für diesen Fall maßgeblich von Kelso und Crawford in~\cite{KelsoCrawford} entwickelt.
Man kann den Fall mehrfachen Angebots einzelner Güter darauf herunterbrechen, indem man jedes Vorkommen eines Guts unabhängig bepreisen lässt:
\newcommand{\tild}[1]{\widetilde{#1}}
\begin{definition}[Unabhängige Brutto-Substituts-Bewertung]
	Eine Bewertungsfunktion $v: \ffirst{s} \rightarrow \Z$ heißt \emph{unabhängige Brutto-Substituts-Bewertung}, falls \[
	\tild{v}: \ffirst{1}^{\sum_{i\in\first{n}}s_i} \rightarrow \Z,
	\quad (x_{i,j})_{i\in\first{n}, j\in\first{s_i}} \mapsto v\left( \left( \sum_{j\in\first{s_i}} x_{i,j} \right)_{i\in\first{n}} \right)
	\]
	eine Brutto-Substituts-Bewertung ist.
\end{definition}
\begin{proposition}
	Jede unabhängige Brutto-Substituts-Funktion ist eine Brutto-Substituts-Funktion.
	Es gibt aber Brutto-Substituts-Funktionen, die nicht unabhängig sind.
\end{proposition}
\begin{proof}
	Für Preise $\fp,\fp'\in\R^n$ mit $\fp' \geq \fp$ definiere man $\tild{\fp}\coloneqq (p_i)_{i\in\first{n},j\in\first{s_i}}$ und analog $\tild{\fp'}\coloneqq (p_i')_{i\in\first{n},j\in\first{s_i}}$.
	Für ein Bündel $x\in\NB(v,\fp)$ ist $\tild{x}$ mit $\tild{x}_{i,j} = 1$ für $i\in\first{n}$ und $j\leq x_i$ in $\NB(\tild{v}, \tild{\fp})$ enthalten.
	Es existiert ein Bündel $\tild{y}$ mit $\tild{y}_{i,j} \geq \tild{x}_{i,j}$ für alle $i,j$ mit $p_i \geq p_i'$ und $j\in\first{s_i}$ in $\NB(\tild{v}, \tild{\fp'})$.
	Dementsprechend ist $y\coloneqq(\sum_{j\in\first{s_i}} \tild{y}_{i,j})_{i\in\first{n}}$ in $\NB(v, \fp')$ und es gilt $y_i \geq x_i$ für alle $i\in\first{n}$, sodass $v$ die Brutto-Substituts-Eigenschaft erfüllt.
	
	Nun ein Beispiel einer nicht unabhängigen Brutto-Substituts-Funktion: Wir betrachten einen Markt mit nur einem Gut mit $s=2$ und der Bewertungsfunktion $v(k)\coloneqq k^2$, welche automatisch eine Brutto-Substituts-Funktion ist.
	Setzt man Preise $\tild{p}_j \coloneqq 2$ für $j\in\first{2}$ fest,
	so ist der Nutzen beim Kauf beider Einheiten maximal; es gilt $(1,1)\in \NB(\tild{v}, \tild{\fp})$.
	Erhöht man den Preis der ersten Einheit auf $\tild{p}'_1 \coloneqq 3$ und behält den Preis der zweiten Einheit bei, so ist $(0,0)$ das einzige nachgefragte Bündel.
	Daher ist $\tilde{v}$ keine Brutto-Substituts-Funktion.
\end{proof}

\begin{bemerkung}
	In~\cite{PaesLeme2018} wurden Brutto-Substituts-Bewertungen für allgemeine Angebote $\ffirst{\fs}$ direkt als unabhängige Brutto-Substituts-Bewertungen definiert.
	In~\cite{AkiyoshiShioura2015} wird diese stärkere Version als \glqq strong gross-substitute-condition $(SS)$\grqq\ bezeichnet.
\end{bemerkung}

\subsection{$\mnath$-Konkave und Matroidale Funktionen}\label{section-m-concavity}

Eine elementare Charakterisierung von Brutto-Substituts-Bewertungen findet sich in der diskreten Analysis.
So entspricht das Konzept der $\mnath$-Konkavität, also einer diskreten Version von Konkavität, in gewisser Weise der Brutto-Substituts-Eigenschaft.

Dazu wird hier die Beschreibung der Eigenschaft von Murota aus~\cite[Theorem~6.2 bzw.~Abschnitt~11.3]{Murota2003} eingeführt.

\begin{definition}
	Es bezeichne $\dom(v)\coloneqq \{ z \mid v(z)\in\R \}$ die \emph{Domäne}, also den reellwertigen Bereich, einer Funktion $v:\Z^n \rightarrow \R \cup \{ -\infty\}$.
\end{definition}

\begin{definition}[$\mnath$-Konkavität]
	Eine Funktion $v:\Z^n \rightarrow \R \cup \{-\infty\}$ mit $\dom(v) \neq \emptyset$ heißt \emph{$\mnath$-konkav} (\glqq M-natürlich konkav\grqq), falls \[
	v(x) + v(y) \leq
	\max \left\{
		v(x - e_i) + v(y + e_i),
		\max_{j: x_j < y_j} v(x - e_i + e_j) + v(y + e_i - e_j)
	\right\}
	\] für alle $x,y\in\dom(f)$ und $i\in\first{n}$ mit $x_i > y_i$ gilt.
	Hier gelte $\max(\emptyset)=-\infty$.
\end{definition}


In diesem Abschnitt werden meist Funktionen $v: D\rightarrow \R$ mit $D\subseteq \Z^n$ betrachtet.
Um die Definition oben anwenden zu können, werden solche Funktionen $v$ hier als Funktionen $v: \Z^n\rightarrow \R \cup \{ -\infty\}$ interpretiert, wobei $v(x) = -\infty$ für $x\notin D$ gilt.
Hier gilt $D = \dom(v)$.

Murota gibt in~\cite[Theorem~6.24 bzw. Abschnitt~11.3]{Murota2003} eine äquivalente Beschreibung von $\mnath$-konkaven Funktionen, die für die Anwendung hier sehr nützlich erscheint:

\begin{theorem}\label{thm-char-m-concave}
	Eine Funktion $v:\Z^n \rightarrow \R \cup \{ -\infty \}$ ist genau dann \emph{M}$^\natural$-konkav, wenn für alle $\fp\in\R^n$ und $x,y\in\dom(v)\neq\emptyset$ die Ungleichung $v(x) - \fp\bcdot x < v(y) - \fp \bcdot y$ bereits \[ 
		v(x) - \fp\bcdot x < 
			\max_{i:\, x_i > y_i \,  \vee \, i = 0} \,\,
				\max_{j:\, x_j < y_j \, \vee \, j=0} \,\,
					v(x - e_i + e_j) - \fp \bcdot (x - e_i + e_j)
	\]
	impliziert.
\end{theorem}
\begin{korollar}\label{cor-concave-local-global}
	Es ist $v: \Z^n \rightarrow \R \cup \{ -\infty \}$ genau dann \emph{M}$^\natural$-konkav, wenn für alle $\fp\in\R^n$ lokale Maxima von $x \mapsto v(x) - \fp \bcdot x$ bereits global maximal sind, also 
	falls für alle $\fp\in\R^n$ und $x\in\dom(v)$ mit $
	v(x)\geq \max_{i,j\in\ffirst{n}} v(x - e_i + e_j) + p_i - p_j
	$
	bereits $v(x) - \fp \bcdot x \geq v(y) - \fp \bcdot y$ für alle $y\in\dom(v)$ gilt.
\end{korollar}
\begin{proof}
	Es gelte $v(x) \geq max_{i,j\in\ffirst{n}} v(x - e_i + e_j) + p_i - p_j$ und man nehme zusätzlich $v(x) - \fp\bcdot x < v(y) - \fp \bcdot y$ an.
	Es ist $y\in\dom(v)$ und nach Theorem~\ref{thm-char-m-concave} gibt es $i,j\in\ffirst{n}$ mit $v(x) - \fp \bcdot x < v(x - e_i + e_j) - \fp\bcdot (x - e_i + e_j)$, was der lokalen Maximalität widerspricht.	
\end{proof}

Ein wichtiges Resultat, das die Brutto-Substituts-Eigenschaft mit $\mnath$-Konkavität in Beziehung stellt, liefert~\cite[Theorem 2.1]{Fujishige2003}:

\begin{theorem}\label{thm-fujishige-gs-iff-concave}
	Eine Bewertungsfunktion $\tild{v}:\ffirst{1}^n\rightarrow \Z$ erfüllt genau dann die Brutto-Substituts-Eigenschaft, wenn sie $\mnath$-konkav ist.
\end{theorem}

Damit lässt sich auch die $\mnath$-Konkavität unabhängiger Brutto-Substituts-Bewertungen zeigen.

\begin{korollar}\label{cor-indep-gs-m-concave}
	Jede unabhängige Brutto-Substituts-Funktion $v:\ffirst{\fs} \rightarrow \Z$ ist $\mnath$-konkav.
\end{korollar}
\begin{proof}
	Nach Theorem~\ref{thm-fujishige-gs-iff-concave} ist $\tild{v}$ schon $\mnath$-konkav.
	Man zeige die notwendigen Voraussetzungen in~\ref{cor-concave-local-global} für $v$:
	Seien $x\in\ffirst{\fs}$ und $\fp\in\R^n$ mit $v(x)\geq \max_{i,j\in\ffirst{n}} v(x - e_i + e_j) + p_i - p_j$ gegeben.
	Man definiere das Bündel $\tild{x}\in\ffirst{1}^{\sum_{i\in\first{n}} s_i}$ mit $\tild{x}_{i,j} \coloneqq 1$ für $j\in\first{x_i}$ und $\tild{x}_{i,j} \coloneqq 0$ für $x_i < j \leq s_i$ sowie die Preise $\tild{p}_{i,j}\coloneqq p_i$ für alle $i\in\first{n}$ und $j\in\first{s_i}$.
	Dann
	gilt \[
	\tild{v}(\tild{x}) \geq \tild{v}(\tild{x} - e_k + e_l) + \tild{p}_k - \tild{p}_l
	\] für alle $k, l \in \{ (i,j) \mid i\in\first{n}, j\in\first{s_i} \}\cup\{ 0 \}$, da diese Ungleichung nur von Gütern $i\in\first{n}$ und nicht von $j\in\first{s_i}$ abhängt und sich dadurch die Voraussetzungen an $x$ ausnutzen lassen.
	Aufgrund der $\mnath$-Konkavität von $\tild{v}$ gilt $\tild{x}\in\NB(\tild{v}, \tild{\fp})$ nach Korollar~\ref{cor-concave-local-global}. Daraus folgt auch $x\in\NB(v,\fp)$.
\end{proof}
\begin{bemerkung}
	Nach~\cite[Theorem~11.4]{Murota2003} erfüllt jede $\mnath$-konkave Funktion die Brutto-Substituts-Eigenschaft.
	Ob diese auch unabhängig ist, bleibt an dieser Stelle offen.
\end{bemerkung}

Wie sich zeigt, ist $\mnath$-Konkavität nicht die einzige interessante Charakterisierung von Brutto-Substituts-Funktionen.
So lässt sich auch das Konzept von sogeannten matroidalen Funktionen wiederfinden:
\begin{definition}[Matroidale Funktion]
	Eine Funktion $v:\ffirst{\fs}^n\rightarrow \Z$ heißt \emph{matroidal}, falls
	der Greedy-Algorithmus für alle Preise $\fp\in\R^n$ ein Bündel aus $\NB(v, \fp)$ berechnet.
	\begin{description}
		\item[Greedy-Algorithmus:] Initialisiere das Bündel $x$ mit $x\leftarrow \zero$.
		Solange ein $i\in\first{n}$ existiert, das $x_i < s_i$ und $v(x + e_i) - p_i - v(x) > 0$ erfüllt,
		aktualisiere $x$ mit $x\leftarrow x + e_{i^*}$, wobei $i^*$ ein Index mit $x_{i^*}=0$ sei, der $v(x + e_{i^*}) - p_{i^*}$ mit $x_{i^*} = 0$ maximiert.
	\end{description}
\end{definition}

In~\cite{PaesLeme2017} wird nun folgende Äquivalenz bewiesen:
\begin{theorem}\label{thm-matroidal-iff-single-unit-gs}
	Eine Funktion $\tild{v}:\ffirst{1}^n \rightarrow \Z$ ist genau dann matroidal, wenn sie eine Brutto-Substituts-Bewertung ist.
\end{theorem}
Dieses Ergebnis lässt sich auf unabhängige Brutto-Substituts-Bewertungen erweitern.
Außerdem kann man in diesem Fall die Simulation eines Aggregierte-Nachfrage-Orakels durch ein Wert-Orakel deutlich beschleunigen:
Für allgemeine Bewertungen benötigt man für eine Abfrage des Aggregierte-Nachfrage-Orakels nach Abschnitt~\ref{section-market-access} $m\cdot\abs{\ffirst{\fs}}$ Abfragen des Wert-Orakels.
\begin{korollar}\label{cor-indep-gs-matroidal}
	Eine unabhängige Brutto-Substituts-Bewertung ist matroidal.
	Insbesondere kann man mit $mnS$ Abfragen eines Wert-Orakels einen aggregierten  Nachfrage-Vektor berechnen, falls alle Käufer eine unabhängige Brutto-Substituts-Bewertung haben.
\end{korollar}
\begin{proof}
	Für eine unabhängige Brutto-Substituts-Funktion $v:\ffirst{\fs} \rightarrow \Z$ ist $\tild{v}$ nach Theorem~\ref{thm-matroidal-iff-single-unit-gs} matroidal.
	Zu Preisen $\fp$ kann also mit dem Greedy-Algorithmus ein Vektor $\tild{x}\in\NB(\tild{v}, \tild{\fp})$ mit $\tild{\fp} \coloneqq (p_i)_{i\in\first{n}, j\in\first{s_i}}$ berechnet werden.
	Dieser kann mit $x=(\sum_{j\in\first{s_i}} x_j)_{i\in\first{n}}$ in ein Nachfragebündel aus $\NB(v, \fp)$ transformiert werden.
	Es ist leicht zu sehen, dass der Greedy-Algorithmus -- führt man ihn auf der Eingabe $(v,\fp)$ aus -- diese Aggregation bereits während der Berechnung durchführt und das gleiche Ergebnis ausgibt.
	
	Um also ein nachgefragtes Bündel $\dem_i(\fp)$ zu berechnen, benötigt der Greedy-Algorithmus auf der Eingabe $(v, \fp)$ bis zu $\sum_{i\in\first{n}} s_i \leq nS$ Durchläufe der Schleife, welche selbst jeweils bis zu $n$ Auswertungen von $v$, also bis zu $n$ Abfragen des Wert-Orakels, tätigt.
	Für einen aggregierten Nachfrage-Vektor $d(\fp) = \sum_{i\in\first{m}} d_i(\fp)$ werden also maximal $mnS$ Abfragen benötigt.
\end{proof}


\subsection{Auswirkung auf Walras-Preise}

Mit den Resultaten aus Abschnitt~\ref{section-m-concavity} lässt sich nun folgende Eigenschaft über die Menge der Walras-Preise zeigen, falls alle Käufer eine unabhängige Brutto-Substituts-Bewertung haben:

\begin{theorem}\label{thm-walras-prices-integral-polytope}
	Haben alle Käufer unabhängige Brutto-Substituts-Bewertungen, so sind alle Ecken des Zulässigkeitsbereichs von~\eqref{TDP} ganzzahlig.
	Insbesondere ist in dem Fall die Menge der Walras-Preise ein ganzzahliges Polytop.
\end{theorem}
Der Beweis dieses Theorems stellt eine zulässige Lösung durch Rundung als Konvexkombination ganzzahliger Lösungen dar.
Dafür sind folgende Notation und Proposition hilfreich:
\begin{notation}
	Der Nachkommaanteil einer Zahl $a\in\R$ sei notiert als $\fraction(a)\coloneqq a - \floor{a}$.
\end{notation}
\begin{proposition}\label{prop-integral-polytop-helper}
	Seien $p_1,p_2\in\R$ gegeben und sei $\theta$ zufällig aus dem Intervall $[0,1]$ gewählt.
	Setzt man $\hat{p_i}\coloneqq \ceil{p_i}$ für $\fraction(p_i)> \theta$ und $\hat{p_i}\coloneqq \floor{p_i}$ für $\fraction(p_i) \leq \theta$, so erfüllt der Erwartungswert $\E[\hat{p}_i] = p_i$ für $i\in\first{2}$. Weiter gilt $\hat{p}_1 - \hat{p}_2 \in \{ \floor{p_1 - p_2}, \ceil{p_1 - p_2} \}$.
\end{proposition}
\begin{proof}
	Ist $p_i$ ganzzahlig, so gilt $\hat{p_i} = p_i$.
	Für $p_i\notin\Z$ gilt $\ceil{p_i} = \floor{p_i} + 1$.
	Die Wahrscheinlichkeit für $\hat{p_i} = \ceil{p_i}$ ist $\fraction(p_i)$ und die Wahrscheinlichkeit für $\hat{p_i} = \floor{p_i}$ ist $(1-\fraction(p_i))$.
	Dementsprechend gilt für den Erwartungswert: \begin{align*}
	\E(\hat{p_i})
	&= \fraction(p_i) \cdot \ceil{p_i} + (1-\fraction(p_i))\cdot \floor{p_i}
	= \fraction(p_i)\cdot (\floor{p_i} + 1) + (1-\fraction(p_i)) \floor{p_i} \\
	&= \fraction(p_i) + \floor{p_i} = p_i.
	\end{align*}
	
	Die zweite Aussage lässt sich in drei Fälle aufgeteilt zeigen:
	\begin{description}
		\item[1. Fall:] Die Zahlen $p_1$ und $p_2$ werden beide auf- oder beide abgerundet.
		Bei Aufrundung gilt $\hat{p_1} - \hat{p_2} = \ceil{p_1 - p_2}$ für $\fraction(p_1) \geq \fraction(p_2)$ und $\hat{p_1} - \hat{p_2} = \floor{p_1 - p_2}$ sonst.
		Bei Abrundung ist dies umgekehrt.
		\item[2. Fall]: Es gilt $\fraction(p_1) \leq \theta < \fraction(p_2)$.
		Dann ist $p_2$ nicht ganzzahlig und es gilt die Gleichung $\hat{p_1} - \hat{p_2} = \floor{p_1} - \ceil{p_2} = \floor{p_1} - \floor{p_2} - 1 = \floor{p_1 - p_2}$.
		\item[3. Fall:] Es gilt $\fraction(p_2) \leq \theta < \fraction(p_1)$. Analog gilt $\hat{p_1} - \hat{p_2} = \floor{p_1} + 1 - \floor{p_2} = \ceil{p_1 - p_2}$.
	\end{description}
	\vspace{-0.5em}
\end{proof}
Diese Aussagen lassen nun den folgenden Beweis zu:
\begin{proof}[Beweis von Theorem~\ref{thm-walras-prices-integral-polytope}]
	Sei $(u, \fp)$ ein zulässiger Punkt von~\eqref{TDP}.
	Das heißt es gilt $u \geq \sum_{i\in\first{m}} ( v(x^{(i)}) - \fp \bcdot x^{(i)})$ für alle $\fx = (x^{(i)})_{i\in\first{m}}\in\ffirst{\fs}^m$.
	Sei nun $\fx = (x^{(i)})_{i\in\first{m}}$ eine Allokation aus $\arg\max_{\fx\in\ffirst{\fs}^m} \sum_{i\in\first{m}} (v(x^{(i)}) - \fp \bcdot x^{(i)})$ und man definiere die Differenz $w \coloneqq u - \sum_{i\in\first{m}} (v_i(x^{(i)}) - \fp \bcdot x^{(i)})$.
	Man beachte, dass $w \geq 0$ gilt.
	
	Nun wird die folgende Zufallsverteilung definiert:
	Es wird ein Wert $\theta$ zufällig aus dem Intervall $[0,1]$ entnommen.
	Wie in Proposition~\ref{prop-integral-polytop-helper} wird $\hat{p_i}$ auf $\ceil{p_i}$ gesetzt, falls $\fraction(p_i)$ größer als $\theta$ ist, und sonst auf $\floor{p_i}$ für alle $i\in\first{m}$.
	Analog wird $\hat{w}$ definiert.
	Schließlich wird $\hat{u}\coloneqq\hat{w}+ \sum_{i\in\first{m}} (v_i(x^{(i)}) - \hat{\fp} \bcdot x^{(i)})$ definiert.
	Nach Proposition~\ref{prop-integral-polytop-helper} gilt mit der Linearität des Erwartungswerts auch $\E[(\hat{u},\hat{\fp})] = (u, \fp)$.
	Dabei ist der Erwartungswert tatsächlich eine Konvexkombination endlich vieler, genauer maximal $\abs{ \{ \fraction(w) \} \cup \{ \fraction(p_i) \mid i\in\first{n} \} } + 1$ vieler ganzzahliger Vektoren.
	Es genügt also zu zeigen, dass jeder dieser Vektoren für~\eqref{TDP} zulässig ist.
	
	Seien also ein $\theta\in [0,1]$ fest und $\hat{\fp}$ sowie $\hat{w}$ die entsprechend $\theta$ gerundeten Werte von $\fp$ und $w$.
	Es wird gezeigt, dass $x^{(i)}$ für alle $i\in\first{m}$ ein nachgefragtes Bündel von Käufer $i$ zu Preisen $\hat{\fp}$ ist.
	Mit $\hat{u}\geq \sum_{i\in\first{m}}(v_i(x^{(i)}) - \hat{\fp}\bcdot x^{(i)})$ wegen $\hat{w}\geq0$ folgt dann die Behauptung.
	
	Sei ein $i\in\first{m}$ gegeben. Wegen $x^{(i)}\in\NB(v_i, \fp)$ gilt $v_i(x^{(i)}) \geq v_i(x^{(i)} + e_j - e_k) - \fp \bcdot (e_j - e_k)$ für alle $j,k\in\ffirst{n}$.
	Diese Ungleichung bleibt bei Rundung von $\fp$ zu $\hat{\fp}$ erhalten, weil $\hat{\fp}\bcdot (e_j - e_k)$ nach Proposition~\ref{prop-integral-polytop-helper} in $\{ \floor{\fp\bcdot(e_j - e_k)}, \ceil{\fp\bcdot(e_j - e_k)} \}$ liegt und die rest\-lichen Terme der Ungleichung ganzzahlig sind.
	Daher ist $x^{(i)}$ ein lokales Maximum und nach den Korollaren~\ref{cor-concave-local-global} und~\ref{cor-indep-gs-m-concave} aufgrund der unabhängigen Brutto-Substituts-Eigenschaft von $v_i$ auch globales Maximum von $x\mapsto v(x) - \hat{\fp} \bcdot x$ auf dem Definitionsbereich $\ffirst{\fs}$.
	Somit gilt $x^{(i)}\in\NB(v_i, \hat{\fp})$.
\end{proof}
\begin{theorem}
	Sei ein Markt mit unabhängigen Brutto-Substituts-Bewertungen und $\interior(K)\neq \emptyset$ für die Menge $K$ der Walras-Preise gegeben.
	Dann kann man entweder mit $\bigO(n\log(Mn))$ Abfragen des Aggregierte-Nachfrage-Orakels einen Walras-Preisvektor oder mit $\bigO(n^2 m S\log(Mn))$ Abfragen eines Wert-Orakels ein Walras-Gleichgewicht bestimmen.
\end{theorem}
\begin{proof}
	Hier kann man den gleichen Beweis wie in Theorem~\ref{thm-compute-walras-with-ellipsoid} führen; jedoch kann hier statt der Abschätzung der Nenner durch Lemma~\ref{lemma-prices-bounded-S} die Ganzzahligkeit der Ecken von $K$ nach Theorem~\ref{thm-walras-prices-integral-polytope} genutzt werden.
	Daher kann $R=1$ gewählt werden.
	
	Nutzt man statt des Aggregierte-Nachfrage-Orakels ein Wert-Orakel, so kann man nach Korollar~\ref{cor-indep-gs-matroidal} das gleiche bei $mnS$ so vielen Abfragen erreichen.
	Allerdings hat dies den Vorteil, dass aus folgendem Grund sogar eine Walras-Allokation berechnet wird:
	Das Trennorakels liefert bei der Ellipsoid-Methode schließlich die Meldung \glqq$\fp\in K$\grqq für Preise $\fp\in\R^n$, was nach dem Beweis von Theorem~\ref{thm-compute-walras-with-ellipsoid} nur erfolgt, wenn $\dem(\fp) - \fs = 0$ gilt.
	Da der Nachfrage-Vektor als Summe $\sum_{i\in\first{m}} \dem_i(\fp)$ berechnet wurde, bildet $(\dem_i(\fp))_{i\in\first{m}}$ eine Walras-Allokation.
\end{proof}

\clearpage          % neue Seite für Literaturverzeichnis

%%%%%%%%%%%%%%%%%%%%%%%%%%%%%%%%%%%%%%%%%%%%%%%%%%%%%%%%%%%%%%%%%%%%%%%%%%%%%%%%%%%%%%%%%%%%
% Literaturverzeichnis
\nocite*  % Nicht zitierte Quellen werden auch ins Literaturverzeichnis aufgenommen
\thispagestyle{empty}
\bibliography{literature/seminararbeit}  % Literaturverzeichnis liegt in der Datei seminararbeit

%%%%%%%%%%%%%%%%%%%%%%%%%%%%%%%%%%%%%%%%%%%%%%%%%%%%%%%%%%%%%%%%%%%%%%%%%%%%%%%%%%%%%%%%%%%%
%%%%%%%%%%%%%%%%%%%%%%%%%%%%%%%%%%%%%%%%%%%%%%%%%%%%%%%%%%%%%%%%%%%%%%%%%%%%%%%%%%%%%%%%%%%%
% Ende des Dokuments
\end{document}			
